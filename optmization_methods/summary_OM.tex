% \documentclass[a2paper, 10pt, twocolumn, headings=standardclasses, parskip=half]{scrartcl}
\documentclass[9pt, headings=standardclasses, parskip=half]{scrartcl}
\usepackage{ifthen}
\usepackage{iftex}


\usepackage[automark]{scrlayer-scrpage}
\clearpairofpagestyles
\ofoot{\pagemark} % für ein einseitiges Dokument: \ofoot platziert den Inhalt in der äußeren Fußzeile (unten rechts)
% \pagestyle{scrheadings}

% \renewcommand{\familydefault}{\sfdefault} % sans serif font for text (math font is still serif)

\usepackage{graphicx}
\usepackage[dvipsnames]{xcolor}
\usepackage{soul}

% \usepackage[left=20mm, right=20mm, top=20mm, bottom=30mm]{geometry}
% \usepackage[left=25mm, right=25mm, top=20mm, bottom=30mm]{geometry}
\usepackage[left=30mm, right=30mm, top=20mm, bottom=30mm]{geometry}
% \usepackage[left=43mm, right=43mm, top=20mm, bottom=30mm]{geometry}

\usepackage{amsmath, amssymb, amsthm}
\usepackage{mathtools}
\usepackage{mathdots} % without this package, only \vdots and \ddots are taken from the text font (not the math font), which looks bad if the text font is different from the math font
\DeclareMathOperator*{\argmax}{arg\,max}
\DeclareMathOperator*{\argmin}{arg\,min}

\usepackage{enumitem}
\renewcommand{\labelitemi}{\textbullet}
\renewcommand{\labelitemii}{\raisebox{0.1ex}{\scalebox{0.8}{\textbullet}}}
\renewcommand{\labelitemiii}{\raisebox{0.2ex}{\scalebox{0.6}{\textbullet}}}
\renewcommand{\labelitemiv}{\raisebox{0.3ex}{\scalebox{0.4}{\textbullet}}}

\usepackage{caption, subcaption}

\usepackage{pifont}


\usepackage{tikz}
\usetikzlibrary{arrows, arrows.meta, shapes, positioning, calc, fit, patterns, intersections, math, 3d, tikzmark, decorations.pathreplacing}
\usepackage{tikz-3dplot}
\usepackage{tikzpagenodes}
\usetikzlibrary{external}
\tikzexternalize[prefix=tikz/]
\usepackage{pgfplots}


% define colors
\definecolor{funblue}{rgb}{0.10, 0.35, 0.66}
\definecolor{alizarincrimsonred}{rgb}{0.85, 0.17, 0.11}
\definecolor{amethyst}{rgb}{0.6, 0.4, 0.8}
\definecolor{mypurple}{rgb}{128, 0, 128} 
\renewcommand{\emph}[1]{\textcolor{mypurple}{#1}}

\usepackage{thmtools}

\newlength{\thmspace}
\setlength{\thmspace}{3pt plus 1pt minus 1pt} 


\declaretheoremstyle[
  headfont=\bfseries,
  bodyfont=\normalfont,
  spaceabove=\thmspace,
  spacebelow=\thmspace,
  % qed=\ensuremath{\blacklozenge},
  qed=\ensuremath{\vartriangleleft},
  postheadspace=1em
]{assertionstyle}

\declaretheorem[
  style=assertionstyle,
  name=Theorem
]{theorem}

\declaretheorem[
  style=assertionstyle,
  name=Lemma,
  sibling=theorem
]{lemma}

\declaretheorem[
  style=assertionstyle,
  name=Corollary,
  sibling=theorem
]{corollary}

\declaretheorem[
  style=assertionstyle,
  name=Proposition,
  sibling=theorem
]{proposition}

\declaretheorem[
  style=assertionstyle,
  name=Conjecture,
  sibling=theorem
]{conjecture}

\declaretheorem[
  style=assertionstyle,
  name=Claim,
  sibling=theorem
]{claim}


\declaretheoremstyle[
  headfont=\bfseries,
  bodyfont=\normalfont,
  spaceabove=\thmspace,
  spacebelow=\thmspace,
  qed=\ding{45},
  postheadspace=1em
]{definitionstyle}

\declaretheorem[
  style=definitionstyle,
  name=Definition
]{definition}

% \declaretheoremstyle[
%   headfont=\bfseries,
%   bodyfont=\normalfont,
%   spaceabove=6pt,
%   spacebelow=6pt,
%   qed=\ensuremath{\blacktriangleleft},
%   postheadspace=1em
% ]{definitionstyle}

% \declaretheorem[
%   style=definitionstyle,
%   name=Definition
% ]{definition}


\declaretheoremstyle[
  headfont=\bfseries,
  bodyfont=\normalfont,
  spaceabove=6pt,
  spacebelow=6pt,
  qed=\ensuremath{\square},
  postheadspace=1em
]{proofstyle}
\let\proof\relax
\let\endproof\relax
\declaretheorem[
  style=proofstyle,
  name=Proof,
  numbered=no
]{proof}



\declaretheoremstyle[
  headfont=\bfseries\color{funblue},
  bodyfont=\normalfont\normalsize,
  spaceabove=\thmspace,
  spacebelow=\thmspace,
  qed=\ensuremath{\color{funblue}\blacktriangleleft},
  postheadspace=1em
]{examplestyle}

\declaretheorem[
  style=examplestyle,
  name=Example,
]{example}


\declaretheoremstyle[
  headfont=\bfseries,
  bodyfont=\normalfont\normalsize,
  spaceabove=\thmspace,
  spacebelow=\thmspace,
  qed=\ensuremath{\blacktriangleleft},
  postheadspace=1em
]{remarkstyle}

\declaretheorem[
  style=remarkstyle,
  name=Remark,
]{remark}

\declaretheoremstyle[
  headfont=\color{alizarincrimsonred}\bfseries,
  bodyfont=\normalfont\normalsize,
  spaceabove=\thmspace,
  spacebelow=\thmspace,
  qed=\ensuremath{\color{alizarincrimsonred}\blacktriangleleft},
  postheadspace=1em
]{cautionstyle}

\declaretheorem[
  style=cautionstyle,
  name=Caution,
  sibling=remark
]{caution}

\declaretheoremstyle[
  headfont=\bfseries,
  bodyfont=\normalfont\footnotesize,
  spaceabove=\thmspace,
  spacebelow=\thmspace,
  postheadspace=1em
]{smallremarkstyle}

\declaretheorem[
  style=smallremarkstyle,
  name=Remark,
  sibling=remark
]{smallremark}


\newenvironment{verticalhack}
  {\begin{array}[b]{@{}c@{}}\displaystyle}
  {\\\noalign{\hrule height0pt}\end{array}} % to make qed symbol aligned


\usepackage{algorithm}
\usepackage{algorithmicx}
\usepackage[italicComments=false]{algpseudocodex} % macht probleme mit TikZ externalization

% The algopseudocodex package uses TikZ internally. The easiest solution would be to disable the externalization for the algorithmic environment by doing
\AddToHook{env/algorithmic/begin}{\tikzexternaldisable} 


% commands for functions etc.
\newcommand{\im}{\operatorname{Im}}

% commands for vectors and matrices
\newcommand{\matr}[1]{\underline{\boldsymbol{#1}}}
\newcommand{\vect}[1]{\vec{\boldsymbol{#1}}}
% \newcommand{\vect}[1]{\vec{{#1}}\,}

% commands for derivatives
\newcommand{\dif}{\mathrm{d}}

% commands for number sets
\newcommand{\R}{\mathbb{R}}
\newcommand{\N}{\mathbb{N}}
\newcommand{\Z}{\mathbb{Z}}
\newcommand{\Q}{\mathbb{Q}}
\newcommand{\C}{\mathbb{C}}

% commands for probability
\newcommand{\Var}{\operatorname{Var}}
\newcommand{\Cov}{\operatorname{Cov}}
\newcommand{\Exp}{\operatorname{E}}
% \newcommand{\P}{\operatorname{P}} % this is already defined in amsmath/amsopn
\newcommand{\Prob}{\operatorname{P}}
\newcommand{\numof}{\ensuremath{\# \,}} % number of elements in a set
\newcommand{\blackheight}{\operatorname{bh}}

% \algnewcommand{\LeftComment}[1]{\(\triangleright\) #1}
\algnewcommand{\TO}{, \ldots ,}
\algnewcommand{\DOWNTO}{, \ldots ,}
\algnewcommand{\OR}{\vee}
\algnewcommand{\AND}{\wedge}
\algnewcommand{\NOT}{\neg}
\algnewcommand{\LEN}{\operatorname{len}}
\algnewcommand{\tru}{\ensuremath{\mathrm{\texttt{true}}}}
\algnewcommand{\fals}{\ensuremath{\mathrm{\texttt{false}}}}
\algnewcommand{\append}{\circ}

\algnewcommand{\nil}{\ensuremath{\mathrm{\textsc{nil}}}}
\algnewcommand{\red}{\ensuremath{\mathrm{\textsc{red}}}}
\algnewcommand{\black}{\ensuremath{\mathrm{\textsc{black}}}}
\algnewcommand{\gray}{\ensuremath{\mathrm{\textsc{gray}}}}
\algnewcommand{\white}{\ensuremath{\mathrm{\textsc{white}}}}

% \newcommand{\attribute}[1]{\ensuremath{\mathtt{#1}}}
\newcommand{\attrib}[2]{\ensuremath{#1\mathtt{.}\mathtt{#2}}} % object.field with field in math typewriter (like code)
\newcommand{\attribnormal}[2]{\ensuremath{#1\mathtt{.}#2}} 





\title{Optimization Methods}
% \author{\today}
% \date{}
\author{Fabian Bosshard}
\date{\today}



% \usepackage[colorlinks=true, linkcolor=blue, urlcolor=blue, citecolor=green]{hyperref}
% \usepackage{hyperref} % for printing use this version (without colorlinks)
\usepackage[
  colorlinks=false,        % don't wrap links in a colour
  pdfborder={0 0 0}        % no border around links
]{hyperref}
\usepackage{cleveref}




% -------------------------------------------------
% 2. load glossaries *after* hyperref
\usepackage[acronym,              % create an “acronym” glossary
            nomain,               % omit the main glossary (only acronyms)
            toc,                  % add list of acronyms to the ToC
            nonumberlist          % omit page list in the printed glossary
           ]{glossaries-extra}

% choose how the first appearance looks:
%   long-short  → “Karush–Kuhn–Tucker (KKT)”
%   short-long  → “KKT (Karush–Kuhn–Tucker)”
\setabbreviationstyle[acronym]{long-short}

% must be issued once *after* loading glossaries
\makeglossaries

% -------------------------------------------------
% 3. define your acronyms
\newacronym{kkt}{KKT}{Karush--Kuhn--Tucker conditions}
\newacronym{licq}{LICQ}{Linear Independence Constraint Qualification}
\newacronym{lp}{LP}{Linear Programming}
\newacronym{qp}{QP}{Quadratic Programming}








%%%~~~ special TikZ and pgfplots settings for gradient descent visualization ~~~%%%

\tikzset{
  arrowed/.style={
    decorate,
    decoration={
      show path construction,
      lineto code={
        \draw[#1, color=red, thick] (\tikzinputsegmentfirst) -- (\tikzinputsegmentlast);
      }
    }
  },
  arrowed/.default=-stealth
}

\pgfplotsset{
  gradient function/.initial=f,
  dx/.initial=0.01,
  dy/.initial=0.01
}

\pgfmathdeclarefunction{xgrad}{2}{%
  \begingroup
  \pgfkeys{/pgf/fpu,/pgf/fpu/output format=fixed}%
  \edef\myfun{\pgfkeysvalueof{/pgfplots/gradient function}}%
  \pgfmathparse{(\myfun(#1+\pgfkeysvalueof{/pgfplots/dx},#2)-\myfun(#1,#2))/\pgfkeysvalueof{/pgfplots/dx}}%
  \pgfmathsmuggle\pgfmathresult\endgroup
}

\pgfmathdeclarefunction{ygrad}{2}{%
  \begingroup
  \pgfkeys{/pgf/fpu,/pgf/fpu/output format=fixed}%
  \edef\myfun{\pgfkeysvalueof{/pgfplots/gradient function}}%
  \pgfmathparse{(\myfun(#1,#2+\pgfkeysvalueof{/pgfplots/dy})-\myfun(#1,#2))/\pgfkeysvalueof{/pgfplots/dy}}%
  \pgfmathsmuggle\pgfmathresult\endgroup
}

%%%~~~~~~~~~~~~~~~~~~~~~~~~~~~~~~~~~~~~~~~~~~~~~~~~~~~~~~~~~~~~~~~~~~~~~~~~~~~~~%%%



\begin{document}

\thispagestyle{empty}

\maketitle
\tableofcontents















\clearpage
\pagenumbering{arabic}  
\section{General formulation of an optimization problem}

main components:
\begin{itemize}
	\item \textbf{objective function} $f(\vect{x})$: function to be minimized or maximized
	\item \textbf{variables} $\vect{x} = (x_1, \ldots, x_n) \in \mathbb{R}^n$: quantities of the problem that can be adjusted and need to be chosen \emph{optimally}
	\item \textbf{constraint functions} $c_i(\vect{x})$: conditions that limit the possbile values of $\vect{x}$. they can be \emph{equality} or \emph{inequality} constraints
	\item \textbf{feasible region}: set of all variables $\vect{x}$ satisfying all constraints
\end{itemize}

general form of an optimization problem:
\begin{equation}
\label{eq:form_optimization_problem}
\boxed{
 \min_{\vect{x} \in \mathbb{R}^{n}} f(\vect{x}) \quad \text{subject to} \quad \begin{cases}c_{i}(\vect{x})=0, & i \in \mathcal{E} \\ c_{i}(\vect{x}) \geq 0, & i \in \mathcal{I}\end{cases}
 }
\end{equation}
with
\begin{itemize}
  \item \(f: \mathbb{R}^{n} \rightarrow \mathbb{R}\) the objective function
  \item \(\mathcal{E}\) the index set of equality constraints
  \item \(\mathcal{I}\) the index set of inequality constraints
\end{itemize}


\begin{remark}\leavevmode
  \begin{itemize}
    \item We do not loose generality by restricting our analysis to minimization problems, since maximizing \(f\) is \emph{equivalent to} minimizing \(-f\).
    \item We do not loose generality by considering constraints of the form in \eqref{eq:form_optimization_problem} since:
    \begin{itemize}
      \item a constraint of the form \(c_{i}(\vect{x})=b\) can be transformed into \(\bar{c}_{i}(\vect{x})=0\) with \(\bar{c}_{i}(\vect{x})=c_{i}(\vect{x})-b\).
      \item a constraint of the form \(c_{i}(\vect{x}) \leq 0\) can be transformed into \(\bar{c}_{i}(\vect{x}) \geq 0\) with \(\bar{c}_{i}(\vect{x})=-c_{i}(\vect{x})\).
    \end{itemize}
    \item If \(n=1\), the function \(f\) is \emph{univariate} and we also say that the optimization problem is univariate. If \(n=2\), it is a \emph{bivariate} problem, and if \(n \geq 2\), it is a \emph{multivariate} problem. \qedhere
  \end{itemize}
\end{remark}

A point \(\vect{x}\in\mathbb{R}^{n}\) satisfying all the constraints is called a \emph{feasible point} and the set of all feasible points is called the \emph{feasible set} (or, feasible region) of the optimization problem. It is formally defined as
\begin{equation}
  \label{eq:feasible_set}
\Omega = \left\{ \vect{x}\in\mathbb{R}^{n} : c_{i}(\vect{x})=0,\; i\in\mathcal{E},\; c_{i}(\vect{x})\geq 0,\; i\in\mathcal{I} \right\}
\end{equation}

This also implies that the problem \eqref{eq:form_optimization_problem} can be re-written as
\[
\min_{\vect{x}\in\Omega} f(\vect{x})
\]

The general formulation of optimization problems includes a large variety of problems and each category of problem requires a specific methodology. There is no universal optimization algorithm that is applicable to all problems. Thus, one needs to understand the specificity of each problem and algorithm to identify the most suitable methodology.




\clearpage
\section{Useful definitions and properties}

\tikzexternaldisable
\begin{theorem}[Cauchy--Schwarz Inequality]
  The \emph{Cauchy--Schwarz inequality} states that for any vectors \(\vect{x}\), \(\vect{y}\) of an inner product space,
  \(
  |\langle \vect{x}, \vect{y} \rangle|^{2} \leq \langle \vect{x}, \vect{x} \rangle \langle \vect{y}, \vect{y} \rangle
  %\text{,}
  \)
  where \(\langle \cdot, \cdot \rangle\) is the inner product. Or written in terms of the induced norm \(\|\cdot\|:=\sqrt{\langle \cdot, \cdot \rangle}\),
  \(
  |\langle \vect{x}, \vect{y} \rangle| \leq \|\vect{x}\| \|\vect{y}\|
  %\text{.}
  \)
  For \(\mathbb{R}^{n}\) and the dot product, this means
  \[
  \begin{aligned}
  |\vect{x} \cdot \vect{y}| &\leq \|\vect{x}\| \|\vect{y}\| \\
  \left|\sum_{i=1}^{n} x_{i}y_{i}\right| &\leq \sqrt{\sum_{i=1}^{n} x_{i}^{2}} \sqrt{\sum_{i=1}^{n} y_{i}^{2}} %\text{.} 
  \end{aligned}
  \]
\end{theorem}
The dot product can be expressed in terms of the angle \(\theta\) between the two vectors:
\[
\vect{x} \cdot \vect{y} = \vect{x}^{T}\vect{y} = \|\vect{x}\| \|\vect{y}\| \cos(\theta) %\text{.}
\]

% spectral norm of a matrix
\begin{definition}[Spectral Norm]\label{def:spectral_norm}
The \emph{spectral norm} of a matrix \(\matr{A}\in\mathbb{R}^{m\times n}\) is defined as
\[
\|\matr{A}\|_{2} := \max_{\vect{x} \neq \vect{0}} \frac{\|\matr{A}\vect{x}\|_{2}}{\|\vect{x}\|_{2}} = \max_{\|\vect{x}\|_{2}=1} \|\matr{A}\vect{x}\|_{2} = \sqrt{\lambda_{\max}(\matr{A}^{T}\matr{A})}\tikzmark{spectral_norm_general}
%\text{,}
\]
where \(\lambda_{\max}(\matr{A}^{T}\matr{A})\) is the largest eigenvalue of \(\matr{A}^{T}\matr{A} \in \mathbb{R}^{n\times n}\). 
If \(\matr{A}\) is symmetric, then the spectral norm is equal to the largest eigenvalue (in absolute value) of \(\matr{A}\),
\[
\|\matr{A}\|_{2} = \max_{i \in \{1,\ldots,n\}} |\lambda_{i}(\matr{A})|
\tikzmark{spectral_norm_symmetric}
%\text{,}
\]
where \(\lambda_{i}(\matr{A})\) is the \(i\)-th eigenvalue of \(\matr{A}\).
% \begin{tikzpicture}[remember picture, overlay]
%   \coordinate (A_spectral_norm_general) at ($ (pic cs:spectral_norm_general)+(4cm,0) $); % I want this to be on the same height as the text but "sticking" to the right margin
%   \node[left] at (A_spectral_norm_general) {\footnotesize general case};
%   \coordinate (A_spectral_norm_symmetric) at ($ (pic cs:spectral_norm_symmetric)+(7cm,0) $); % I want this to be on the same height as the text but "sticking" to the right margin
%   \node[left] at (A_spectral_norm_symmetric) {\footnotesize symmetric case};
%   \end{tikzpicture}
  % Define a helper coordinate from the pic cs:
\begin{tikzpicture}[remember picture, overlay]
  \coordinate (spectral_norm_general_y) at (pic cs:spectral_norm_general);
  \coordinate (A_spectral_norm_general_border) at (current page.east |- spectral_norm_general_y);
  \coordinate (A_spectral_norm_general_margin) at ($(A_spectral_norm_general_border) + 0.5*(\textwidth,0) + 0.5*(-\paperwidth, 0) $);
  %\node[anchor=east, inner sep = 0pt] at (A_spectral_norm_general_margin) {\color{red}\ding{43} general case};
  \coordinate (spectral_norm_symmetric_y) at (pic cs:spectral_norm_symmetric);
  \coordinate (A_spectral_norm_symmetric_border) at (current page.east |- spectral_norm_symmetric_y);
  \coordinate (A_spectral_norm_symmetric_margin) at ($(A_spectral_norm_symmetric_border) + 0.5*(\textwidth,0) + 0.5*(-\paperwidth, 0) $);
  \node[anchor=east, inner sep = 0pt] at (A_spectral_norm_symmetric_margin) {\color{red}\ding{43} if $\matr{A}^T=\matr{A}$};
\end{tikzpicture}
\end{definition}


\begin{definition}[Condition Number]
  The \emph{condition number} of a matrix \(\matr{A}\in\mathbb{R}^{m\times n}\) is defined as
  \[
  \kappa(\matr{A}) := \|\matr{A}\|_{2} \cdot \|\matr{A}^{-1}\|_{2} \geq 1
  \tikzmark{condition_number_general}
  %\text{.}
  \]
  If \(\matr{A}\) is symmetric positive definite, 
  \[
  \kappa(\matr{A}) = \frac{\lambda_{\max}(\matr{A})}{\lambda_{\min}(\matr{A})}
  \tikzmark{condition_number_spd}
  %\text{.}
  \]
  \begin{tikzpicture}[remember picture, overlay]
    \coordinate (A_condition_number_general) at (pic cs:condition_number_general);
    \coordinate (A_condition_number_general_border) at (current page.east |- A_condition_number_general);
    \coordinate (A_condition_number_general_margin) at ($(A_condition_number_general_border) + 0.5*(\textwidth,0) + 0.5*(-\paperwidth, 0) $);
    %\node[anchor=east, inner sep = 0pt] at (A_condition_number_general_margin) {\color{red}\ding{43} general case};
    \coordinate (A_condition_number_spd) at (pic cs:condition_number_spd);
    \coordinate (A_condition_number_spd_border) at (current page.east |- A_condition_number_spd);
    \coordinate (A_condition_number_spd_margin) at ($(A_condition_number_spd_border) + 0.5*(\textwidth,0) + 0.5*(-\paperwidth, 0) $);
    \node[anchor=east, inner sep = 0pt] at (A_condition_number_spd_margin) {\color{red}\ding{43} if $\matr{A} \succ 0$};
    \end{tikzpicture}
\end{definition}
\tikzexternalenable










\clearpage
\section{Classification of Problems}

Note that ``programming'' is used synonymously with ``optimization''.

\subsection{Linear vs Nonlinear Programming}

An optimization problem is said to be \emph{linear} if \(f\) is linear (or, affine) and the constraints are linear; otherwise, the problem is \emph{nonlinear}.

\begin{definition}[Linear Function]
A function \(f:\mathbb{R}^{n}\rightarrow\mathbb{R}\) is \emph{linear} if:
\begin{itemize}
	\item for any \(\vect{x},\vect{y}\in\mathbb{R}^{n}\), \(f(\vect{x}+\vect{y})=f(\vect{x})+f(\vect{y})\),
	\item for any \(\vect{x}\in\mathbb{R}^{n}\) and \(\lambda\in\mathbb{R}\), \(f(\lambda\vect{x})=\lambda f(\vect{x})\).
\end{itemize}
Equivalently, \(f\) is linear if it can be written as
\[
f(\vect{x})=\vect{c}^{T}\vect{x}=\sum_{i=1}^{n} c_{i}x_{i}
\]
with \(\vect{c}\in\mathbb{R}^{n}\).
\end{definition}

\begin{definition}[Affine Function]\label{def:affine_function}
A function \(f:\mathbb{R}^{n}\rightarrow\mathbb{R}\) is \emph{affine} if
\[
f(\alpha\vect{x}+(1-\alpha)\vect{y})=\alpha f(\vect{x})+(1-\alpha)f(\vect{y}),\quad \forall\, \vect{x},\vect{y}\in\mathbb{R}^{n},\; \forall\, \alpha\in[0,1]
\]
or equivalently, if there exist \(\vect{c}\in\mathbb{R}^{n}\) and \(b\in\mathbb{R}\) such that
\[
f(\vect{x})=\vect{c}^{T}\vect{x}+b\qedhere
\]
\end{definition}

\begin{example}
The following minimization problem is linear:
\[
\min_{\vect{x}\in\mathbb{R}^{n}} \sum_{i=1}^{n} c_{i}x_{i} \quad \text{subject to} \quad \sum_{i=1}^{n} x_{i}=1
\]
with \(c_{1}, \ldots, c_{n}\in\mathbb{R}\).
\end{example}

\subsection{Continuous vs Discrete Programming}

If the components of \(\vect{x}\) are real numbers and in an uncountable set, it is a \emph{continuous} optimization problem. Otherwise, it is a \emph{discrete} programming problem (if the variables can only be integers, it is an \emph{integer} programming problem).

\begin{example}
of a continuous as well as a discrete problem:
\begin{itemize}
	\item If \(x_{1},\ldots, x_{n}\) can take any value in \(\mathbb{R}\) or in \([a,b]\) with \(a<b\), then the problem is continuous.
	\item If \(x_{1},\ldots, x_{n}\) can take values in \(\mathbb{N}\) or \(\mathbb{Z}\), then the problem is discrete.\qedhere
\end{itemize}
\end{example}

\subsection{Convex vs Non-convex Optimization}

\begin{figure}[h]
% from https://tex.stackexchange.com/questions/302589/alignment-of-tikz-pictures-in-subfigures
\centering
\begin{subfigure}{0.2\textwidth}
    \centering
    \begin{tikzpicture}
    \draw[color=green!50!black, rotate=-45,fill=green!20] (0,0) ellipse (30pt and 45pt);
    \end{tikzpicture}
    \caption*{Convex set}
\end{subfigure}
% \hspace{0.1\textwidth}
% \hspace{0.1\linewidth}
\begin{subfigure}{0.2\textwidth}
    \centering
    \begin{tikzpicture}
    \useasboundingbox (-1,-1.35) rectangle (1.5,1.35);
    \draw[color=green!50!black, fill=green!20] (0,0) to [out=140,in=90] (-1,-1)
    to [out=-90,in=240] (0.8,-0.6)
    to [out=60,in=-60] (1.2,1.2)
    to [out=120,in=90] (0.3,0.7)
    to [out=-90,in=20] (0.3,0)
    to [out=200,in=-40] (0,0);
    \draw (-0.5,-0.5) -- (0.7,0.7);
    \fill (-0.5,-0.5) circle[radius=1.5pt];
    \fill (0.7,0.7) circle[radius=1.5pt];
    \end{tikzpicture}
    \caption*{Non-convex set}
\end{subfigure}
% \caption*{Graphical interpretation of convex sets.}
\label{fig:convexSet}
\end{figure}


An optimization problem is said to be \emph{convex} if the objective function \(f\) is convex, the equality constraints are linear, and the inequality constraints are concave. Equivalently, the problem is convex if the feasible set \(\Omega\) is convex.

\begin{definition}[Convex Set]\label{def:convex_set}
A set \(S\subset\mathbb{R}^{n}\) is \emph{convex} if \(\forall\,\vect{x},\vect{y}\in S,\; \forall\, t\in[0,1]\),
\[
t\vect{x}+(1-t)\vect{y}\in S
\qedhere
\]
\end{definition}

\begin{example}
A ball defined as
\[
\mathcal{B}_{r}(\vect{x}_{0})=\left\{ \vect{x}\in\mathbb{R}^{n} : \|\vect{x}-\vect{x}_{0}\|\leq r \right\}
\]
with center \(\vect{x}_{0}\in\mathbb{R}^{n}\) and radius \(r\), is convex.
\end{example}

Unless specified otherwise, the norm \(\|\cdot\|\) is the Euclidean norm:
\[
\|\vect{x}\|=\|\vect{x}\|_{2}=\sqrt{\sum_{i=1}^{n} x_{i}^{2}} 
\]



\begin{definition}[Convex Function]\label{def:convex_function}
A function \(f:S\rightarrow\mathbb{R}\) is \emph{convex} if its domain $S$ is convex and \(\forall\, \vect{x}, \vect{y} \in S\; \forall\, t\in[0,1]\),
\[
f\left(t\vect{x}+(1-t)\vect{y}\right)\leq t f(\vect{x})+(1-t)f(\vect{y})
\qedhere
\]
\end{definition}

Concavity is the opposite of convexity: \(f\) is \emph{concave} if \(-f\) is convex.

\begin{definition}[Strictly Convex Function]\label{def:strictly_convex_function}
A function \(f:S\rightarrow\mathbb{R}\) is \emph{strictly convex} if its domain $S$ is convex and \(\forall\, \vect{x}, \vect{y} \in S\; \forall\, t\in[0,1]\),
\[
f\left(t\vect{x}+(1-t)\vect{y}\right) < t\,f(\vect{x})+(1-t)\,f(\vect{y})
\qedhere
\]
\end{definition}

\begin{figure}[h]
\centering
\begin{tikzpicture}[scale=0.6]
    % Define the parabola function
    \draw[thick, name path=parabola, domain=-4:5, smooth, samples=100] plot (\x, {0.2*\x*\x - 1});
    % label the function
    % \node at (-4, 2.8) {$f(\, \cdot \,)$};
    \node at (5.2, 3.5) {$f$};

    % Define the secant line endpoints using intersections
    \path[name path=secant] (-4, 0) -- (4, 2);
    \path [name intersections={of=parabola and secant, by={A,B}}];

    % Draw secant line dynamically
    \draw (A) -- (B);

    % Mark points
    \fill (A) circle (2.5pt) node[left] {\footnotesize$(x, f(x))$};
    \fill (B) circle (2.5pt) node[right] {\footnotesize$(y, f(y))$};

\end{tikzpicture}
\caption*{Example of a convex function.}
\label{fig:convex_fun}
\end{figure}

\begin{definition}[\(\mu\)-Strongly Convex Function]\label{def:mu_strongly_convex_function}
A function \(f:S\rightarrow\mathbb{R}\) is \(\mu\)-\emph{strongly convex} (with \(\mu>0\)) if its domain $S$ is convex and \(\forall\, \vect{x}, \vect{y} \in S\; \forall\, t\in[0,1]\),
\[
f\left(t\vect{x}+(1-t)\vect{y}\right) \leq t\,f(\vect{x})+(1-t)\,f(\vect{y}) -\frac{\mu}{2}\,t(1-t)\|\vect{x}-\vect{y}\|^{2}
\qedhere
\]
\end{definition}

Observe that ``strong convexity'' $\Longrightarrow$ ``strict convexity'' $\Longrightarrow$ ``convexity''.

\begin{example}\label{ex:convex_functions}
The following examples and properties of convex functions are important:
\begin{itemize}
\item The exponential function is convex, while the logarithm function is concave. 
\item Powers of positive numbers $x^{\alpha}$ with $x>0$ are convex if $\alpha \geq 1$ or $\alpha \leq 0$, and concave if $0 \leq \alpha \leq 1$.
\item Affine functions are both convex and concave.
\item Absolute value and norms, e.g., $\ell_p$-norms are convex:
\begin{itemize}
  \item $\| \vect{x}\|_{p} = \left(\sum_{i=1}^{n}|x_{i}|^{p}\right)^{1/p}$
  \item $\| \vect{x}\|_{\infty}=\max\{|x_{1}|,\dots,|x_{n}|\}$
\end{itemize}
\item Positive part (Rectified Linear Unit) $\max\{x, 0\}$ is convex, negative part $\min\{0,x\}$ is concave.
\item Positive linear combinations of convex functions are convex.
\item If \(h\) is convex and \(g\) is convex and non-decreasing, then \(g\left(h(\vect{x})\right)\) is convex.\qedhere
\end{itemize}
\end{example}

\subsection{Constrained vs Unconstrained Optimization}

An optimization problem is \emph{constrained} if \(\mathcal{E}\) OR \(\mathcal{I}\) are NOT empty. It is \emph{unconstrained} if both \(\mathcal{E}\) AND \(\mathcal{I}\) are empty, i.e., \(\mathcal{E}=\mathcal{I}=\emptyset\).

\subsection{Smooth vs Non-smooth Optimization}
\label{subsec:smooth_non_smooth}
An optimization problem is said to be \emph{smooth} if the objective function \(f\) is smooth, i.e., $f$ is continuously differentiable and its derivative is Lipschitz continuous; otherwise, the problem is \emph{non-smooth}. 

\begin{definition}[Continuity]
A function \(f:\mathbb{R}\rightarrow\mathbb{R}\) is \emph{continuous} on $\mathbb{R}$ if for all \(x_{0}\in\mathbb{R}\),
\[
\lim_{x\to x_{0}}f(x)=f(x_{0})
\qedhere
\]
\end{definition}

\begin{definition}[Lipschitz Continuity]
A function \(f:\mathbb{R}\rightarrow\mathbb{R}\) is \(L\)-\emph{Lipschitz continuous} with \(L>0\) if for all \(x,y\in\mathbb{R}\),
\[
|f(x)-f(y)|\leq L|x-y|
\qedhere
\]
\end{definition}

Observe that ``Lipschitz continuity'' $\Longrightarrow$ ``continuity''.

\begin{definition}[Derivative]
The derivative of \(f:S\rightarrow\mathbb{R}\) at \(x\in S\subset\mathbb{R}\) is defined as
\[
f'(x)=\lim_{t\to 0}\frac{f(x+t)-f(x)}{t}=\frac{df}{dx}(x)
\qedhere
\]
\end{definition}

\begin{definition}[Differentiability]
A univariate function \(f:S\rightarrow\mathbb{R}\) with domain \(S\subseteq\mathbb{R}\) is \emph{differentiable} if \(S\) is an open set and the derivative \(f'(x)\) exists for every \(x\in S\). It is \emph{continuously differentiable} if \(f'\) is continuous.
\end{definition}


\begin{figure}[h]
  % from https://tex.stackexchange.com/questions/698164/draw-a-free-form-shape-and-mark-it-with-a-curved-line
  \centering
  \begin{tikzpicture}[
    >={Triangle},%                      arrow tip; try als {Stealth} etc.
    arr/.style={->,line width=1.5pt},%  arrows style
    UF/.style={color=green!50!black, dashed, fill=green!20},%   style of U-area
    circle_filling_style/.style={white!65},
    circle_border_style/.style={color=green!50!black, very thick},
    scale=0.7
  ]
  
  % ~~~ open set U ~~~~~~~~~~~~~~~~~~~
  \draw[UF] plot [smooth cycle,tension=1]
    coordinates {(0.5,.5)(0,2)(1.5,3)(2.5,5)(5,4)(7,3)(6,.5)(4,.5)};
  \node at (3,4) {\footnotesize $U$};
  
  % ~~~ δ-ball ~~~~~~~~~~~~~~~~~~~~~~~
  \coordinate (M) at (4.5,1.9); % saves typing
  \def\Radius{1.25}   % radius of the circle
  \fill[circle_filling_style] (M) circle (\Radius);
  \draw[fill=black,line width=0.8pt] (M) circle (1mm) node[left] {\footnotesize $\vect{x}$} -- node[above] {\footnotesize $\delta$} ++(30:\Radius);
  \node (B) at ($(M)+(-90:0.5*\Radius)$) {\footnotesize $\mathcal{B}_{\delta}(\vect{x})$};
  \draw[circle_border_style] (M) circle (\Radius);
  
  \end{tikzpicture}
  \caption*{Illustration of an open set.}
  \label{fig:open_set}
\end{figure}
\begin{definition}[Open Set]
An open set \(U\subseteq\mathbb{R}^{n}\) is a set that contains an open ball around each point, i.e.,
\[
\forall\, \vect{x}\in U, \quad \exists\, \delta>0, \quad \mathcal{B}_{\delta}(\vect{x})=\{ \vect{y}\in\mathbb{R}^{n} : \|\vect{y}-\vect{x}\|<\delta \} \subseteq U
\qedhere
\]
\end{definition}



\subsection{Global vs Local Optimization}\label{subsec:global_local_optimization}
We say that an optimization problem is solved globally if a \emph{global minimizer} is found; otherwise, if only a \emph{local minimizer} is found, the problem is solved locally. \hyperref[eq:feasible_set]{Recall} that the feasible set is denoted by \(\Omega\).

\begin{definition}[Global Minimizer]\label{def:global_minimizer}
A vector \(\vect{x}^{*}\in\Omega\) is a \emph{global minimizer} of \(f\) if
\[
f(\vect{x}^{*})\leq f(\vect{x}),\quad \forall\, \vect{x}\in\Omega \qedhere
\]
\end{definition}

\begin{definition}[Local Minimizer]\label{def:local_minimizer}
A vector \(\vect{x}^{*}\in\Omega\) is a \emph{local minimizer} of \(f\) if
\[
f(\vect{x}^{*})\leq f(\vect{x})
\]
for any \(\vect{x}\) in a neighborhood of \(\vect{x}^{*}\).
\end{definition}

\begin{definition}[Strict Local Minimizer]\label{def:strict_local_minimizer}
A vector \(\vect{x}^{*}\in\Omega\) is a \emph{strict local minimizer} of \(f\) if
\[
f(\vect{x}^{*})<f(\vect{x})
\]
for any \(\vect{x}\) in a neighborhood of \(\vect{x}^{*}\).
\end{definition}

\begin{definition}[Isolated Local Minimizer]
A vector \(\vect{x}^{*}\in\Omega\) is an \emph{isolated local minimizer} of \(f\) if it is the only local minimizer in a neighborhood of \(\vect{x}^{*}\).
\end{definition}


\begin{remark}
  An example for a neighborhood \(\mathcal{N}(\vect{x}^{*})\) is the open ball
\[
\mathcal{B}_{r}(\vect{x}^{*}) := \left\{ \vect{x}\in\mathbb{R}^{n} : \|\vect{x}-\vect{x}^{*}\|<r \right\}
\]
with some radius \(r>0\).
\end{remark}

\begin{remark}\leavevmode % or "\" or "\quad"
\begin{itemize}
  \item An isolated local minimizer is always a strict minimizer but the converse is not true: there are strict minimizers that are not isolated.
  \item A global minimizer can also be strict (by replacing \(\leq\) with \(<\) in Definition \ref{def:global_minimizer}). It can also be isolated (if there exists a neighborhood with no local minimizer).\qedhere
\end{itemize}
\end{remark}


\subsection{Stochastic vs Deterministic Optimization}

The distinction here refers to the type of algorithms used to solve the optimization problem. 
A stochastic optimization algorithm incorporates some randomness in its iterations. 
Such algorithm may find different local minimizers even if it is initialised at the same point.




%%%%%%%% ~ Week 2 ~ %%%%%%%%
%%%%%%%%%%%%%%%%%%%%%%%%%%%%

\clearpage
\section{Unconstrained Optimization}\label{sec:unconstrained_optimization}

\subsection{Univariate Objective Functions}
The general form of a univariate unconstrained optimization problem is
\[
\min_{x\in\mathbb{R}} f(x)
\]
with \(f:\mathbb{R}\to\mathbb{R}\) the objective function. Since there are no equality or inequality constraints, \(x\) may take any real value. 
In this course we restrict ourselves to \hyperref[subsec:smooth_non_smooth]{smooth} functions (see \ref{subsec:smooth_non_smooth}).

In this chapter (and later ones) we will see that the concept of derivatives is fundamental for solving optimization problems; in particular, derivatives allow us:
\begin{enumerate}
  \item to construct local approximating models (via Taylor's theorem),
  \item to characterise the solutions of the problem (i.e., by deriving optimality conditions).
\end{enumerate}

\subsubsection{Taylor Expansions for Univariate Functions}
We recall the definitions of the first and second derivatives for a univariate function.

\begin{definition}[First Derivative]
For a function \(f:\mathbb{R}\to\mathbb{R}\) and a point \(x\in\mathbb{R}\), the first (or first-order) derivative is defined as
\[
f'(x)=\lim_{d\to 0}\frac{f(x+d)-f(x)}{d}
\]
\end{definition}

\begin{definition}[Second Derivative]
The second (or second-order) derivative of \(f\) at \(x\) is defined as
\[
f''(x)=\lim_{d\to 0}\frac{f'(x+d)-f'(x)}{d}
\]
\end{definition}

Recall that \(f'(x)\) measures the rate of change (the slope of the tangent line) of \(f\) at \(x\). Thus for small \(d\) one has
\[
f(x+d)\approx f(x)+f'(x)\,d
\]
In fact, the tangent line at \(x\) is given by
\[
t(y;x)=f(x)+f'(x)(y-x),\quad y\in\mathbb{R}
\]
or, equivalently, writing \(y=x+d\),
\[
t(x+d;x)=f(x)+f'(x)\,d
\]
This linear approximation is known as the \emph{first-order Taylor approximation} of \(f\).

By incorporating the second derivative, one can construct a local quadratic model (the second-order Taylor approximation):
\begin{equation}
f(x+d)\approx q(x+d;x)=f(x)+f'(x)\,d+\frac{1}{2}f''(x)\,d^2,\quad d\in\mathbb{R}
\end{equation}
or equivalently,
\begin{equation}\label{eq:second_order_taylor_approximation_univariate}
f(y)\approx q(y;x)=f(x)+f'(x)(y-x)+\frac{1}{2}f''(x)(y-x)^2,\quad y\in\mathbb{R}
\end{equation}
Here, \(f''(x)\) is sometimes called the \emph{curvature} of \(f\) at \(x\). Note that these approximations are valid in a neighbourhood of \(x\) (i.e. for small \(d=y-x\)).

Taylor's theorem provides an exact expansion. 

\begin{theorem}[First and Second Order Taylor's Expansion for Univariate Functions]
\label{thm:univariate_taylor_expansion}
If \(f:\mathbb{R}\to\mathbb{R}\) is continuously differentiable, then for any \(x\in\mathbb{R}\) and \(d\in\mathbb{R}\) there exists \(t\in (0,1)\) such that
\[
f(x+d)=f(x)+f'\left(x+t\,d\right)d
\]
Moreover, if \(f\) is twice continuously differentiable, then for any \(x\in\mathbb{R}\) and \(d\in\mathbb{R}\) there exists \(t\in (0,1)\) such that
\[
f(x+d)=f(x)+f'(x)d+\frac{1}{2}f''\left(x+t\,d\right)d^2
\]
\end{theorem}

A useful consequence of Taylor's theorem is a bound on the approximation errors of the linear and quadratic models.

\begin{corollary}
\label{cor:approximation_error_univariate_taylor}  
If \(f\) is twice continuously differentiable then for any \(x\in\mathbb{R}\) there exist constants \(L>0\) and \(\epsilon>0\) such that
\[
\left|f(y)-t(y;x)\right|\le L\,(y-x)^2
\]
for all \(y\in\mathbb{R}\) with \(|y-x|\le\epsilon\). If \(f\) is three times continuously differentiable, there exist \(L>0\) and \(\epsilon>0\) such that
\[
\left|f(y)-q(y;x)\right|\le L\,|y-x|^3
\]
for all \(y\in\mathbb{R}\) with \(|y-x|\le\epsilon\).
\end{corollary}

That is, for \(|y-x|\) sufficiently small, the error of the linear model is of order \(O\left((y-x)^2\right)\) and the quadratic model approximates \(f\) with error of order \(O\left(|y-x|^3\right)\).


Actually Taylor's expansion does not stop at the second derivatives. It can also include higher-order derivatives of $f$ (if they exist) in order to construct increasingly better approximation of $f$:
$$
f(x+d) \approx f(x)+\frac{1}{1!} f^{\prime}(x) d+\frac{1}{2!} f^{\prime \prime}(x) d^2+\frac{1}{3!} f^{(3)}(x) d^3+\cdots+\frac{1}{k!} f^{(k)}(x) d^k
$$
Nonetheless in optimization we will essentially need the first and second order approximations.

\subsubsection{First and Second Order Optimality Conditions for Univariate Functions}
In unconstrained optimization we can derive ``recipes'' to characterise solutions of our problem. These ``recipes'' are optimality conditions.

\begin{theorem}[Necessary First-Order Optimality Condition for Univariate Functions]\label{thm:nec_first_order_optimality_univariate}
If \(f\) is continuously differentiable and \(x^*\) is a local minimizer of \(f\), then \(f'(x^*)=0\), i.e. \(x^*\) is a \emph{stationary point}.
\end{theorem}
\begin{proof}
(Contradiction)
Assume $f^{\prime}\left(x^*\right) \neq 0$. We will use the fact that around $x^*$ the tangent line approximates $f$ so that if its slope is not null, we can decrease $f$ by either decreasing or increasing $x^*$. Let $d:=-f^{\prime}\left(x^*\right)$. Thus $d f^{\prime}\left(x^*\right)<0$ and since $f^{\prime}$ is continuous there exist $t_0$ such that $\forall t \in\left[0, t_0\right]$
$$
d f^{\prime}\left(x^*+t d\right)<0
$$
Let $t \in\left[0, t_0\right]$. By the first part of  \hyperref[thm:univariate_taylor_expansion]{Taylor's theorem}, there exists $\bar{t} \in(0, t)$ such that
$$
f\left(x^*+t d\right)=f\left(x^*\right)+t d f^{\prime}\left(x^*+\bar{t} d\right)
$$
Since $\bar{t} \leq t \leq t_0$ we have $d f^{\prime}\left(x^*+\bar{t} d\right)<0$ which implies $f\left(x^*+t d\right)<f\left(x^*\right)$. Since one can choose $t$ arbitrarily close to 0 , this proves that $x^*$ is not a local minimizer.
\end{proof}


It is important to note that stationary points may correspond to local minimizers, local maximizers, or saddle points. Therefore, further analysis (using second derivatives) is often required.

\begin{theorem}[Necessary Second-Order Optimality Condition]
If \(f\) is twice continuously differentiable and \(x^*\) is a local minimizer of \(f\), then
\[
f''(x^*)\ge 0
\]
\end{theorem}
\begin{proof}
(Contradiction)
Assume $f^{\prime \prime}\left(x^*\right)<0$. Note that if $f^{\prime}\left(x^*\right) \neq 0$ we already proved that $x^*$ is NOT a local minimizer so we can assume that $f^{\prime}\left(x^*\right)=0$. Let $d \neq 0$. Since $f^{\prime \prime}\left(x^*\right)<0$, we have $f^{\prime \prime}\left(x^*\right) p^2<0$. By continuity of $f^{\prime \prime}$, there exists $t_0>0$ such that
$$
f^{\prime \prime}\left(x^*+t d\right) d^2<0, \quad \forall t \in\left(0, t_0\right)
$$
Let $t \in\left(0, t_0\right)$. By the second part of \hyperref[thm:univariate_taylor_expansion]{Taylor's theorem} there exists $\bar{t} \in(0, t)$ such that
$$
f\left(x^*+t d\right)=f(x)+\frac{1}{2} f^{\prime \prime}(x+\bar{t} d) t^2 d^2
$$


Since $\bar{t} \leq t \leq t_0, f^{\prime \prime}(x+\bar{t} d)<0$, which implies that $f\left(x^*+t d\right)<f\left(x^*\right)$. As before we can conclude that $x^*$ is NOT a local minimizer.
\end{proof}

However, the above conditions are necessary but not sufficient. For example, if \(f'(x)=0\) and \(f''(x)=0\) the test is inconclusive.

A sufficient condition is obtained by strengthening the second-order condition.

\begin{theorem}[Sufficient Optimality Conditions]
If \(f\) is twice continuously differentiable, \(f'(x^*)=0\) and \(f''(x^*)>0\), then \(x^*\) is a strict local minimizer of \(f\).
\end{theorem}
\begin{proof}
Since \(f''\) is continuous, there exists \(r>0\) such that for all \(d\) with \(|d|<r\) we have \(f''(x^*+d)>0\). 
Then by the second-order Taylor expansion, for all such \(d\) (with \(d\ne0\)) there exists \(t\in(0,1)\) such that
\[
f(x^*+d)=f(x^*)+\frac{1}{2}f''\left(x^*+t d\right)d^2>f(x^*)
\]
Thus \(x^*\) is a strict local minimizer.
\end{proof}

\begin{remark}
Note that the sufficient condition is not necessary; for instance, \(f(x)=x^4\) has a strict local minimizer at \(x=0\) despite \(f''(0)=0\).
\end{remark}

\subsubsection{Minimizers for Convex Univariate Functions}
Identifying minimizers is considerably simpler when \(f\) is \hyperref[def:convex_function]{convex} (Definition \ref{def:convex_function}). 
If \(f\) is twice differentiable, the following characterisations are equivalent:
\[
\begin{aligned}
f\text{ is convex} \quad \Longleftrightarrow & \quad f''(x)\ge0,\quad \forall x\in\mathbb{R}\\[1mm]
\Longleftrightarrow & \quad f(y)\ge f(x)+f'(x)(y-x),\quad \forall x,y\in\mathbb{R}
\end{aligned}
\]

Two important properties follow immediately:

\begin{proposition}
If \(f\) is convex, any local minimizer is a global minimizer.
\end{proposition}

\begin{proposition}
If \(f\) is convex and differentiable, then any stationary point is a global minimizer.
\end{proposition}


\subsection{Multivariate Objective Functions}

The general form of a multivariate unconstrained optimization problem is
\[
\min_{\vect{x}\in\mathbb{R}^n} f(\vect{x})
\]
with \(f:\mathbb{R}^n\to\mathbb{R}\) the objective function, i.e. its argument is a vector with \(n\) coordniates:
\[
f(\vect{x})=f(x_1, \ldots, x_n)
\]
We will again assume that \(f\) is \hyperref[subsec:smooth_non_smooth]{smooth} (see \ref{subsec:smooth_non_smooth}):
A function \(f:\mathbb{R}^n\to\mathbb{R}\) is said to be {\color{red}\(L\)-smooth} if it is continuously differentiable and its gradient \(\nabla f\) is Lipschitz continuous with Lipschitz constant \(L>0\):
\begin{equation}
  \label{eq:L-smoothness_multivariate}
  {\color{Red}
  \setlength{\fboxrule}{1pt}
  \boxed{ 
  \color{black}
\|\nabla f(\vect{x})-\nabla f(\vect{y})\|\le L\|\vect{x}-\vect{y}\|,\quad \forall\, \vect{x},\vect{y}\in\mathbb{R}^n %\text{.}
  }
  }
\end{equation}
If \(f\) is twice continuously differentiable, this is equivalent to \(\|\matr{H}_f(\vect{x})\|_{2}\le L\) for all \(\vect{x}\in\mathbb{R}^n\), where \(\|\cdot\|_{2}\) denotes the spectral norm (Definition \ref{def:spectral_norm}).
Additionally, we assume that \(f\) is bounded from below, i.e., there exists \(M\in\mathbb{R}\) such that \(f(\vect{x})\ge M\) for all \(\vect{x}\in\mathbb{R}^n\).





%%%%%%%% ~ Week 3 ~ %%%%%%%%
%%%%%%%%%%%%%%%%%%%%%%%%%%%%


\subsubsection{First Order Derivatives for Multivariate Functions}
The first-order derivative of a multivariate function is given by its \emph{gradient}. To define the gradient we first recall partial derivatives and introduce the canonical basis.

We denote by \(\{\vect{e}_{i}\}_{i=1,\ldots,n}\) the canonical basis of \(\mathbb{R}^{n}\) where
\[
\vect{e}_{i}=\begin{pmatrix}
0\\
\vdots\\
1\\
\vdots\\
0
\end{pmatrix}\quad \text{(with the \(i\)-th coordinate equal to 1)}
\]

\begin{definition}[Partial Derivative]\label{def:partial_derivative}
Let \(f:\mathbb{R}^{n}\to\mathbb{R}\). The partial derivative of \(f\) with respect to the \(i\)-th coordinate at a point \(\vect{x}\in\mathbb{R}^{n}\) is defined as
\[
\frac{\partial f}{\partial x_{i}}(\vect{x})
=\lim_{h\to0}\frac{f\bigl(\vect{x}+h\,\vect{e}_{i}\bigr)-f(\vect{x})}{h}
=\lim_{h\to0}\frac{f(x_{1},\ldots,x_{i}+h,\ldots,x_{n})-f(x_{1},\ldots,x_{i},\ldots,x_{n})}{h}
\]
\end{definition}

\begin{definition}[Gradient]\label{def:gradient}
Let \(f:\mathbb{R}^{n}\to\mathbb{R}\). The \emph{gradient} of \(f\) at \(\vect{x}\) is the vector
\[
\nabla f(\vect{x})
=\begin{pmatrix}
\frac{\partial f}{\partial x_{1}}(\vect{x})\\%[1mm]
%\frac{\partial f}{\partial x_{2}}(\vect{x})\\[1mm]
\vdots\\%[1mm]
\frac{\partial f}{\partial x_{n}}(\vect{x})
\end{pmatrix}\in\mathbb{R}^{n}
\]
\end{definition}
The gradient gives the direction of steepest increase of \(f\) at \(\vect{x}\) and each coordinate quantifies the rate of change of \(f\) in the corresponding direction.
e say that \(f\) is differentiable if \(\nabla f(\vect{x})\) exists for all \(\vect{x}\), and continuously differentiable if \(\nabla f\) is continuous.

% Geometrically, for bivariate functions the gradient vector is orthogonal to the level curves (contours) of \(f\) and points in the direction of steepest ascent.

The gradient can sldo be used to compute the change in any direction. If we define the univariate function
\[
g(t)=f\left(\vect{x}+t\,\vect{d}\right), \quad t\in\mathbb{R}%\text{,}
\]
for a direction \(\vect{d}\in\mathbb{R}^{n}\), then by the chain rule we have
\[
g'(t) = \frac{d}{dt}g(t) = \frac{d}{dt}\left(f\left(\vect{x}+t\,\vect{d}\right)\right) = \nabla f\left(\vect{x}+t\,\vect{d}\right)^{T}\vect{d}
\]
Thus, the \hypertarget{directional_derivative}{\emph{directional derivative}} of \(f\) at \(\vect{x}\) in the direction \(\vect{d}\) is given by
\[
g'(0)=\nabla f(\vect{x})^{T}\vect{d}
\]
Among all unit vectors \(\vect{d}\) (i.e. with \(\|\vect{d}\|=1\)), the maximum value of \(g'(0)\) is attained when \(\vect{d}\) is aligned with \(\nabla f(\vect{x})\).

\subsubsection{First Order Taylor's Expansion for Multivariate Functions}
A first-order Taylor expansion (or linear local approximation) of \(f\) at \(\vect{x}\) is
\[
t(\vect{y};\vect{x})=f(\vect{x})+\nabla f(\vect{x})^{T}(\vect{y}-\vect{x}),\quad \vect{y}\in\mathbb{R}^{n}
\]
That is,
\[
f(\vect{x}+\vect{d})\approx f(\vect{x})+\nabla f(\vect{x})^{T}\vect{d}
\]

\begin{theorem}[First Order Taylor's Expansion for Multivariate Functions]
\label{thm:taylor_multivariate_first_order}
Let \(f:\mathbb{R}^{n}\to\mathbb{R}\) be continuously differentiable. Then for any \(\vect{x}\in\mathbb{R}^{n}\) and \(\vect{d}\in\mathbb{R}^{n}\) there exists \(t\in(0,1)\) such that
\[
f(\vect{x}+\vect{d})=f(\vect{x})+\nabla f(\vect{x}+t\,\vect{d})^{T}\vect{d}
\]
% Compare this to the \hyperref[thm:univariate_taylor_expansion]{univariate case}.
If we set \(n=1\) we recover the univariate case (Theorem~\ref{thm:univariate_taylor_expansion}).
\end{theorem}

\begin{corollary}
\label{cor:taylor_error_multivariate_first_order}
If \(f:\mathbb{R}^{n}\to\mathbb{R}\) is twice continuously differentiable, then for any \(\vect{x}\in\mathbb{R}^{n}\) there exist constants \(L>0\) and \(\epsilon>0\) such that for all \(\vect{y}\) in
\[
\mathcal{B}_{\epsilon}\left(\vect{x}\right)=\left\{\bar{\vect{x}}\in\mathbb{R}^{n}:\left\|\bar{\vect{x}}-\vect{x}\right\|\le\epsilon\right\}
\]
the error in the linear approximation satisfies
\[
\bigl|f(\vect{y})-t(\vect{y};\vect{x})\bigr|\le L\|\vect{y}-\vect{x}\|^{2}
\]
If we set \(n=1\) we recover the univariate case (Corollary~\ref{cor:approximation_error_univariate_taylor}).
\end{corollary}


\subsubsection{First Order Optimality Conditions for Multivariate Functions}
\hyperref[subsec:global_local_optimization]{Recall} the definitions for local and global minimizers (Definitions~\ref{def:local_minimizer} and \ref{def:global_minimizer}).

\begin{theorem}[Necessary First-Order Optimality Condition]\label{thm:nec_first_order_optimality_multivariate}
If \(f\) is continuously differentiable and \(\vect{x}^{*}\) is a local minimizer of \(f\) then 
\(\nabla f(\vect{x}^{*})=\vect{0}\), i.e. \(\vect{x}^{*}\) is a \emph{stationary point}. If we set \(n=1\) we recover the \hyperref[thm:nec_first_order_optimality_univariate]{univariate case} (Theorem~\ref{thm:nec_first_order_optimality_univariate}).
\end{theorem}
\begin{proof}
(Contradiction)
Assume that \(\nabla f(\vect{x}^{*})\neq \vect{0}\). Define the direction
\(
\vect{d}=-\nabla f(\vect{x}^{*})
\)
Consider the univariate function
\[
g(t)=f(\vect{x}^{*}+t\,\vect{d}),\quad t\in\mathbb{R}
\]
Then,
\(
g'(0)=\nabla f(\vect{x}^{*})^{T}\vect{d}=-\|\nabla f(\vect{x}^{*})\|^{2}<0
\)
Thus, for sufficiently small \(t>0\) we have \(g(t)<g(0)\), meaning
\[
f(\vect{x}^{*}+t\,\vect{d})<f(\vect{x}^{*})
\]
This contradicts the local minimality of \(\vect{x}^{*}\). Hence, it must be that \(\nabla f(\vect{x}^{*})=\vect{0}\).
\end{proof}

%\clearpage
\subsection{Gradient Descent}
\label{subsec:gradient_descent}
% from https://tex.stackexchange.com/questions/544796/plot-gradient-descent
\begin{center}
\begin{tikzpicture}[scale=0.9]
  \begin{axis}[
    hide axis,
    clip bounding box=upper bound,
    colormap/viridis,
    width=0.7*\linewidth,
    height=0.66*0.7*\linewidth, % or adjust proportionally
    declare function={
      f(\x,\y)=cos(deg(\x)*0.8)*cos(deg(\y)*0.6)*exp(0.1*\x);
    }
  ]

    % Plot the surface
    \addplot3[
      surf,
      shader=faceted,
      opacity=0.7,
      samples = 40,
      samples y = 40,
      domain=-4:4,
      y domain=-6:6,
    ] {f(x,y)};
    
    % Gradient descent path parameters
    \pgfmathsetmacro{\xcur}{-0.1}
    \pgfmathsetmacro{\ycur}{-0.8}
    \pgfmathsetmacro{\step}{-2}
    \pgfmathsetmacro{\zcur}{f(\xcur,\ycur)}
    \edef\pathCoords{(\xcur,\ycur,\zcur)}
    
    % Compute gradient descent steps
    \pgfplotsforeachungrouped \i in {0,...,6} {
      \pgfmathsetmacro{\xcur}{\xcur+\step*xgrad(\xcur,\ycur)}
      \pgfmathsetmacro{\ycur}{\ycur+\step*ygrad(\xcur,\ycur)}
      \pgfmathsetmacro{\zcur}{f(\xcur,\ycur)}
      \edef\pathCoords{\pathCoords\space (\xcur,\ycur,\zcur)}
    }
    
    % Plot the gradient descent path with an arrow style
    \addplot3[samples y=0,arrowed] coordinates \pathCoords;
  \end{axis}
\end{tikzpicture}
\end{center}

Since the gradient \(\nabla f(\vect{x})\) points in the direction of steepest increase, the negative gradient \(-\nabla f(\vect{x})\) gives the direction of steepest descent. 
Recall that for any direction \(\vect{d}\in\mathbb{R}^{n}\) the \hyperlink{directional_derivative}{directional derivative} of \(f\) at \(\vect{x}\) in the direction \(\vect{d}\) is
\(
g'(0)=\nabla f(\vect{x})^{T}\vect{d}
\). We can write 
\[
\nabla f(\vect{x})^{T}\vect{d} = \cos(\theta)\|\nabla f(\vect{x})\| \|\vect{d}\| 
\]
where \(\theta\) is the angle between \(\vect{d}\) and \(\nabla f(\vect{x})\).
Among all unit directions (i.e. \(\|\vect{d}\|=1\)), the decrease is maximized when \(\cos(\theta)=-1\), i.e. when \(\vect{d}\) is collinear with \(-\nabla f(\vect{x})\), but points in the opposite direction. 
Since it is of unit norm, we have
\[
\vect{d}=-\frac{\nabla f(\vect{x})}{\|\nabla f(\vect{x})\|}
\] 
which is called the \emph{direction of steepest descent}.

A gradient descent algorithm then generates a sequence \(\{\vect{x}^{(k)}\}\) via
\[
\vect{x}^{(k+1)}=\vect{x}^{(k)}-\alpha\,\nabla f\bigl(\vect{x}^{(k)}\bigr)
\]
where \(\alpha>0\) is the step size. 
A common stopping rule is to terminate when \(\|\nabla f(\vect{x}^{(k)})\|\) falls below a prescribed tolerance level \(\epsilon\) or after a maximum number of iterations \(k_{\max}\).

\subsubsection{Gradient Descent Algorithm}
\begin{algorithm}[H]
\caption{Gradient Descent}\label{alg:gd}
\begin{algorithmic}[1]
    \State \textbf{Input:} Starting point \(\vect{x}^{(0)}\), step size \(\alpha>0\), tolerance \(\epsilon>0\), and maximum iterations \(k_{\max}\)
    \State Set \(k\gets0\)
    \State Compute \(\vect{g}^{(0)}=\nabla f\bigl(\vect{x}^{(0)}\bigr)\)
    \While{\(\|\vect{g}^{(k)}\|>\epsilon\) and \(k<k_{\max}\)}
        \State Update: \(\vect{x}^{(k+1)}=\vect{x}^{(k)}-\alpha\,\vect{g}^{(k)}\)
        \State Set \(k\gets k+1\)
        \State Compute \(\vect{g}^{(k)}=\nabla f\bigl(\vect{x}^{(k)}\bigr)\)
    \EndWhile
    \State \textbf{Output:} \(\vect{x}^{(k)}\)
\end{algorithmic}
\end{algorithm}

\subsubsection{Convergence Analysis}
\label{subsec:gd_convergence}
\begin{theorem}[Global Convergence of Gradient Descent]
\label{thm:gd_conv}
Let \(f\) be twice continuously differentiable and \(L\)-smooth (i.e. its gradient is Lipschitz continuous with constant \(L>0\)). Assume that \(f\) is bounded from below by \(m\) and let \( 0<\alpha<\frac{2}{L}\).
Then for any starting point \(\vect{x}^{(0)}\), the iterates \(\{\vect{x}^{(k)}\}\) produced by Algorithm~\ref{alg:gd} satisfy
\begin{equation}
\label{eq:gd_conv}
\lim_{k\to\infty}\nabla f\bigl(\vect{x}^{(k)}\bigr)=\vect{0}
\end{equation}
\end{theorem}

\begin{proof}
Since \(f\) is \(L\)-smooth, for any \(\vect{x},\vect{x}'\in\mathbb{R}^{n}\) one has
\begin{equation}
\label{eq:L_smooth_gradient}
f(\vect{x}')\le f(\vect{x})+\nabla f(\vect{x})^{T}(\vect{x}'-\vect{x})+\frac{L}{2}\|\vect{x}'-\vect{x}\|^{2}
\end{equation}
Apply this inequality with \(\vect{x}=\vect{x}^{(k)}\) and \(\vect{x}'=\vect{x}^{(k+1)}=\vect{x}^{(k)}-\alpha\,\nabla f\bigl(\vect{x}^{(k)}\bigr)\). Then
\begin{align}
f\bigl(\vect{x}^{(k+1)}\bigr)
&\le f\bigl(\vect{x}^{(k)}\bigr)
-\alpha\,\|\nabla f\bigl(\vect{x}^{(k)}\bigr)\|^{2}
+\frac{\alpha^{2}L}{2}\|\nabla f\bigl(\vect{x}^{(k)}\bigr)\|^{2} \nonumber \\
&= f\bigl(\vect{x}^{(k)}\bigr)-\underbrace{\left(\alpha-\frac{\alpha^{2}L}{2}\right)}_{=: \beta}\|\nabla f\bigl(\vect{x}^{(k)}\bigr)\|^{2} \label{eq:gd_Lsmooth}
\end{align}
Since \(\alpha < \frac{2}{L}\), we have \(\beta = \alpha - \frac{\alpha^{2}L}{2} = \alpha\left(1-\frac{\alpha L}{2}\right) > 0\). Then,
\begin{equation}
\label{eq:gd_progress}
f\bigl(\vect{x}^{(k+1)}\bigr)\le f\bigl(\vect{x}^{(k)}\bigr)-\beta\|\nabla f\bigl(\vect{x}^{(k)}\bigr)\|^{2}
\end{equation}
We can sum these inequalities until iteration \(k\) (\emph{accumulation of progress}):
\[
f\left(\vect{x}^{(k)}\right) \le f\left(\vect{x}^{(0)}\right) - \beta\sum_{i=0}^{k-1}\left\|\nabla f\left(\vect{x}^{(i)}\right)\right\|^{2}
\]
Since \(f\) is bounded, 
\(
m \le f(\vect{x}^{(k)}) \le f(\vect{x}^{(0)}) - \beta\sum_{i=0}^{k-1}\|\nabla f(\vect{x}^{(i)})\|^{2}
\).
Therefore 
\(
\sum_{i=0}^{k-1}\|\nabla f(\vect{x}^{(i)})\|^{2} < \infty
\).
But this implies that
\(
\lim_{k\to\infty}\|\nabla f\bigl(\vect{x}^{(k)}\bigr)\|=0
\), which in turn implies~\eqref{eq:gd_conv}.
\end{proof}

\begin{remark}\
\begin{itemize}
\item Using the update inequality \eqref{eq:gd_progress}, the best step size \(\alpha^{*} \in \left(0,\frac{2}{L}\right)\) is obtained by maximizing
\(
\beta(\alpha)=\alpha-\frac{\alpha^{2}L}{2}
\)
\[
\beta'(\alpha) = 1 - \alpha L \overset{!}{=} 0 \Rightarrow \alpha^{*} = \frac{1}{L}
\]
\item It is a \emph{global} convergence result, since it is valid for any starting point \(\vect{x}^{(0)}\).
\item It does NOT necessarily imply that the iterates \(\{\vect{x}^{(k)}\}\) converge.
\item Often, \(L\) is unknown and we need to carefully tune the step size \(\alpha\).
\end{itemize}
\end{remark}


\begin{definition}[Global Convergence]
An optimization algorithm is globally convergent if, from any starting point \(\vect{x}^{(0)}\), its iterates \(\vect{x}^{(k)}\) satisfy
\[
\nabla f\left(\vect{x}^{(k)}\right) \xrightarrow[k \rightarrow \infty]{} \vect{0}
\]
\end{definition}

\begin{definition}[Convergence of Iterates]
We say that the sequence of iterates \(\{\vect{x}^{(k)}\}\) converges to \(\vect{x}^{*}\) if
\[
\|\vect{x}^{(k)}-\vect{x}^{*}\| \xrightarrow[k \rightarrow \infty]{} 0
\]
\end{definition}

%%%%%%%%%%%%%%%%%%%%%%%%%%%%%%%%%%%%%%%%%%%%%%%%%%%%%%%%%%%%%%%%%%%%%%%%%%%%%%%
\subsubsection{Convergence Rate for Strongly Convex Functions}
\label{subsec:gd_strongly_convex}
% \begin{lemma}
% \label{lem:optimality_error_strongly_convex}
% If \(f\) is differentiable, being \hyperref[def:mu_strongly_convex_function]{\(\mu\)-strongly convex} (Definition~\ref{def:mu_strongly_convex_function}) is equivalent to
% \begin{equation}
% \label{eq:strong_convex_differentiable_inequality} % arrow should start here
% \llap{\textbf{!}\enspace}
% {\color{Red}
% \setlength{\fboxrule}{1pt}
% \boxed{ 
% \color{black}
% f(\vect{y})\ge f(\vect{x})+\nabla f(\vect{x})^{T}(\vect{y}-\vect{x})
% +\frac{\mu}{2}\|\vect{y}-\vect{x}\|^{2},%
% }
% }
% \end{equation}
% and we have the inequality
% \begin{equation}
% \label{eq:optimality_error_inequality}
% {\color{Red}
% \setlength{\fboxrule}{1pt}
% \boxed{ 
% \color{black}
% 2\mu\Bigl(f(\vect{x})-f(\vect{x}^{*})\Bigr)\le\|\nabla f(\vect{x})\|^{2},\quad\forall\,\vect{x}\in\mathbb{R}^{n},
% }
% }
% \end{equation}
% where \(\vect{x}^{*}\) is the unique global minimizer of \(f\). 
% \end{lemma}
% Equation \eqref{eq:optimality_error_inequality} says that the gradient norm \(\|\nabla f(\vect{x})\|\) grows with the \emph{optimality error} \(f(\vect{x})-f(\vect{x}^{*})\).
% This allows us to quantify how close we are from the global minimum by looking at the gradient norm.
% \begin{proof}
% Rearranging equation \eqref{eq:strong_convex_differentiable_inequality} with \(\vect{y}=\vect{x}^{*}\) yields
% \begin{align*}
% % f(\vect{x})+\nabla f(\vect{x})^{T}(\vect{x}^{*}-\vect{x}) + \frac{\mu}{2}\|\vect{x}^{*}-\vect{x}\|^{2} &\le f(\vect{x}^{*}) \\
% f(\vect{x})-f(\vect{x}^{*}) &\le -\nabla f(\vect{x})^{T}(\vect{x}^{*}-\vect{x}) - \frac{\mu}{2}\|\vect{x}^{*}-\vect{x}\|^{2} \\ % arrow should end here
% 2\mu\left(f(\vect{x})-f(\vect{x}^{*})\right) &\le - 2\mu\nabla f(\vect{x})^{T}(\vect{x}^{*}-\vect{x}) - \mu^{2}\|\vect{x}^{*}-\vect{x}\|^{2}. \\
% \end{align*}
% Adding \(0 = \|\nabla f(\vect{x})\|^{2} - \|\nabla f(\vect{x})\|^{2}\) to the right-hand side, allows us to rewrite the last line as
% \begin{align*}
% 2\mu\left(f(\vect{x})-f(\vect{x}^{*})\right) &\le  \|\nabla f(\vect{x})\|^{2} - \|\nabla f(\vect{x})\|^{2} - 2\mu\nabla f(\vect{x})^{T}(\vect{x}^{*}-\vect{x}) - \mu^{2}\|\vect{x}^{*}-\vect{x}\|^{2} \\
% &= \|\nabla f(\vect{x})\|^{2} - \left(\|\nabla f(\vect{x})\|^2 + 2\mu\nabla f(\vect{x})^{T}(\vect{x}^{*}-\vect{x}) + \mu^{2}\|\vect{x}^{*}-\vect{x}\|^{2}\right) \\
% &= \|\nabla f(\vect{x})\|^{2} - \|\nabla f(\vect{x}) + \mu(\vect{x}^{*}-\vect{x})\|^{2} \\
% &\le \|\nabla f(\vect{x})\|^{2}.
% \end{align*}
% \end{proof}
\newlength\letterheight
\settoheight{\letterheight}{A}
\newlength\letterdepth
\settoheight{\letterdepth}{g} 
\tikzexternaldisable
\begin{lemma}
\label{lem:optimality_error_strongly_convex}
If \(f\) is differentiable, being \hyperref[def:mu_strongly_convex_function]{\(\mu\)-strongly convex} (Definition~\ref{def:mu_strongly_convex_function}) is equivalent to
\begin{equation}
\label{eq:strong_convex_differentiable_inequality} % arrow should start here
\llap{\textbf{!}\enspace}
{\color{Red}
\setlength{\fboxrule}{1pt}
\boxed{ 
\color{black}
f(\vect{y})\ge f(\vect{x})+\nabla f(\vect{x})^{T}(\vect{y}-\vect{x})
+\frac{\mu}{2}\|\vect{y}-\vect{x}\|^{2}\tikzmark{mu_strong_convex}%
}
}
\end{equation}
and we have the inequality
\begin{equation}
\label{eq:optimality_error_inequality}
{\color{Red}
\setlength{\fboxrule}{1pt}
\boxed{ 
\color{black}
2\mu\Bigl(f(\vect{x})-f(\vect{x}^{*})\Bigr)\le\|\nabla f(\vect{x})\|^{2},\quad\forall\,\vect{x}\in\mathbb{R}^{n}
}
}
\end{equation}
where \(\vect{x}^{*}\) is the unique global minimizer of \(f\). 
\end{lemma}
\tikzmark{para1A}Equation \eqref{eq:optimality_error_inequality} says that the gradient norm \(\|\nabla f(\vect{x})\|\) grows with the \emph{optimality error} \(f(\vect{x})-f(\vect{x}^{*})\).
This allows us to quantify how close we are from the global minimum by looking at the gradient norm.\tikzmark{para1B}
\begin{proof}
Rearranging equation \eqref{eq:strong_convex_differentiable_inequality} with \(\vect{y}=\vect{x}^{*}\) yields
\begin{align*}
% f(\vect{x})+\nabla f(\vect{x})^{T}(\vect{x}^{*}-\vect{x}) + \frac{\mu}{2}\|\vect{x}^{*}-\vect{x}\|^{2} &\le f(\vect{x}^{*}) \\
f(\vect{x})-f(\vect{x}^{*}) &\le -\nabla f(\vect{x})^{T}(\vect{x}^{*}-\vect{x}) - \frac{\mu}{2}\|\vect{x}^{*}-\vect{x}\|^{2}\tikzmark{rearrangement} \\ % arrow should end here
2\mu\left(f(\vect{x})-f(\vect{x}^{*})\right) &\le - 2\mu\nabla f(\vect{x})^{T}(\vect{x}^{*}-\vect{x}) - \mu^{2}\|\vect{x}^{*}-\vect{x}\|^{2}\tikzmark{rearrangement_times2}.
\end{align*}
Adding \(0 = \|\nabla f(\vect{x})\|^{2} - \|\nabla f(\vect{x})\|^{2}\) to the right-hand side, allows us to rewrite the last line as
\begin{align*}
2\mu\left(f(\vect{x})-f(\vect{x}^{*})\right) &\le \|\nabla f(\vect{x})\|^{2} - \|\nabla f(\vect{x})\|^{2} - 2\mu\nabla f(\vect{x})^{T}(\vect{x}^{*}-\vect{x}) - \mu^{2}\|\vect{x}^{*}-\vect{x}\|^{2}\tikzmark{rearrangement_times2_plus0} \\
&= \|\nabla f(\vect{x})\|^{2} - \left(\|\nabla f(\vect{x})\|^2 + 2\mu\nabla f(\vect{x})^{T}(\vect{x}^{*}-\vect{x}) + \mu^{2}\|\vect{x}^{*}-\vect{x}\|^{2}\right) \\
&= \|\nabla f(\vect{x})\|^{2} - \|\nabla f(\vect{x}) + \mu(\vect{x}^{*}-\vect{x})\|^{2} \\
&\le \|\nabla f(\vect{x})\|^{2}. \qedhere
\begin{tikzpicture}[remember picture,overlay]
  \coordinate (A_start) at ($ (pic cs:mu_strong_convex)+(2.5mm,0) $);
  \coordinate (A_end) at ($ (pic cs:rearrangement)+(1.0mm,1ex) $);
  % (1) reference points for the paragraph
  \coordinate (parbegin) at (pic cs:para1A);
  \coordinate (parend)   at (pic cs:para1B);
  % \node[draw, red, thick, circle, inner sep = 0pt, minimum size=1mm] at (parbegin) {}; % for debugging
  % \node[draw, red, thick, circle, inner sep = 0pt, minimum size=1mm] at (parend) {}; % for debugging
  % (2) paragraph “silhouette” - eight corners
  \coordinate (p1) at ($ (parbegin) - (0,\letterdepth)$);
      \coordinate (p2) at ($ (parbegin) + (0,\letterheight)$);
      \coordinate (p3) at ($ (current page text area.east |- p2)$);
      \coordinate (tmp4) at ($ (parend) + (0,0.875*\baselineskip - \letterdepth)$);
      % \node[draw, red, thick, circle, inner sep = 0pt, minimum size=1mm] at (tmp4) {}; % for debugging
      \coordinate (p4) at ($ (current page text area.east |- tmp4)$);
      \coordinate (p5) at  (tmp4);
      \coordinate (p6) at ($ (parend) - (0,\letterdepth)$);
      % \node[draw, red, thick, circle, inner sep = 0pt, minimum size=1mm] at (p6) {}; % for debugging
      \coordinate (p7) at ($ (current page text area.west |- p6)$);
      \coordinate (tmp8) at ($ (parbegin) + (0,\letterheight - 0.875*\baselineskip)$);
      % \node[draw, red, thick, circle, inner sep = 0pt, minimum size=1mm] at (tmp8) {}; % for debugging
      \coordinate (p8) at ($ (current page text area.west |- tmp8)$);
  % (3) clip out the paragraph
  % \path[name path=MaskedGap,draw,dashed,red] (p1)--(p2)--(p3)--(p4)--(p5)--(p6)--(p7)--(p8)--cycle; % for debugging
  \begin{scope}[even odd rule]
    \clip
      (-\maxdimen,-\maxdimen) rectangle (\maxdimen,\maxdimen) % whole page
        (p1)--(p2)--(p3)--(p4)--(p5)--(p6)--(p7)--(p8)--cycle; % minus the hole
        \draw[->, shorten >=2mm, shorten <=2mm, color=Green] (A_start) to [out=-10, in=20] node[pos=0.75, right, xshift=1mm]{\footnotesize rearrange with \(\vect{y}=\vect{x}^{*}\)} (A_end); 
      \end{scope}
  \coordinate (B_start) at ($ (pic cs:rearrangement_times2)+(2.5mm,1ex) $);
  \coordinate (B_end) at ($ (pic cs:rearrangement_times2_plus0)+(1.0mm,1ex) $);
  \coordinate (B_mid) at ($ (B_start)!0.56!(B_end) + (22.5mm,0mm) $);
  \draw[->, color=Green] (B_start) .. controls ($(B_mid) + (0, 5mm)$) and ($(B_mid) + (0, -5mm)$) .. (B_end) node[pos=0.2, above right, xshift=-3mm, yshift=-0.5mm] {\footnotesize add \(0\) on the right};
\end{tikzpicture}%
\end{align*}%
\end{proof}%
\tikzexternalenable%
\begin{theorem}\label{thm:gd_linear_rate}
Let \(f\) be \(L\)-smooth and \(\mu\)-strongly convex with unique minimizer \(\vect{x}^{*}\). If the step size satisfies \(\alpha\in\left(0,\frac{1}{L}\right]\), then for every iteration \(k\ge 0\) of gradient descent,
\[
f\bigl(\vect{x}^{(k+1)}\bigr)-f\bigl(\vect{x}^{*}\bigr)
\le\left(1-\alpha\mu\right)\left(f\bigl(\vect{x}^{(k)}\bigr)-f\bigl(\vect{x}^{*}\bigr)\right)
\]
\end{theorem}
\begin{proof}
From the smoothness of \(f\) \hyperref[eq:gd_Lsmooth]{we have already derived} (Equation~\ref{eq:gd_Lsmooth}) that
\[
f\bigl(\vect{x}^{(k+1)}\bigr)\le f\bigl(\vect{x}^{(k)}\bigr)
-\alpha\left(1-\tfrac{\alpha L}{2}\right)\|\nabla f\bigl(\vect{x}^{(k)}\bigr)\|^{2}
\]
Since \(\alpha \le \frac{1}{L}\), we have \(\left(1-\frac{\alpha L}{2}\right)\ge\left(1-\frac{1}{2}\right)=\frac{1}{2}\) and therefore
\[
f\bigl(\vect{x}^{(k+1)}\bigr)-f\bigl(\vect{x}^{*}\bigr)
\le f\bigl(\vect{x}^{(k)}\bigr)-f\bigl(\vect{x}^{*}\bigr)
-\frac{\alpha}{2}\|\nabla f\bigl(\vect{x}^{(k)}\bigr)\|^{2}
\]
Using Equation \eqref{eq:optimality_error_inequality} from Lemma~\ref{lem:optimality_error_strongly_convex},
\(
2\mu\bigl(f\bigl(\vect{x}^{(k)}\bigr)-f\bigl(\vect{x}^{*}\bigr)\bigr)
\le\|\nabla f\bigl(\vect{x}^{(k)}\bigr)\|^{2},
\)
we obtain
\[
f\bigl(\vect{x}^{(k+1)}\bigr)-f\bigl(\vect{x}^{*}\bigr)
\le f\bigl(\vect{x}^{(k)}\bigr)-f\bigl(\vect{x}^{*}\bigr)
-\alpha\mu\Bigl(f\bigl(\vect{x}^{(k)}\bigr)-f\bigl(\vect{x}^{*}\bigr)\Bigr)
\]
which simplifies to the stated result.
\end{proof}

Theorem~\ref{thm:gd_linear_rate} shows that the error in function value decreases by a factor of \(1-\alpha\mu\) at each iteration. In other words, gradient descent converges \emph{linearly} with rate \(C=1-\alpha\mu\).

\begin{definition}[Linear Convergence]
Let \(\{z^{(k)}\}\) be a sequence with \(z^{(k)} \to z^*\) and there exists \(C \in [0,1)\) and \(\hat{k}\) such that
\[
\|z^{(k+1)}-z^{*}\|\le C\,\|z^{(k)}-z^{*}\|%\text{,}
\]
for all \(k\ge \hat{k}\). Then the sequence converges \emph{linearly} to \(z^*\) with rate \(C\).
\end{definition}


\begin{remark}\
\begin{itemize}
\item The smaller the linear rate \(C\), the faster is the convergence.
\item The rate \(1-\alpha \mu\) is the ``worst-case'' rate, but it can be faster for certain starting points.
\item The best theoretical rate is when \(1-\alpha \mu\) is smallest, i.e., \(\alpha^{*}=\frac{1}{L}\) (we do not always know \(L\)!)
\item Linear rate is generally considered slow, but unfortunately this is the best ``worst-case'' rate a first-order method can generally achieve!
\item In fact if \(f\) is only convex, the worst-case rate is sub-linear, i.e., slower than any linear rate.
\end{itemize}
\end{remark}

For strongly convex and smooth objective functions, we can also determine how many iterations we need until \(f\bigl(\vect{x}^{(k)}\bigr)-f\bigl(\vect{x}^{*}\bigr) \leq \epsilon\) for a given \(\epsilon\).

Since the error decreases 
\[
f\bigl(\vect{x}^{(k)}\bigr)-f\bigl(\vect{x}^{*}\bigr)
\le \Bigl(1-\alpha\mu\Bigr)^{k}\Bigl(f\bigl(\vect{x}^{(0)}\bigr)-f\bigl(\vect{x}^{*}\bigr)\Bigr)
\]
then to guarantee that
\[
f\bigl(\vect{x}^{(k)}\bigr)-f\bigl(\vect{x}^{*}\bigr)\le \epsilon
\]
it suffices to have
\[
k\ge \frac{1}{-\log(1-\alpha\mu)}\log\left(\frac{f\bigl(\vect{x}^{(0)}\bigr)-f\bigl(\vect{x}^{*}\bigr)}{\epsilon}\right)
\]
i.e. the number of iterations is {\color{red}\(O\left(\log \frac{1}{\epsilon} \right)\)}.



\subsection{Newton's Method}
\label{subsec:newton_method}
\subsubsection{Newton's Method for Univariate Functions}
\label{subsubsec:newton_univariate}


\begin{center}
  % stolen from https://tex.stackexchange.com/questions/551160/plot-that-demonstrate-newtons-method
  \begin{tikzpicture}[thick,scale=0.6]
  
    % Axes
    \draw[->,name path=xaxis] (-1,0) -- (12,0) node[above]{$y$};
    \draw[->] (0,-1) -- (0,8)node[right]{};;
    
    % Function plot
    \draw[ultra thick, orange, name path=function]  plot[smooth,domain=1:9.5] (\x, {0.1*\x^2-1.5}) node[left]{$f'(y)$};
    
    % plot tangent line
    \node[violet,right=0.2cm] at (8,4.9) {$q'(y;x^{(k)}) = f'(x^{(k)})+f''(x^{(k)})(y-x^{(k)})$};
    
    \draw[gray, thin, dotted] (8,0) -- (8,4.9) node[circle,fill,inner sep=2pt]{};
    \draw[very thick, dashed, violet, name path=Tfunction]  plot[smooth,domain=4.1:9.5] (\x, {1.6*\x-7.9});
     
    
    % x-axis labels
    \draw (8,0.1) -- (8,-0.1) node[below] {$x^{(k)}$};
    \draw [name intersections={of=Tfunction and xaxis}] ($(intersection-1)+(0,0.1)$) -- ++(0,-0.2) node[below,fill=white] {$x^{(k+1)}$} ;
    
    \end{tikzpicture}
  \end{center}
  
\hyperref[eq:second_order_taylor_approximation_univariate]{Recall} (Equation~\ref{eq:second_order_taylor_approximation_univariate}) the second-order Taylor approximation for a univariate \(f\). 
Since \(f \approx q(y;x^{(k)})\), it makes sense to choose \(x^{(k+1)}\) as the minimizer of \(q(y;x^{(k)})\), i.e.
\[
x^{(k+1)}=\argmin_{y} q(y;x^{(k)}) \hyperref[thm:nec_first_order_optimality_univariate]{\Rightarrow} q'(y;x^{(k)}) = f'(x^{(k)})+f''(x^{(k)})(y-x^{(k)}) \overset{!}{=} 0 
\]
\begin{equation}\label{eq:newton_step}
\quad \Longrightarrow \quad x^{(k+1)}=x^{(k)}-\frac{f'(x^{(k)})}{f''(x^{(k)})} %%\text{.}
\end{equation}
and if \(q''(x^{(k+1)};x^{(k)}) = f''(x^{(k)}) > 0\), then \(x^{(k+1)}\) is the global minimizer of \(q(\, \cdot \,; x^{(k)})\), since the latter is convex. 
The direction
\[
d_{N}=-\frac{f^{\prime}\left(x^{(k)}\right)}{f^{\prime \prime}\left(x^{(k)}\right)}
\]
is called the Newton's direction and the step~\eqref{eq:newton_step} called the Newton's step (see Algorithm~\ref{alg:newton}).
\begin{algorithm}[H]
\caption{Newton's Algorithm}\label{alg:newton}
\begin{algorithmic}[1]
    \State \textbf{Input:} Starting point \(x^{(0)}\), tolerance \(\epsilon>0\), and maximum iterations \(k_{\max}\)
    \State Set \(k\gets0\)
    \While{\(\left|f'(x^{(k)})\right|>\epsilon\) and \(k\leq k_{\max}\)}
        \State Compute \(f'(x^{(k)})\) and \(f''(x^{(k)})\)
        \State Update: \(x^{(k+1)}=x^{(k)}-\frac{f'(x^{(k)})}{f''(x^{(k)})}\)
        \State Set \(k\gets k+1\)
    \EndWhile
    \State \textbf{Output:} \(x^{(k)}\)
\end{algorithmic}
\end{algorithm}

\begin{remark}
Newton's algorithm does \emph{not} work when \(f''(x^{(k)}) \leq 0\), but there are possible remedies. If \(f\) is convex, then \(f''(x) \geq 0\) for any \(x\), and if \(f''(x^{(k)})>0\), the Newton's step is well defined.
\end{remark}

\begin{remark}
Originally, Newton's algorithm is a method for finding the root of a differentiable function, i.e., finding \(x\) such that \(g(x)=0\). The form of the iterates comes from the first-order Taylor approximation:
\[
g(y) \approx t(y;x^{(k)})=g(x^{(k)})+g'(x^{(k)})(y-x^{(k)})
\]
The next iterate \(x^{(k+1)}\) is chosen as the root of \(t(y;x^{(k)})\), i.e., \(t(x^{(k+1)};x^{(k)})=0\), which gives
\[
x^{(k+1)}=x^{(k)}-\frac{g(x^{(k)})}{g'(x^{(k)})}
\]
In optimzation, we use Newton's method to find a stationary point (which \emph{may} be a local minimizer and \emph{is} a global minimizer if \(f\) is convex). Therefore, we use Newton's method to find a root of \(g=f'\), which corresponds to the update
\[
x^{(k+1)}=x^{(k)}-\frac{f'(x^{(k)})}{f''(x^{(k)})}
\]
which is the same as~\eqref{eq:newton_step}, obtained from minimizing the second-order Taylor approximation. If the objective is not convex, we need to check that the stationary point found by Newton's method is a local minimizer.
\end{remark}

\subsubsection{Second Order Taylor's Expansion for Multivariate Functions}
\label{subsubsec:second_order_taylor_multivariate}
Assume \(f:\mathbb{R}^{n}\to\mathbb{R}\) is twice continuously differentiable. The second-order Taylor approximation is
\begin{equation}
\label{eq:second_order_taylor_approximation_multivariate}
{\color{Red}
\setlength{\fboxrule}{1pt}
\boxed{ 
\color{black}
f(\vect{y})\approx q(\vect{y};\vect{x})=f(\vect{x})+\nabla f(\vect{x})^{T}(\vect{y}-\vect{x})+\frac{1}{2}(\vect{y}-\vect{x})^{T}\matr{H}_{f}(\vect{x})(\vect{y}-\vect{x}) %\text{,}
}
}
\end{equation}
where \(\matr{H}_{f}(\vect{x})\) is the Hessian matrix defined as 
\begin{equation}\label{eq:hessian}
\matr{H}_{f}(\vect{x})
=
\left(\frac{\partial^{2}f}{\partial x_{i}\partial x_{j}}\right)_{i,j \in \{1,\ldots,n\}}
=
\begin{pmatrix}
\frac{\partial^{2}f}{\partial x_{1}^{2}} & \cdots & \frac{\partial^{2}f}{\partial x_{1}\partial x_{n}} \\
\vdots & \ddots & \vdots \\
\frac{\partial^{2}f}{\partial x_{n}\partial x_{1}} & \cdots & \frac{\partial^{2}f}{\partial x_{n}^{2}}
\end{pmatrix}
\in \mathbb{R}^{n \times n} %%\text{.}
\end{equation}
Since \(f\) is twice continuously differentiable, the Hessian is symmetric, i.e., \(\matr{H}_{f}(\vect{x})=\matr{H}_{f}(\vect{x})^{T}\).

We have already seen the first order Taylor's expansion for multivariate functions (Theorem~\ref{thm:taylor_multivariate_first_order}, Corollary~\ref{cor:taylor_error_multivariate_first_order}). 
Now let's extend this to the second order:
\begin{theorem}[Second Order Taylor's Expansion for Multivariate Functions]
  \label{thm:multivariate_taylor_expansion_second}
  If \(f:\mathbb{R}^{n}\to\mathbb{R}\) is continuously differentiable, then for any \(\vect{x}\in\mathbb{R}^{n}\) and \(\vect{d}\in\mathbb{R}^{n}\) there exists \(t\in (0,1)\) such that
  \[
  f(\vect{x}+\vect{d})=f(\vect{x})+\nabla f(\vect{x}+t\,\vect{d})^{T}\vect{d} %\text{.}
  \]
  Moreover, if \(f\) is twice continuously differentiable, then for any \(\vect{x}\in\mathbb{R}^{n}\) and \(\vect{d}\in\mathbb{R}^{n}\) there exists \(t\in (0,1)\) such that
  \[
  f(\vect{x}+\vect{d})=f(\vect{x})+\nabla f(\vect{x})^{T}\vect{d}+\frac{1}{2}\vect{d}^{T}\matr{H}_{f}(\vect{x}+t\,\vect{d})\vect{d} %\text{.}
  \]
  For \(n = 1\), this reduces to the univariate case (Theorem~\ref{thm:univariate_taylor_expansion}).
\end{theorem}

\begin{corollary}
\label{cor:approximation_error_multivariate_taylor_second}
If \(f\) is three times continuously differentiable, then for any \(\vect{x}\in\mathbb{R}^{n}\) there exists \(L>0\) and \(\epsilon>0\) such that
\[
\left|f(\vect{y})-q(\vect{y};\vect{x})\right|\le L\|\vect{y}-\vect{x}\|^3 %\text{,}
\]
for any \(\vect{y}\in\mathbb{R}^{n}\) such that \(\|\vect{x}-\vect{y}\|\le\epsilon\). For \(n=1\), this reduces to the univariate case (Corollary~\ref{cor:approximation_error_univariate_taylor}).
\end{corollary}

\subsubsection{Second Order Optimality Conditions for Multivariate Functions}
\label{subsubsec:second_order_optimality_conditions}
\begin{theorem}[Second Order Necessary Optimality Condition]
\label{thm:second_order_necessary_optimality_condition_multivariate}
If \(f\) is twice continuously differentiable and \(\vect{x}^{*}\) is a local minimizer of \(f\), then \(\matr{H}_{f}(\vect{x}^{*})\succeq 0\), i.e., \(\vect{p}^{T}\matr{H}_{f}(\vect{x}^{*})\vect{p}\ge 0 \text{ for all } \vect{p}\in\mathbb{R}^{n}\).
\end{theorem}
\begin{proof} (Contradiction)
Assume \(\vect{x}^{*}\) is a local minimizer of \(f\) and \(\matr{H}_{f}(\vect{x}^{*})\not \succeq 0\). 
Then there exists \(\vect{p}\in\mathbb{R}^{n}\) such that \(\vect{p}^{T}\matr{H}_{f}(\vect{x}^{*})\vect{p}<0\). 
Since \(\matr{H}_{f}(\, \cdot \,)\) is continuous there exists \(t_{0}>0\) such that for all \(t\in(0,t_{0})\), 
\[
\vect{p}^{T}\matr{H}_{f}(\vect{x}^{*}+t\vect{p})\vect{p}<0 %\text{.}
\]
Let \(t \in (0,t_{0})\). If \(\nabla f(\vect{x}^{*}) \neq 0\) we already know that \(\vect{x}^{*}\) cannot be a local minimizer (Theorem~\ref{thm:nec_first_order_optimality_multivariate}), so we assume \(\nabla f(\vect{x}^{*}) = 0\). 
Theorem \ref{thm:multivariate_taylor_expansion_second} says there exists \(\overline{t}\in(0,t)\) such that
\[
f(\vect{x}^{*}+t\vect{p})=f(\vect{x}^{*})+\frac{1}{2}t^{2}\vect{p}^{T}\matr{H}_{f}(\vect{x}^{*}+\overline{t}\vect{p})\vect{p} %\text{,}
\]
and since \(0 < \overline{t} < t < t_{0}\), we have \(\vect{p}^{T}\matr{H}_{f}(\vect{x}^{*}+\overline{t}\vect{p})\vect{p}<0\), which implies \(f(\vect{x}^{*}+t\vect{p})<f(\vect{x}^{*})\), contradicting the assumption that \(\vect{x}^{*}\) is a local minimizer.
\end{proof}

\begin{theorem}[Sufficient Optimality Conditions]\label{thm:sufficient_optimality_conditions_multivariate}
If \(f\) is twice continuously differentiable and \(\vect{x}^{*}\) is such that \(\nabla f(\vect{x}^{*})=\vect{0}\) and \(\matr{H}_{f}(\vect{x}^{*})\succ 0\), then \(\vect{x}^{*}\) is a strict local minimizer of \(f\).
\end{theorem}
\begin{proof}
Since \(\matr{H}_{f}(\, \cdot \,)\) is continuous, there exists \(r>0\) such that for any \(\vect{p}\) with \(\|\vect{p} - \vect{x}^{*}\| < r\), we have \(\matr{H}_{f}(\vect{x}^{*}+\vect{p}) \succ 0\). 
If \(\vect{p} \neq \vect{0}\), then \(\vect{p}^{T}\matr{H}_{f}(\vect{x}^{*}+\vect{p})\vect{p} > 0\), and by Theorem~\ref{thm:multivariate_taylor_expansion_second} there exists \(t \in (0,1)\) such that
\[
f(\vect{x}^{*}+\vect{p})=f(\vect{x}^{*})+\frac{1}{2}\vect{p}^{T}\matr{H}_{f}(\vect{x}^{*}+t\vect{p})\vect{p} > f(\vect{x}^{*}) %\text{,}
\]
which shows that \(\vect{x}^{*}\) is a strict local minimizer.
\end{proof}

\subsubsection{Minimization of Quadratic Functions}
\label{subsubsec:quadratic_minimization}
For a multivariate function \(f\), at iteration \(k\) of Newton's method (Algorithm~\ref{alg:newton_multivariate}), we need to find the minimizer \(\vect{x}^{*}\) of the second-order Taylor approximation (Equation~\ref{eq:second_order_taylor_approximation_multivariate}) at \(\vect{x}^{(k)}\) and define the new iterate as \(\vect{x}^{(k+1)} = \vect{x}^{*}\).
Equation~\ref{eq:second_order_taylor_approximation_multivariate} is a multivariate quadratic function of the form
\begin{align}
q(\vect{x}) &= b + \vect{g}^{T}\vect{x} + \vect{x}^{T}\matr{Q}\vect{x} \label{eq:multivariate_quadratic_function} \\
&= b + \sum_{i=1}^{n}g_{i}x_{i} + \sum_{i=1}^{n}\sum_{j=1}^{n}q_{ij}x_{i}x_{j} \label{eq:multivariate_quadratic_function_expanded}
\end{align}
where \(\vect{x} = \vect{y} - \vect{x}^{(k)}\), \(b = f(\vect{x}^{(k)})\), \(\vect{g} = \nabla f(\vect{x}^{(k)})\), and \(\matr{Q} = \frac{1}{2}\matr{H}_{f}(\vect{x}^{(k)})\).
We assume that \(\matr{Q}\) is symmetric, which is not restricive, since any quadratic model with a non-symmetric matrix can be re-written with a symmetric matrix \(\tilde{\matr{Q}} = \frac{1}{2}(\matr{Q}+\matr{Q}^{T})\).

Using Equation \eqref{eq:multivariate_quadratic_function_expanded}, it is easier to compute the derivatives:
\[
\frac{\partial}{\partial x_{k}}q(\vect{x}) = \sum_{i=1}^{n}\frac{\partial}{\partial x_{k}}g_{i}x_{i} + \sum_{i=1}^{n}\sum_{j=1}^{n}\frac{\partial}{\partial x_{k}}q_{ij}x_{i}x_{j} = g_{k} + 2\sum_{j=1}^{n}q_{kj}x_{j} %\text{,}
\]
\[
\frac{\partial^{2}}{\partial x_{k}\partial x_{l}}q(\vect{x}) = 2q_{kl} %\text{,}
\]
or in vectorized form:
\[
\nabla q(\vect{x}) = \vect{g} + 2\matr{Q}\vect{x} %\text{,}
\]
\[
\matr{H}_{q}(\vect{x}) = 2\matr{Q} %\text{.}
\]
A stationary point of \(q\) thus satisfies
\[
\nabla q(\vect{x}^{*}) = \vect{0} \quad \Longleftrightarrow  \quad 2\matr{Q}\vect{x}^{*} = -\vect{g}
\]
so \(\vect{x}^{*}\) is the solution of a linear system of equations. If \(\matr{Q}\) is invertible, then 
\(\vect{x}^{*} = -\frac{1}{2}\matr{Q}^{-1}\vect{g}\).

\begin{remark}
It is always less computationally expensive to solve a linear system \(\matr{A}\vect{x} = \vect{b}\) than inverting \(\matr{A}\), i.e., computing \(\matr{A}^{-1}\vect{b}\).
\end{remark}

Observe that for quadratic functions to be a local minimizer the
\begin{itemize}
\item necessary second-order optimality condition is \(\matr{Q} \succeq 0\),
\item sufficient second-order optimality condition is \(\matr{Q} \succ 0 \wedge 2\matr{Q}\vect{x}^{*} = -\vect{g}\).
\end{itemize}

\begin{caution}
Some important matrix properties. Let \(\matr{A} \in \mathbb{R}^{n \times n}\).
\begin{itemize}
\item The determinant \(\det(\matr{A})\) is equal to the product of the eigenvalues of \(\matr{A}\).
\item The trace \(\mathrm{tr}(\matr{A})\) is equal to the sum of the eigenvalues of \(\matr{A}\).
\item \(\matr{A}\) is positive definite (\(\matr{A} \succ 0\)) if and only if it is symmetric and all its eigenvalues are positive.
\item \(\matr{A}\) is positive semi-definite (\(\matr{A} \succeq 0\)) if and only if it is symmetric and all its eigenvalues are non-negative.
\item \(\matr{A}\) is negative definite (\(\matr{A} \prec 0\)) if and only if it is symmetric and all its eigenvalues are negative.
\item \(\matr{A}\) is negative semi-definite (\(\matr{A} \preceq 0\)) if and only if it is symmetric and all its eigenvalues are non-positive.
\item If \(\matr{A}\) is symmetric and has both positive and negative eigenvalues, then it is indefinite.
\end{itemize}
\end{caution}

Now consider \(\vect{x}^{*}\) a stationary point of \(q(\vect{x}) = b + \vect{g}^{T}\vect{x} + \vect{x}^{T}\matr{Q}\vect{x}\), i.e. a point satisfying \(2\matr{Q}\vect{x}^{*} = -\vect{g}\).
\begin{itemize}
\item If \(\matr{Q} \succ 0\), then \(q\) is strictly convex and \(\vect{x}^{*}\) is the global minimizer of \(q\).
\item If \(\matr{Q} \prec 0\), then \(q\) is strictly concave and \(\vect{x}^{*}\) is the global maximizer of \(q\).
\item If \(\matr{Q}\) has at least one negative eigenvalue then \(q\) is unbounded below and has no minimizer.
\item If \(\matr{Q}\) is indefinite, \(\vect{x}^{*}\) is a saddle point of \(q\).
\item If \(\matr{Q} \succeq 0\), then \(q\) is convex. If \(\matr{Q}\) is not positive definite, then it is singular and in this case \(q\) is either unbounded or has an infinite number of global minimizers.
\end{itemize}

\subsubsection{Newton's Method for Multivariate Functions}
\label{subsubsec:newton_multivariate}

% \begin{tikzpicture}[scale=3]

%   % Axes
%   \draw[->] (-1, -0.8) -- (0.2, 0.2) node[anchor=west] {$x_1$};
%   \draw[->] (-0.5, 0) -- (1.5, 0) node[anchor=west] {$x_2$};
%   \draw[->] (0, 0) -- (0, 1.2) node[anchor=south] {$f,q$};

%   % Level set ellipses (horizontal curves)
%   \foreach \y in {0.3, 0.6, 0.9} {
%     \draw[dashed] (0.3-\y/3, \y) ellipse (0.5 and 0.1);
%   }

%   % Parabolic bowl sides
%   \draw[thick] (-0.5,0) .. controls (-0.2,0.5) and (-0.1,1) .. (0,1.1);
%   \draw[thick] (0.5,0) .. controls (0.2,0.5) and (0.1,1) .. (0,1.1);

%   % Approximating function q
%   \draw[thick] (-0.4,0) .. controls (-0.1,0.4) and (-0.05,0.9) .. (0,1);
%   \draw[thick] (0.4,0) .. controls (0.1,0.4) and (0.05,0.9) .. (0,1);

%   % Labels for f and q
%   \node at (-0.55,1.1) {$f$};
%   \node at (0.55,1.1) {$q$};

%   % Points and vertical dashed lines
%   \fill (0.15,0.9) circle (0.5pt); % Current point
%   \draw[dashed] (0.15,0.9) -- (0.15,0) node[below] {\footnotesize $x^{(k)}$};
%   \node[anchor=west] at (0.15,0.5) {Current Point};

%   \fill (0.45,0.3) circle (0.5pt); % Predicted minimizer
%   \draw[dashed] (0.45,0.3) -- (0.45,0) node[below] {\footnotesize $x^{(k+1)}$};
%   \node[anchor=west] at (0.45,0.1) {Predicted Minimizer};

%   \fill (0.6,0.2) circle (0.5pt); % Optimal point
%   \draw[dashed] (0.6,0.2) -- (0.6,0) node[below] {\footnotesize $\mathbf{x}^*$};

% \end{tikzpicture}

\begin{algorithm}[H]
  \caption{Newton's Algorithm (Multivariate)}
  \label{alg:newton_multivariate}
  \begin{algorithmic}[1]
      \State \textbf{Input:} Starting point \(\vect{x}^{(0)}\), tolerance \(\epsilon>0\), and maximum iterations \(k_{\max}\)
      \State Set \(k \gets 0\)
      \While{\(\left|\nabla f(\vect{x}^{(k)})\right|>\epsilon\) and \(k\leq k_{\max}\)}
          \State Compute \(\nabla f(\vect{x}^{(k)})\) and \(\matr{H}_f(\vect{x}^{(k)})\)
          % \State Update: \(\vect{x}^{(k+1)}=\vect{x}^{(k)}-\frac{\nabla f(\vect{x}^{(k)})}{\matr{H}_f(\vect{x}^{(k)})}\)
          \State Compute \(\vect{u}_{\min}\) as the minimizer of \(q(\vect{u};\vect{x}^{(k)})\).\label{alg:line:calculate_newton_direction} \Comment{If it exists}
          \State Update: \(\vect{x}^{(k+1)}=\vect{u}_{\min}\) \Comment{If it exists: \(\vect{x}^{(k+1)} = \vect{x}^{(k)} - (\matr{H}_{f}(\vect{x}^{(k)})^{-1} \nabla f(\vect{x}^{(k)})\)}
          \State Set \(k\gets k+1\)
      \EndWhile
      \State \textbf{Output:} \(\vect{x}^{(k)}\)
  \end{algorithmic}
\end{algorithm}

In Line~\ref{alg:line:calculate_newton_direction} of Algorithm~\ref{alg:newton_multivariate}, we need to find the minimizer of the quadratic local model at \(\vect{x}^{(k)}\) given by the \hyperref[eq:second_order_taylor_approximation_multivariate]{second-order Taylor approximation} (Section \ref{subsubsec:second_order_taylor_multivariate}, Equation \ref{eq:second_order_taylor_approximation_multivariate}).
If we set \(\vect{y} = \vect{x}^{(k)} + \vect{d}\) in \eqref{eq:second_order_taylor_approximation_multivariate}, we obtain
\[
q(\vect{d};\vect{x}^{(k)}) = f(\vect{x}^{(k)}) + \nabla f(\vect{x}^{(k)})^{T}\vect{d} + \frac{1}{2}\vect{d}^{T}\matr{H}_{f}(\vect{x}^{(k)})\vect{d} %\text{.}
\]
Denoting \(b = f(\vect{x}^{(k)})\), \(\vect{g} = \nabla f(\vect{x}^{(k)})\), and \(\matr{Q} = \frac{1}{2}\matr{H}_{f}(\vect{x}^{(k)})\), we have a quadratic function of the same form as in~\eqref{eq:multivariate_quadratic_function}:
\[
q(\vect{d};\vect{x}^{(k)}) = b + \vect{g}^{T}\vect{d} + \vect{d}^{T}\matr{Q}\vect{d} %\text{.}
\]
Thus a stationary point of \(q\) satisfies
\[
\vect{d}^{*} = -\frac{1}{2}\matr{Q}^{-1}\vect{g} = -\big(\matr{H}_{f}(\vect{x}^{(k)})\big)^{-1} \nabla f(\vect{x}^{(k)}) %\text{.}
\]
If \(\matr{Q} \succ 0\), then \(\vect{d}^{*}\) is the global minimizer of \(q(\vect{d};\vect{x}^{(k)})\), and is the \emph{Newton's direction}, denoted \(\vect{d}_{N}\).
Hence in the original problem with \(\vect{y} = \vect{x}^{(k)} + \vect{d}\), the minimizer \(\vect{x}^{k+1}\) is
\[
\vect{x}^{(k+1)} = \vect{x}^{(k)} + \vect{d}_{N} = \vect{x}^{(k)} - \big(\matr{H}_{f}(\vect{x}^{(k)})\big)^{-1} \nabla f(\vect{x}^{(k)}) %\text{.}
\]
This update rule is called the \emph{Newton's step}.
\begin{caution} Two important remarks:
\begin{itemize}
\item If \(\matr{H}_{f}(\vect{x}^{(k)})\) is not positive definite, then we \underline{cannot} compute the Newton's direction.
We need to detect this case by checking if all eigenvalues of \(\matr{H}_{f}(\vect{x}^{(k)})\) are positive.
If they are not, an easy solution is to take a gradient step for that iteration.
\item To compute \(\vect{d}_{N}\), we should \underline{not} invert \(\matr{H}_{f}(\vect{x}^{(k)})\), but solve the linear system 
\[
2\matr{Q}\vect{d}_{N} = -\vect{g} \text{, i.e., }\matr{H}_{f}(\vect{x}^{(k)})\vect{d}_{N} = -\nabla f(\vect{x}^{(k)})
\]
for \(\vect{d}_{N}\).\qedhere
\end{itemize}
\end{caution}


\subsubsection{Convergence of Newton's Method}
\label{subsubsec:newton_convergence}

\begin{theorem}[Convergence of Newton's Method]\label{thm:convergence_newton_method}
If \(f:\mathbb{R}^{n}\to\mathbb{R}\) is twice continuously differentiable, 
\(\vect{x}^{*}\) is a point such that \(\nabla f(\vect{x}^{*}) = \vect{0} \wedge \matr{H}_{f}(\vect{x}^{*}) \succ 0\),
and \(\matr{H}_{f}(\,\cdot \,)\) is \(L\)-Lipschitz continuous with \(L>0\) in a neighborhood of \(\vect{x}^{*}\),
then there exists \(\epsilon>0\) such that for any \(\vect{x}^{(0)}\) with \(\|\vect{x}^{(0)}-\vect{x}^{*}\| \leq \epsilon\), 
at each iteration \(k \geq 0\) of Newton's method (Algorithm~\ref{alg:newton_multivariate}), 
\begin{equation}
\label{eq:convergence_newton_method}
\big\|\vect{x}^{(k+1)}-\vect{x}^{*}\big\| \leq L\left\|\big(\matr{H}_{f}(\vect{x}^{*})\big)^{-1}\right\| \big\|\vect{x}^{(k)}-\vect{x}^{*}\big\|^{2} %\text{,}
\end{equation}
i.e., the iterates \(\{\vect{x}^{(k)}\}\) converge quadratically to \(\vect{x}^{*}\).
\end{theorem}

\begin{remark}
Theorem~\ref{thm:convergence_newton_method} shows that if Algorithm~\ref{alg:newton_multivariate} is initialized sufficiently close to a local minimizer \(\vect{x}^{*}\) at which the second-order sufficient optimality conditions (Theorem~\ref{thm:sufficient_optimality_conditions_multivariate}) hold, then the iterates converge quadratically to \(\vect{x}^{*}\).
It is thus a \emph{local convergence} result and \underline{not} a global convergence result.
\end{remark}

\begin{definition}[Convergence Rates]
\label{def:convergence_rates}
A sequence of vectors \(\{\vect{z}^{(k)}\}\) with \(\vect{z}^{(k)} \to \vect{z}^{*}\) converges
\begin{itemize}
  \item \emph{linearly} if there exists \(c \in (0,1)\) and \(\hat{k} \geq 0\) such that 
  \[
  \|\vect{z}^{(k+1)}-\vect{z}^{*}\| \leq c\|\vect{z}^{(k)}-\vect{z}^{*}\| \text{ for all } k \geq \hat{k} \text{.}
  \]
  \item \emph{quadratically} if there exists \(c \in (0,\infty)\) and \(\hat{k} \geq 0\) such that
  \[
  \|\vect{z}^{(k+1)}-\vect{z}^{*}\| \leq c\|\vect{z}^{(k)}-\vect{z}^{*}\|^{2} \text{ for all } k \geq \hat{k} \text{.}
  \]
  \item \emph{sublinearly} if there exists \(\{c^{(k)}\} \subset (0,1)\) with \(c^{(k)} \to 1\) and \(\hat{k} \geq 0\) such that
  \[
  \|\vect{z}^{(k+1)}-\vect{z}^{*}\| \leq c^{(k)}\|\vect{z}^{(k)}-\vect{z}^{*}\| \text{ for all } k \geq \hat{k} \text{.}
  \]
  \item \emph{superlinearly} if there exists \(\{c^{(k)}\} \subset (0,\infty)\) with \(c^{(k)} \to 0\) and \(\hat{k} \geq 0\) such that
  \[
  \|\vect{z}^{(k+1)}-\vect{z}^{*}\| \leq c^{(k)}\|\vect{z}^{(k)}-\vect{z}^{*}\| \text{ for all } k \geq \hat{k} \text{.}\qedhere
  \]
\end{itemize}
\end{definition}

\begin{remark}
Observe that: quadratic \(\Longrightarrow\) superlinear \(\Longrightarrow\) linear \(\Longrightarrow\) sublinear.
\end{remark}


Theorem~\ref{thm:convergence_newton_method} says that when Algorithm~\ref{alg:newton_multivariate} is initialized sufficiently close to the local minimizer \(x^{*}\), we have \(C = \|(\matr{H}_{f}(\vect{x}^{*}))^{-1}\| > 0\) such that
\begin{equation}\label{eq:newton_quadratic_convergence_recursion_iterates}
\|x^{(k+1)}-x^{*}\|\le C\,\|x^{(k)}-x^{*}\|^{2} %\text{.}
\end{equation}
Denoting the error at iteration \(k\) by
\[
\texttt{err}_{k}=\|x^{(k)}-x^{*}\| %\text{,}
\]
we then can rewrite \eqref{eq:newton_quadratic_convergence_recursion_iterates} as
\begin{equation}\label{eq:newton_quadratic_convergence_recursion_error}
\texttt{err}_{k+1}\le C\,(\texttt{err}_{k})^{2} %\text{.}
\end{equation}
Recursively applying this estimate leads to
\[
\texttt{err}_{K} \le C \Bigg( 
\underbrace{
C \bigg( 
\underset{\vdots}{
\cdots 
\underbrace{
C \Big( 
\underbrace{
C \texttt{err}_{0}^{2} 
}_{\geq \texttt{err}_{1}}
\Big)^{2} 
}_{\geq \texttt{err}_{2}}
\cdots 
}
\bigg)^{2} 
}_{\geq \texttt{err}_{K-1}}
\Bigg)^{2}
=
C^{2^{K}-1}\,(\texttt{err}_{0})^{2^{K}} %\text{.}
\]
For simplicity, assume \(\texttt{err}_{0}=1\). To achieve an error \(\texttt{err}_{K}\le\epsilon\) with \(0<\epsilon\ll 1\), it suffices that
\[
C^{2^{K}-1}\le \epsilon
\]
Taking logarithms gives
\[
\left(2^{K}-1\right)\log C \le \log \epsilon
\]
Since \(\log C<0\) (because \(C<1\) in the quadratic regime), we can rearrange to obtain
\[
2^{K}\ge 1+\frac{\log \epsilon}{\log C}
\]
and taking logarithms once more yields
\[
K \ge \log\left(1+\frac{\log \epsilon^{-1}}{\log C^{-1}}\right)=O\Bigl(\log \log \epsilon^{-1}\Bigr)
\]
Thus, to reduce the optimality error to order \(\epsilon\) we need only {\color{red}\(O(\log \log \epsilon^{-1})\)} iterations. 
This is significantly fewer than the \(O(\log \epsilon^{-1})\) iterations typically required by gradient descent.

In its standard form, Algorithm~\ref{alg:newton_multivariate} is \underline{not} globally convergent, but it can be made so with a slight variation of the update rule using \emph{line search}.
Also, computing second-order derivatives is computationally expensive, but also this can be avoided by using \emph{quasi-Newton methods} while keeping some of the benefits of Newton's method.
  











%%%%%%%%%%%%%%%%%%%%%%%%%%%%%%%%%%%%%%%%%%%%%%%%%%%%%%%%%%%%%%%%%%%%%%%%%%%%%%%
% 5 Line Search Algorithms
%%%%%%%%%%%%%%%%%%%%%%%%%%%%%%%%%%%%%%%%%%%%%%%%%%%%%%%%%%%%%%%%%%%%%%%%%%%%%%%

\subsection{Line Search Algorithms}

To make Newton's algorithm globally convergent from any starting point we need more flexibility in the update rule. One possibility is to take a smaller step in the same direction, i.e.,
\[
\vect{x}^{(k+1)} = \vect{x}^{(k)} + \alpha\, \vect{d}_{N} = \vect{x}^{(k)} + \alpha 
\underbrace{
\left(-\big(\matr{H}_{f}(\vect{x}^{(k)})\big)^{-1}\nabla f(\vect{x}^{(k)})\right) 
}_{\vect{d}_{N}}
%\text{,}
\]
with \(\alpha \in (0,1)\) a step size. This is similar to Gradient Descent where instead of 
\(
\vect{d}_{N} %\text{,}
\)
we have
\(
\vect{d}_{S} = -\nabla f(\vect{x}^{(k)}) %\text{.}
\)
In fact, both Gradient Descent and Newton's algorithms (and their variants) fall in the class of \emph{line search} algorithms where finding a “good” step size is a sub-routine. A line search method has an update rule of the form
\[
\vect{x}^{(k+1)} = \vect{x}^{(k)} + \alpha^{(k)}\, \vect{d}^{(k)} %\text{,}
\]
where:
\begin{itemize}
  \item \(\vect{d}^{(k)} \in \mathbb{R}^n\) is a search direction;
  \item \(\alpha^{(k)} > 0\) is a chosen step size.
\end{itemize}

Performing a line search specifically refers to the sub-routine of choosing the best possible (or at least, good enough) \(\alpha^{(k)}\), given \(\vect{d}^{(k)}\). 
At a minimum we require that we achieve a decrease of \(f\), i.e.,
\[
f\bigl(\vect{x}^{(k+1)}\bigr) = f\bigl(\vect{x}^{(k)} + \alpha^{(k)}\,\vect{d}^{(k)}\bigr) < f\bigl(\vect{x}^{(k)}\bigr) %\text{.}
\]
This is always satisfied for sufficiently small \(\alpha^{(k)}\) if \(\vect{d}^{(k)}\) is a descent direction. 
Recall that if we define
\[
g(\alpha) = f(\vect{x}^{(k)} + \alpha\, \vect{d}^{(k)}) %\text{,}
\]
then its derivative is
\(
g'(\alpha) = \nabla f(\vect{x}^{(k)} + \alpha\, \vect{d}^{(k)})^T \vect{d}^{(k)} %\text{,}
\)
with
\(
g'(0) = \nabla f(\vect{x}^{(k)})^T \vect{d}^{(k)} %\text{.}
\)
Since
\[
g(\alpha) \approx f\bigl(\vect{x}^{(k)}\bigr) + g'(0)\,\alpha %\text{,}
\]
we can guarantee \(g(\alpha) < f\bigl(\vect{x}^{(k)}\bigr)\) for sufficiently small \(\alpha>0\) if
\[
g'(0) = \nabla f\bigl(\vect{x}^{(k)}\bigr)^T \vect{d}^{(k)} < 0 %\text{,}
\]
i.e., if \(\vect{d}^{(k)}\) is indeed a descent direction. Unfortunately, this minimal requirement is not enough to choose a good step size in practice.

One possibility is to choose the step size that minimizes \(g(\alpha)\), i.e.,
\begin{equation}
\label{eq:exact_line_search}
\min_{\alpha > 0} \; g(\alpha) \quad \text{with} \quad g(\alpha)=f\bigl(\vect{x}^{(k)} + \alpha\, \vect{d}^{(k)}\bigr) %\text{,}
\end{equation}
which is called \emph{exact line search}; however, solving this subproblem can be expensive. Therefore, in practice, \(\alpha^{(k)}\) is chosen to satisfy less strong requirements, for instance the Wolfe conditions.

\begin{remark}
Whenever \(\alpha^{(k)}\) is \emph{not} chosen by solving the minimization problem \eqref{eq:exact_line_search} exactly, we say it is an \emph{inexact line search} method.
\end{remark}


%%%%%%%%%%%%%%%%%%%%%%%%%%%%%%%%%%%%%%%%%%%%%%%%%%%%%%%%%%%%%%%%%%%%%%%%%%%%%%%
% 5.1 Wolfe Conditions
%%%%%%%%%%%%%%%%%%%%%%%%%%%%%%%%%%%%%%%%%%%%%%%%%%%%%%%%%%%%%%%%%%%%%%%%%%%%%%%

\subsubsection{Wolfe Conditions}


% \begin{center}
% \begin{tikzpicture}[>=stealth]
%   \begin{axis}[
%     axis lines=middle,
%     axis line style={
%     ->,
%       >=Computer Modern Rightarrow
%     },
%     xlabel={\footnotesize$\alpha$},
%     xlabel style={below},
%     % ylabel={$y$},
%     xmin=-0.05, xmax=1.25,
%     ymin=-0.13, ymax=1.5,
%     samples=200,
%     domain=0:1,
%     % legend style={draw=none, at={(0.97,0.97)}, anchor=north east},
%     xtick=\empty,
%     ytick={1.2},
%     yticklabel={\footnotesize$f\bigl(\vect{x}^{(k)}\bigr)$},
%     clip=false,
%     x =6cm,
%     y = 2cm
%   ]
%     %---------------------------------------------------------------------
%     % 1) The polynomial f(x) = a4*x^4 + a3*x^3 + a2*x^2 + a1*x + a0
%     %---------------------------------------------------------------------
%     \tikzmath{
%       \a4 = 34.3;
%       \a3 = -68.7;
%       \a2 = 43.0;
%       \a1 = -9.1;
%       \a0 = 1.2;
%     }
%     \addplot [ black, thick, domain=0:1]
%       {(\a4)*x^4 + (\a3)*x^3 + (\a2)*x^2 + (\a1)*x + (\a0)}
%       node [pos=1.0, anchor=south west, xshift=-1.8mm, yshift=0.8mm] {\footnotesize$g(\alpha) = f\bigl(\vect{x}^{(k)} + \alpha\, \vect{d}^{(k)}\bigr)$};
%     % \addlegendentry{\(f(x)\)}

%     %---------------------------------------------------------------------
%     % 2) Tangent line at x=0:
%     %    f(0) = a0
%     %    f'(0) = a1
%     %    => tangent y = f(0) + f'(0)*(x - 0) = a0 + a1*x
%     %---------------------------------------------------------------------
%     \addplot [thick, dashed, black, domain=0:0.1]
%       {(\a0) + (\a1)*x}
%       node [pos=1, anchor=north west, xshift=-1.2mm, yshift=0.2mm] {\footnotesize$f\bigl(\vect{x}^{(k)}\bigr) + g'(0)\,\alpha $};
%     % \addlegendentry{tangent at \(x=0\)}

%     %---------------------------------------------------------------------
%     % 3) Relaxed line \ell(x) = f(0) + \eta * x * f'(0)
%     %    for some 0 < eta < 1
%     %    => \ell(x) = a0 + \eta * a1 * x
%     %---------------------------------------------------------------------
%     \addplot [thick, black, dotted, domain=0:1]
%       {(\a0) + 0.1*(\a1)*x}
%       node [pos=1.0, anchor=west, xshift=0.2mm] {\footnotesize$\ell(\alpha) = f\bigl(\vect{x}^{(k)}\bigr) + \eta\,\alpha\, \nabla f\bigl(\vect{x}^{(k)}\bigr)^T \vect{d}^{(k)}$};
%     % \addlegendentry{\(\ell(x)\) (relaxed, \(\eta=0.5\))}

%   \end{axis}
% \end{tikzpicture}
% \end{center}
\tikzexternaldisable%
\[
\begin{tikzpicture}%[scale=0.3] %scale does not work here

  \ifLuaTeX
  %---------------------------------------------------------------
  % LEFT PANEL : level sets of  f(x,y)=x^2+2y^2
  %---------------------------------------------------------------
  \begin{axis}[
    scale=0.6,
      width=9cm, height=6cm,
      domain=-1.5:1.5,
      y domain=-1:1,
      view={0}{90},
      samples=100,
      ticks=none,
    %   xlabel={$x$},
    %   ylabel={$y$},
      name=left  % so we can anchor the second axis to it
    ]
      % Contours of f
      % v v v need lualatex compilation!!!!!! v v v
      \addplot3[
        contour lua={
          draw color=black,
          number=10,
          labels=false
        },
        % thick
      ] {(1 + x^2 + 2*y^2)};
      % ^ ^ ^ ^ ^ ^ ^ ^ ^ ^ ^ ^ ^ ^ ^ ^ ^ ^ ^ ^ ^ ^

    %   % Starting point  x0 = (1,0.5)
    %   \addplot+[only marks, mark=*, mark size=2.2pt]
    %     coordinates {(1,0.5)};

      \node[circle, fill=black, inner sep=1pt, outer sep=0pt] at (axis cs:1,0.5) {};
      \node[fill=white,inner sep=0pt,outer sep=2pt, anchor=south west] at (axis cs:1,0.5) {\footnotesize$\vect{x}^{(k)}$};

      % Steepest–descent ray  x0 + α ( -∇f(x0) )
      \addplot[
        draw=none,
        domain=0:0.6, samples=2
      ] ({1 - 2*x}, {0.5 - 2*x})
      node[pos=1, fill=white,anchor=north, outer sep = 0pt, inner sep=0pt]
        {\footnotesize$\vect{x}^{(k)} + \alpha \vect{d}^{(k)}$};
        \addplot[
        thick,
        ->,
        domain=0:0.6, samples=2
      ] ({1 - 2*x}, {0.5 - 2*x});

      % In-plot label for f
      \node[fill=white,inner sep=0pt,outer sep=0pt] at (rel axis cs:0.15,0.85) {\footnotesize$f(\vect{x})$};
  \end{axis}

  %---------------------------------------------------------------
  % RIGHT PANEL : g(α) and its two lines
  %---------------------------------------------------------------
  \begin{axis}[
    scale=0.8,
    %\ifLuaTeX %The safe rule is: Never put a TeX conditional inside a PGF/TikZ option list. Decide outside the environment which complete block you want.
    at={(left.east)}, anchor=west,   % stick to the right of the first axis
    xshift=1.5cm,             
    width=9cm, height=6cm,
    axis lines=middle,
    axis line style={->},
    xlabel={\footnotesize$\alpha$},
    xmin=0, xmax=0.7,
    ymin=0, ymax=2.4,
    xtick=\empty,
    ytick={2.0},
    yticklabel={\footnotesize$f\bigl(\vect{x}^{(k)}\bigr)$},
    yticklabel style={inner sep=1pt, outer sep=0pt},
    clip=false,
    ]

    % -------------------- user-set parameters --------------------------------
    \pgfmathsetmacro{\normalEta}{0.2}            %  Armijo parameter  η  ∈ (0,1)
    \pgfmathsetmacro{\alphaL}{0.15}         %  α_min  (must satisfy 0 < α_min < 1/3)
    % ------------------------------------------------------------------------
    
    \pgfmathsetmacro{\gprimeZero}{-8}       %  g'(0)

    % g(α)
    \addplot[black, thick, domain=0:0.6, samples=200]
    {2.0 - 8*x + 12*x^2}
    node[pos=1, anchor= south west, outer sep = 2pt, inner sep=0pt]
    {\footnotesize$g(\alpha) = f\bigl(\vect{x}^{(k)} + \alpha\, \vect{d}^{(k)}\bigr)$};
    % {\footnotesize$g(\alpha)$};

    % Tangent:  f(x0) + g'(0) α  
    \addplot[black, dashed, domain=0:0.21] {2.0 - 8*x}
    node[pos=1, anchor= west, outer sep = 2pt, inner sep=0pt] %{\footnotesize$f\bigl(\vect{x}^{(k)}\bigr) + g'(0)\,\alpha $};
    {\tiny{slope: $g'(0)$}};

    % Relaxed line: ℓ(α) = f(x0) + η α g'(0) 
    \addplot[black, thick, domain=0:0.6]
    {2.0 + \normalEta * \gprimeZero * x}
    node[pos=1, anchor=west, outer sep = 2pt, inner sep=0pt]
    %  {\footnotesize$\ell(\alpha) = f\bigl(\vect{x}^{(k)}\bigr) + \eta\,\alpha\, \nabla f\bigl(\vect{x}^{(k)}\bigr)^T \vect{d}^{(k)}$};
    {\footnotesize$\ell(\alpha) = f\bigl(\vect{x}^{(k)}\bigr) + \eta\, g'(0)\,\alpha$};
    % {\footnotesize$\ell(\alpha)$};
    % {\footnotesize$f\bigl(\vect{x}^{(k)}\bigr) + \eta\,\alpha\, g'(0) =: \ell(\alpha)$};
    \addplot[draw=none, domain=0:0.21, samples=2] {2.0 + \normalEta * \gprimeZero * x}
    node[pos=1, anchor= west, inner sep=0pt, outer sep=2pt, yshift=1mm]
    {\tiny{slope: $\eta\,g'(0)$}};

  % derived curvature parameter  \barη  via  α_min = (1- \barη)/3  (for this quadratic)
  \pgfmathsetmacro{\barEta}{1 - 3*\alphaL}

  % slopes
  \pgfmathsetmacro{\slopeEta}{\normalEta    * \gprimeZero}   %  η g'(0)
  \pgfmathsetmacro{\slopeBar}{\barEta * \gprimeZero}   %  \barη g'(0)

  % ordinates
  \pgfmathsetmacro{\gAlphaL}{2.0 - 8*\alphaL + 12*\alphaL*\alphaL}

  % upper bound α_max from Armijo line–function intersection
  \pgfmathsetmacro{\alphaU}{(1 - \normalEta)*abs(\gprimeZero)/12}
  \pgfmathsetmacro{\gAlphaU}{2.0 - 8*\alphaU + 12*\alphaU*\alphaU}

  % dashed \barη–slope reference line
  \addplot[black, dashed, domain=0:0.21, samples=2] {2.0 + \slopeBar*x}
    node[pos=1, anchor=west, inner sep=0pt, outer sep=2pt] %{\footnotesize$f(\vect{x}^{(k)}) + \bar{\eta}\,g'(0)\,\alpha$};
    {\tiny{slope: $\bar{\eta}\,g'(0)$}};

  % ─── lower bound α_min (curvature condition) ─────────────────────────────
  \node[circle,fill=black,inner sep=1pt] at (axis cs:\alphaL,\gAlphaL) {};
  \draw[dotted] (axis cs:\alphaL,\gAlphaL) -- (axis cs:\alphaL,0);

  % tangent to g at α_min   (slope = \barη g'(0))
  \addplot[black, domain={\alphaL-0.05}:{\alphaL+0.05}, samples=2] {\gAlphaL + \slopeBar*(x - \alphaL)};
    % node[pos=0.55, anchor=south west, inner sep=0pt, outer sep=1pt]
    %   {\footnotesize$\text{slope} = \bar{\eta}\,g'(0)$};

  % ─── upper bound α_max (Armijo condition) ────────────────────────────────
  \node[circle,fill=black,inner sep=1pt] at (axis cs:\alphaU,\gAlphaU) {};
  \draw[dotted] (axis cs:\alphaU,\gAlphaU) -- (axis cs:\alphaU,0);

  % interval bar on α-axis
%   \draw[very thick] (axis cs:\alphaL,0) -- (axis cs:\alphaU,0);
  \draw (axis cs:\alphaL,0) -- (axis cs:\alphaL,-0.08);
  \draw (axis cs:\alphaU,0) -- (axis cs:\alphaU,-0.08);

  % labels for bounds
%   \node[below] at (axis cs:\alphaL,0) {\footnotesize$\alpha_{\min}$};
%   \node[below] at (axis cs:\alphaU,0) {\footnotesize$\alpha_{\max}$};

% arrow for 2nd Wolfe (curvature) – points to the right
\draw[->, thick]
  (axis cs:\alphaL,0) node[anchor=north west] {\tiny \eqref{eq:second_wolfe_condition} holds}%2. Wolfe}
  -- (axis cs:{\alphaL + 0.05},0);

% arrow for 1st Wolfe (Armijo) – points to the left
\draw[->, thick]
  (axis cs:\alphaU,0) node[anchor=north east] {\tiny \eqref{eq:first_wolfe_condition} holds}%1. Wolfe}
  -- (axis cs:{\alphaU - 0.05},0);

\end{axis}

\else

  \begin{axis}[
    scale=0.8,           
    width=9cm, height=6cm,
    axis lines=middle,
    axis line style={->},
    xlabel={\footnotesize$\alpha$},
    xmin=0, xmax=0.7,
    ymin=0, ymax=2.4,
    xtick=\empty,
    ytick={2.0},
    yticklabel={\footnotesize$f\bigl(\vect{x}^{(k)}\bigr)$},
    yticklabel style={inner sep=1pt, outer sep=0pt},
    clip=false,
    ]
    \pgfmathsetmacro{\normalEta}{0.2}         
    \pgfmathsetmacro{\alphaL}{0.15}       
    \pgfmathsetmacro{\gprimeZero}{-8}       
    \addplot[black, thick, domain=0:0.6, samples=200]
    {2.0 - 8*x + 12*x^2}
    node[pos=1, anchor= south west, outer sep = 2pt, inner sep=0pt]
    {\footnotesize$g(\alpha) = f\bigl(\vect{x}^{(k)} + \alpha\, \vect{d}^{(k)}\bigr)$};
    \addplot[black, dashed, domain=0:0.21] {2.0 - 8*x}
    node[pos=1, anchor= west, outer sep = 2pt, inner sep=0pt]
    {\tiny{slope: $g'(0)$}};
    \addplot[black, thick, domain=0:0.6] {2.0 + \normalEta * \gprimeZero * x} node[pos=1, anchor=west, outer sep = 2pt, inner sep=0pt] {\footnotesize$\ell(\alpha) = f\bigl(\vect{x}^{(k)}\bigr) + \eta\, g'(0)\,\alpha$};
    \addplot[draw=none, domain=0:0.21, samples=2] {2.0 + \normalEta * \gprimeZero * x} node[pos=1, anchor= west, inner sep=0pt, outer sep=2pt, yshift=1mm] {\tiny{slope: $\eta\,g'(0)$}};
    \pgfmathsetmacro{\barEta}{1 - 3*\alphaL}
    \pgfmathsetmacro{\slopeEta}{\normalEta    * \gprimeZero}  
    \pgfmathsetmacro{\slopeBar}{\barEta * \gprimeZero}
    \pgfmathsetmacro{\gAlphaL}{2.0 - 8*\alphaL + 12*\alphaL*\alphaL}
    \pgfmathsetmacro{\alphaU}{(1 - \normalEta)*abs(\gprimeZero)/12}
    \pgfmathsetmacro{\gAlphaU}{2.0 - 8*\alphaU + 12*\alphaU*\alphaU}
    \addplot[black, dashed, domain=0:0.21, samples=2] {2.0 + \slopeBar*x} node[pos=1, anchor=west, inner sep=0pt, outer sep=2pt] {\tiny{slope: $\bar{\eta}\,g'(0)$}};
    \node[circle,fill=black,inner sep=1pt] at (axis cs:\alphaL,\gAlphaL) {};
    \draw[dotted] (axis cs:\alphaL,\gAlphaL) -- (axis cs:\alphaL,0);
    \addplot[black, domain={\alphaL-0.05}:{\alphaL+0.05}, samples=2] {\gAlphaL + \slopeBar*(x - \alphaL)};
    \node[circle,fill=black,inner sep=1pt] at (axis cs:\alphaU,\gAlphaU) {};
    \draw[dotted] (axis cs:\alphaU,\gAlphaU) -- (axis cs:\alphaU,0);
    \draw (axis cs:\alphaL,0) -- (axis cs:\alphaL,-0.08);
    \draw (axis cs:\alphaU,0) -- (axis cs:\alphaU,-0.08);
    \draw[->, thick] (axis cs:\alphaL,0) node[anchor=north west] {\tiny \eqref{eq:second_wolfe_condition} holds} -- (axis cs:{\alphaL + 0.05},0);
    \draw[->, thick] (axis cs:\alphaU,0) node[anchor=north east] {\tiny \eqref{eq:first_wolfe_condition} holds} -- (axis cs:{\alphaU - 0.05},0);
  \end{axis}

\fi

\end{tikzpicture}
\]
\tikzexternalenable%
The Wolfe conditions are two conditions that guide the choice of a “good enough” step size \(\alpha^{(k)}\) at each iteration of a line search method to achieve global convergence. Consider the affine function
\[
\ell(\alpha) = f\bigl(\vect{x}^{(k)}\bigr) + \eta\,\alpha\, \nabla f\bigl(\vect{x}^{(k)}\bigr)^T \vect{d}^{(k)} %\text{,}
\]
with \(\eta \in (0,1)\) a relaxation parameter (sometimes called the \emph{relaxed tangent}). The first Wolfe condition (also called the \emph{Armijo} or \emph{sufficient decrease condition}) stipulates that \(g(\alpha^{(k)})\) should be no larger than \(\ell(\alpha^{(k)})\).
Hence the Armijo condition requires that the decrease in \(f\) is at least proportional to both \(\alpha^{(k)}\) and \(\nabla f\bigl(\vect{x}^{(k)}\bigr)^T \vect{d}^{(k)}\). In practice one typically chooses \(\eta\) to be rather small (e.g., \(\eta = 10^{-4}\) or \(10^{-2}\)).

The second Wolfe condition (also called the \emph{curvature condition}) discards step sizes that are too small, thereby avoiding very slow progress.

\begin{definition}[First Wolfe Condition: Armijo or Sufficient Decrease Condition]
\label{def:first_wolfe_condition}
Given a point \(\vect{x}^{(k)}\), a direction \(\vect{d}^{(k)}\) and a parameter \(\eta \in (0,1)\), the step size \(\alpha^{(k)}>0\) in a line search method should verify
\begin{equation}
\label{eq:first_wolfe_condition}
{\color{Red}
\setlength{\fboxrule}{1pt}
\boxed{ 
\color{black}
g\bigl(\alpha^{(k)}\bigr) 
=
f\bigl(\vect{x}^{(k)} + \alpha^{(k)}\,\vect{d}^{(k)}\bigr) 
\le 
f\bigl(\vect{x}^{(k)}\bigr) + \eta\, \alpha^{(k)}\, \nabla f\bigl(\vect{x}^{(k)}\bigr)^T \vect{d}^{(k)} 
= 
\ell\bigl(\alpha^{(k)}\bigr) %\text{.}
}
}
\end{equation}
\end{definition}

\begin{definition}[Second Wolfe Condition: Curvature Condition]
\label{def:second_wolfe_condition}
Given a point \(\vect{x}^{(k)}\), a descent direction \(\vect{d}^{(k)}\), and a parameter \(\bar{\eta} \in (\eta, 1)\), the step size \(\alpha^{(k)}\) should verify
\begin{equation}
\label{eq:second_wolfe_condition}
{\color{Red}
\setlength{\fboxrule}{1pt}
\boxed{ 
\color{black}
g'\bigl(\alpha^{(k)}\bigr) 
= 
\nabla f\bigl(\vect{x}^{(k)} + \alpha^{(k)}\,\vect{d}^{(k)}\bigr)^T \vect{d}^{(k)} 
\ge 
\bar{\eta}\, \nabla f\bigl(\vect{x}^{(k)}\bigr)^T \vect{d}^{(k)}
=
\bar{\eta}\,g'\bigl(0\bigr)
%\text{.}
}
}
\end{equation}
\end{definition}

Note that the left-hand side of the above inequality is \(g'\bigl(\alpha^{(k)}\bigr)\) while the right-hand side is \(\bar{\eta}\,g'(0)\) (and \(g'(0) < 0\) since \(\vect{d}^{(k)}\) is a descent direction). In other words, the curvature condition ensures that the derivative of \(g\) at \(\alpha^{(k)}\) is not too small in magnitude; otherwise, one could (and should) take a longer step.

The Armijo and curvature conditions together are known as the \emph{Wolfe conditions}. Although they give criteria for selecting a “good” \(\alpha^{(k)}\), they do not, by themselves, indicate how to find such an \(\alpha^{(k)}\). One popular approach is the backtracking line search algorithm.

%%%%%%%%%%%%%%%%%%%%%%%%%%%%%%%%%%%%%%%%%%%%%%%%%%%%%%%%%%%%%%%%%%%%%%%%%%%%%%%
% 5.2 Backtracking Line Search
%%%%%%%%%%%%%%%%%%%%%%%%%%%%%%%%%%%%%%%%%%%%%%%%%%%%%%%%%%%%%%%%%%%%%%%%%%%%%%%

\subsubsection{Backtracking Line Search}

The main idea behind backtracking line search (Algorithm \ref{alg:backtracking-line-search}) is to start from a “large” initial value \(\bar{\alpha}\) (e.g. \(\bar{\alpha}=10\)) and then decrease it until the Armijo condition is met.

\begin{algorithm}[h]
\caption{Backtracking Line Search}\label{alg:backtracking-line-search}
\begin{algorithmic}[1]
    \State \textbf{Given:} Current point \(\vect{x}^{(k)}\), descent direction \(\vect{d}^{(k)}\)
    \State \textbf{Parameters:} Initial trial step size \(\bar{\alpha} > 0\), relaxation parameter \(\eta \in (0,1)\)
    \State Initialize: Set \(\alpha^{(k)} \leftarrow \bar{\alpha}\)
    % \While{\(f\bigl(\vect{x}^{(k)} + \alpha^{(k)}\,\vect{d}^{(k)}\bigr) > f\bigl(\vect{x}^{(k)}\bigr) + \eta\,\alpha^{(k)}\, \nabla f\bigl(\vect{x}^{(k)}\bigr)^T \vect{d}^{(k)}\)}
    \While{\(g(\alpha^{(k)}) > \ell(\alpha^{(k)})\)}
        \State \(\alpha^{(k)} \leftarrow \rho\,\alpha^{(k)}\) \Comment{with \(\rho \in (0,1)\); often \(\rho = \frac{1}{2}\) is used}
    \EndWhile
    \State \textbf{Output:} \(\alpha^{(k)}\)
\end{algorithmic}
\end{algorithm}

The backtracking line search algorithm can be incorporated into both Newton's and Gradient Descent algorithms to choose the step size at each iteration. 
A general line search procedure is given in Algorithm~\ref{alg:general-line-search}.
% In this algorithm, \(\vect{d}^{(k)} \overset{\text{Alg. \ref{alg:gd}}}{=}-\nabla f(\vect{x}^{(k)})\) or \(\vect{d}^{(k)} \overset{\text{Alg. \ref{alg:newton_multivariate}}}{=}-\big(\matr{H}_{f}(\vect{x}^{(k)})\big)^{-1}\nabla f(\vect{x}^{(k)})\) or any other descent direction.
In this algorithm, \(\vect{d}^{(k)} = -\nabla f(\vect{x}^{(k)})\) (Gradient Descent; Algorithm \ref{alg:gd}) or \(\vect{d}^{(k)} = -\big(\matr{H}_{f}(\vect{x}^{(k)})\big)^{-1}\nabla f(\vect{x}^{(k)})\) (Newton's Method; Algorithm \ref{alg:newton_multivariate}) or any other descent direction.


\begin{algorithm}[h]
\caption{General Line Search Algorithm}\label{alg:general-line-search}
\begin{algorithmic}[1]
    \State \textbf{Given:} Starting point \(\vect{x}^{(0)}\)
    \State Initialize: Set \(k \leftarrow 0\)
    \While{\(\|\nabla f(\vect{x}^{(k)})\| > \epsilon\) and \(k \leq k_{\max}\)}
        \State Compute search direction \(\vect{d}^{(k)}\) 
        % \Comment{\(\vect{d}^{(k)} = -\nabla f(\vect{x}^{(k)})\) for GD, or \(\vect{d}^{(k)} = -\big(\matr{H}_{f}(\vect{x}^{(k)})\big)^{-1}\nabla f(\vect{x}^{(k)})\) for Newton's method}
        % \Comment{\(\vect{d}^{(k)} = \begin{cases}
        %     -\nabla f(\vect{x}^{(k)}) & \text{for Algorithm~\ref{alg:gd}} \\
        %     -\big(\matr{H}_{f}(\vect{x}^{(k)})\big)^{-1}\nabla f(\vect{x}^{(k)}) & \text{for Algorithm~\ref{alg:newton_multivariate}}
        % \end{cases}\)}
        \Comment{e.g. using \(-\nabla f(\vect{x}^{(k)})\) or \(-\big(\matr{H}_{f}(\vect{x}^{(k)})\big)^{-1}\nabla f(\vect{x}^{(k)})\)}
        \State Find \(\alpha^{(k)}\) using the backtracking line search (Algorithm~\ref{alg:backtracking-line-search})
        \State Update: \(\vect{x}^{(k+1)} = \vect{x}^{(k)} + \alpha^{(k)}\,\vect{d}^{(k)} \)
        \State Increment: \(k \leftarrow k+1\)
    \EndWhile
    \State \textbf{Output:} \(\vect{x}^{(k)}\)
\end{algorithmic}
\end{algorithm}

%%%%%%%%%%%%%%%%%%%%%%%%%%%%%%%%%%%%%%%%%%%%%%%%%%%%%%%%%%%%%%%%%%%%%%%%%%%%%%%
% 5.3 Global Convergence Theorem
%%%%%%%%%%%%%%%%%%%%%%%%%%%%%%%%%%%%%%%%%%%%%%%%%%%%%%%%%%%%%%%%%%%%%%%%%%%%%%%

\subsubsection{Global Convergence Theorem}

With the addition of the line search sub-routine, it is possible to prove that Newton's algorithm is globally convergent. In fact, a more general result holds for any line search method.

\begin{theorem}[Zoutendijk's Theorem]\label{thm:zoutendijk}
Assume that:
\begin{itemize}
  \item \(f:\mathbb{R}^{n}\to\mathbb{R}\) is continuously differentiable, \(L\)-smooth with \(L>0\), and bounded from below;
  \item For each iteration \(k\), the search direction \(\vect{d}^{(k)}\) is a descent direction and the step size \(\alpha^{(k)}\) satisfies the Wolfe conditions \eqref{eq:first_wolfe_condition} and \eqref{eq:second_wolfe_condition}.
\end{itemize}
Then
\[
\sum_{k \geq 0} \cos^2\theta^{(k)}\, \|\nabla f(\vect{x}^{(k)})\|^2 < \infty %\text{,}
\]
where \(\theta^{(k)}\) is the angle between \(\vect{d}^{(k)}\) and \(-\nabla f(\vect{x}^{(k)})\), i.e.,
\[
\cos\theta^{(k)} = \frac{\nabla f(\vect{x}^{(k)})^T \vect{d}^{(k)}}{\|\nabla f(\vect{x}^{(k)})\|\,\|\vect{d}^{(k)}\|} %\text{.}
\]
\end{theorem}

\begin{proof}
Let \(k \ge 0\). Since \(\alpha^{(k)}\) satisfies the curvature condition for some \(0 < \bar{\eta} < 1\), we have
\[
\nabla f\bigl(\vect{x}^{(k+1)}\bigr)^T \vect{d}^{(k)} = \nabla f\bigl(\vect{x}^{(k)} + \alpha^{(k)}\, \vect{d}^{(k)}\bigr)^T \vect{d}^{(k)} \ge \bar{\eta}\, \nabla f\bigl(\vect{x}^{(k)}\bigr)^T \vect{d}^{(k)} %\text{,}
\]
which can be rewritten as
\[
\Bigl(\nabla f\bigl(\vect{x}^{(k+1)}\bigr)-\nabla f\bigl(\vect{x}^{(k)}\bigr)\Bigr)^T \vect{d}^{(k)} \ge (\bar{\eta} - 1)\, \nabla f\bigl(\vect{x}^{(k)}\bigr)^T \vect{d}^{(k)} %\text{,}
\]
and, since \(\nabla f\) is \(L\)-Lipschitz and by the Cauchy-Schwarz inequality, we have
\[
\Bigl(\nabla f\bigl(\vect{x}^{(k+1)}\bigr)-\nabla f\bigl(\vect{x}^{(k)}\bigr)\Bigr)^T \vect{d}^{(k)} \le L\, \alpha^{(k)}\, \|\vect{d}^{(k)}\|^2 %\text{.}
\]
Combining the two inequalities, we obtain
\[
\alpha^{(k)} \ge \frac{(\bar{\eta}-1)\, \nabla f\bigl(\vect{x}^{(k)}\bigr)^T \vect{d}^{(k)}}{L\, \|\vect{d}^{(k)}\|^2} %\text{.}
\]
Since \(\vect{d}^{(k)}\) is a descent direction and \(\alpha^{(k)}\) satisfies the Armijo condition for some \(\eta \in (0,\bar{\eta})\), it follows that
\[
\begin{aligned}
f\bigl(\vect{x}^{(k+1)}\bigr) &= f\bigl(\vect{x}^{(k)} + \alpha^{(k)}\, \vect{d}^{(k)}\bigr) \\
&\le f\bigl(\vect{x}^{(k)}\bigr) + \eta\, \alpha^{(k)}\, \nabla f\bigl(\vect{x}^{(k)}\bigr)^T \vect{d}^{(k)} \\
&\le f\bigl(\vect{x}^{(k)}\bigr) - \frac{\eta\, (\bar{\eta}-1)}{L}\, \frac{\bigl|\nabla f\bigl(\vect{x}^{(k)}\bigr)^T \vect{d}^{(k)}\bigr|^2}{\|\vect{d}^{(k)}\|^2} \\
&= f\bigl(\vect{x}^{(k)}\bigr) - \frac{\eta\, (\bar{\eta}-1)}{L}\, \cos^2\theta^{(k)}\, \|\nabla f(\vect{x}^{(k)})\|^2 %\text{.}
\end{aligned}
\]
If we denote \(c = \frac{\eta\, (\bar{\eta}-1)}{L}\) and sum these inequalities over \(k\), we obtain
\[
f\bigl(\vect{x}^{(K)}\bigr) \le f\bigl(\vect{x}^{(0)}\bigr) - c \sum_{k=0}^{K} \cos^2\theta^{(k)}\, \|\nabla f(\vect{x}^{(k)})\|^2 %\text{.}
\]
Since \(f\) is bounded from below and \(c > 0\), it follows that
\[
\sum_{k \geq 0} \cos^2\theta^{(k)}\, \|\nabla f(\vect{x}^{(k)})\|^2 < \infty %\text{.}
\]
\end{proof}

An immediate consequence of Theorem~\ref{thm:zoutendijk} is that
\[
\cos^2\theta^{(k)}\, \|\nabla f(\vect{x}^{(k)})\|^2 \to 0 %\text{.}
\]
If the cosine terms are bounded away from zero (i.e., there exists \(\delta > 0\) such that \(\cos\theta^{(k)} \ge \delta\) for all \(k\)), then this implies \(\|\nabla f(\vect{x}^{(k)})\| \to 0\); in other words, the iterates are attracted to a stationary point. For example:
\begin{itemize}
  \item In Gradient Descent, where \(\vect{d}^{(k)} = -\nabla f(\vect{x}^{(k)})\), we have \(\cos\theta^{(k)} = -1\).
  \item In Newton's method, when 
  \[
  \vect{d}^{(k)} = -\big(\matr{H}_{f}(\vect{x}^{(k)})\big)^{-1}\nabla f(\vect{x}^{(k)}),
  \]
  one may show that
    \[
    \cos\theta^{(k)} \ge \frac{\lambda_{\min}\bigl(\matr{H}_{f}(\vect{x}^{(k)})\bigr)}{\lambda_{\max}\bigl(\matr{H}_{f}(\vect{x}^{(k)})\bigr)} %\text{,}
    \]
    where \(\lambda_{\min}\bigl(\matr{H}_{f}(\vect{x}^{(k)})\bigr)\) and \(\lambda_{\max}\bigl(\matr{H}_{f}(\vect{x}^{(k)})\bigr)\) denote the smallest and largest eigenvalues of \(\matr{H}_{f}(\vect{x}^{(k)})\). Hence if the condition number 
    \[
    \kappa\bigl(\matr{H}_{f}(\vect{x}^{(k)})\bigr) = \frac{\lambda_{\max}\bigl(\matr{H}_{f}(\vect{x}^{(k)})\bigr)}{\lambda_{\min}\bigl(\matr{H}_{f}(\vect{x}^{(k)})\bigr)}
    \]
    is uniformly bounded, then \(\cos\theta^{(k)}\) is bounded away from zero.
\end{itemize}

Thus, under appropriate assumptions, the line search methods (whether steepest descent, Newton's method with line search, or quasi-Newton methods with line search) yield
\[
\|\nabla f(\vect{x}^{(k)})\| \to 0 %\text{.}
\]
Note, however, that this global convergence result does not necessarily imply that the iterates converge to a local minimizer without additional assumptions on the Hessian of \(f\).

%%%%%%%%%%%%%%%%%%%%%%%%%%%%%%%%%%%%%%%%%%%%%%%%%%%%%%%%%%%%%%%%%%%%%%%%%%%%%%%
% 6 Variations of Newton's Method
%%%%%%%%%%%%%%%%%%%%%%%%%%%%%%%%%%%%%%%%%%%%%%%%%%%%%%%%%%%%%%%%%%%%%%%%%%%%%%%

\subsection{Variations of Newton's Method}

Even with the addition of a line search, there remain two main drawbacks of Newton's algorithm:
\begin{itemize}
  \item The update is well defined only if the Hessian matrix \(\matr{H}_{f}(\vect{x}^{(k)})\) is positive definite,
  \item It requires the computation of second-order derivatives, which can be computationally expensive.
\end{itemize}
We now discuss possible solutions.

%%%%%%%%%%%%%%%%%%%%%%%%%%%%%%%%%%%%%%%%%%%%%%%%%%%%%%%%%%%%%%%%%%%%%%%%%%%%%%%
% 6.1 Newton with Hessian Correction
%%%%%%%%%%%%%%%%%%%%%%%%%%%%%%%%%%%%%%%%%%%%%%%%%%%%%%%%%%%%%%%%%%%%%%%%%%%%%%%

\subsubsection{Newton with Hessian Correction}

Recall that if 
\(
\matr{H}_{f}(\vect{x}^{(k)}) 
\)
is not positive definite, then the Newton direction
\[
\vect{d}^{(k)} = -\big(\matr{H}_{f}(\vect{x}^{(k)})\big)^{-1}\nabla f(\vect{x}^{(k)}) %\text{,}
\]
may not be well defined or may fail to be a descent direction. A common remedy is to \emph{correct} the Hessian by adding a multiple of the identity matrix. That is, one seeks \(\lambda > 0\) such that
\[
\matr{H}_{f}(\vect{x}^{(k)}) + \lambda \matr{I} %\text{,}
\]
is positive definite.

A practical method for checking whether a symmetric matrix is positive definite is to attempt its Cholesky decomposition. % (e.g., using \texttt{np.linalg.cholesky} in \texttt{numpy}). 
If the decomposition fails, then the matrix is not positive definite. 
Recall that the Cholesky decomposition of a positive definite matrix \(\matr{H}\) is a factorization of the form
\[
\matr{H} = \matr{L}\,\matr{L}^T %\text{,}
\]
where \(\matr{L}\) is a lower triangular matrix (the Cholesky factor). 
This decomposition is widely used to solve linear systems \(\matr{H}\,\vect{x} = \vect{b}\) (by first solving \(\matr{L}\,\vect{v}=\vect{b}\) via forward substitution and then \(\matr{L}^T\,\vect{x}=\vect{v}\) via backward substitution) and to compute the inverse by solving \(\matr{A}\,\matr{X} = \matr{I}\).

Algorithm \ref{alg:hessian-correction} describes a procedure for finding \(\lambda\) (the Hessian correction parameter) using the Cholesky decomposition.

\begin{algorithm}[h]
\caption{Hessian Correction}\label{alg:hessian-correction}
\begin{algorithmic}[1]
    \State \textbf{Given:} Hessian matrix \(\matr{H}_{f}(\vect{x})\)
    \State \textbf{Parameters:} Initial correction parameter \(\bar{\lambda}>0\), increase factor \(c \in (0,1)\)
    \State Try to compute the Cholesky factor \(\matr{L}\) of \(\matr{H}_{f}(\vect{x})\)
    \If{successful}
        \State \textbf{Return} \(\matr{L}\)
    \EndIf
    \State Set \(\lambda \leftarrow \bar{\lambda}\)
    \State Try to compute the Cholesky factor \(\matr{L}\) of \(\matr{H}_{f}(\vect{x}) + \lambda \matr{I}\)
    \While{not successful}
        \State Increase \(\lambda \leftarrow c\,\lambda\)
        \State Try to compute the Cholesky factor \(\matr{L}\) of \(\matr{H}_{f}(\vect{x}) + \lambda \matr{I}\)
    \EndWhile
    \State \textbf{Output:} \(\matr{L}\), the Cholesky factor of the corrected Hessian %\text{.}
\end{algorithmic}
\end{algorithm}

Using the Hessian correction subroutine above, one can incorporate it into Newton's method. The overall algorithm with line search and Hessian correction is given in Algorithm~\ref{alg:newton-hessian-corr}.

\begin{algorithm}[h]
\caption{Newton's Algorithm with Line Search and Hessian Correction}\label{alg:newton-hessian-corr}
\begin{algorithmic}[1]
    \State \textbf{Given:} Starting point \(\vect{x}^{(0)}\)
    \State Initialize: \(k \leftarrow 0\)
    \While{\(\|\nabla f(\vect{x}^{(k)})\| > \epsilon\) and \(k \le k_{\max}\)}
        \State Compute \(\vect{g}^{(k)} = \nabla f(\vect{x}^{(k)})\) and \(\matr{H}_{f}(\vect{x}^{(k)})\)
        \State Compute the Cholesky factor \(\matr{L}\) of \(\matr{H}_{f}(\vect{x}^{(k)})\) using the Hessian Correction subroutine (Algorithm~\ref{alg:hessian-correction})
        \State Solve \(\matr{L}\,\vect{v} = -\vect{g}^{(k)}\) (forward substitution) 
        \State Solve \(\matr{L}^T\,\vect{d}^{(k)} = \vect{v}\) (backward substitution) 
        \State Find \(\alpha^{(k)}\) using a line search procedure (e.g., Algorithm~\ref{alg:backtracking-line-search})
        \State Update: \(\vect{x}^{(k+1)} = \vect{x}^{(k)} + \alpha^{(k)}\,\vect{d}^{(k)}\) 
        \State Increment: \(k \leftarrow k + 1\)
    \EndWhile
    \State \textbf{Output:} \(\vect{x}^{(k)}\) %\text{.}
\end{algorithmic}
\end{algorithm}

%%%%%%%%%%%%%%%%%%%%%%%%%%%%%%%%%%%%%%%%%%%%%%%%%%%%%%%%%%%%%%%%%%%%%%%%%%%%%%%
% 6.2 Quasi-Newton Methods
%%%%%%%%%%%%%%%%%%%%%%%%%%%%%%%%%%%%%%%%%%%%%%%%%%%%%%%%%%%%%%%%%%%%%%%%%%%%%%%

\subsubsection{Quasi-Newton Methods}

Quasi-Newton methods aim to capture some of the advantages of Newton's method while avoiding the expensive computation of second-order derivatives. The main idea is to replace the Hessian \(\matr{H}_{f}(\vect{x}^{(k)})\) by an approximation that is updated using gradient information only.

A natural starting point is the first-order Taylor expansion of \(\nabla f\) at \(\vect{x}^{(k)}\):
\[
\nabla f\bigl(\vect{x}^{(k+1)}\bigr) \approx \nabla f\bigl(\vect{x}^{(k)}\bigr) + \matr{H}_{f}(\vect{x}^{(k)})\,\Bigl(\vect{x}^{(k+1)} - \vect{x}^{(k)}\Bigr) %\text{,}
\]
which can be rearranged to yield the \emph{secant equation}
\[
\matr{H}_{f}(\vect{x}^{(k)})\, \underbrace{\Bigl(\vect{x}^{(k+1)} - \vect{x}^{(k)}\Bigr)}_{\vect{s}^{(k)}} \approx \nabla f\bigl(\vect{x}^{(k+1)}\bigr) - \nabla f\bigl(\vect{x}^{(k)}\bigr) \triangleq \vect{y}^{(k)} %\text{.}
\]
That is, we wish to have
\[
\tilde{\matr{H}}^{(k+1)}\, \vect{s}^{(k)} = \vect{y}^{(k)} %\text{,}
\]
where \(\tilde{\matr{H}}^{(k+1)}\) is a symmetric approximation of the Hessian. Since a symmetric \(n \times n\) matrix has \(\frac{n(n+1)}{2}\) free parameters, the system is underdetermined and there exist infinitely many solutions. One usually chooses the update so that \(\tilde{\matr{H}}^{(k+1)}\) is positive definite (to guarantee that the resulting search direction is a descent direction).

One of the earliest quasi-Newton methods is the DFP (Davidon-Fletcher-Powell) method. Its update is given by
\[
\tilde{\matr{H}}^{(k+1)} = \Biggl(\matr{I} - \frac{\vect{y}^{(k)}\, \vect{s}^{(k)T}}{\rho^{(k)}}\Biggr)^T \tilde{\matr{H}}^{(k)} \Biggl(\matr{I} - \frac{\vect{y}^{(k)}\, \vect{s}^{(k)T}}{\rho^{(k)}}\Biggr) + \frac{\vect{y}^{(k)}\,\vect{y}^{(k)T}}{\rho^{(k)}} %\text{,}
\]
with
\(
\rho^{(k)} = \vect{y}^{(k)T} \vect{s}^{(k)} %\text{.}
\)

A more popular approach is the BFGS (Broyden-Fletcher-Goldfarb-Shanno) method, which instead updates an approximation \(\matr{B}^{(k)}\) of the inverse Hessian. By considering the first-order Taylor expansion of \(\nabla f\), one may wish to satisfy
\[
\matr{B}^{(k+1)}\, \vect{y}^{(k)} = \vect{s}^{(k)} %\text{,}
\]
and the BFGS update is given by
\[
\matr{B}^{(k+1)} = \matr{B}^{(k)} - \frac{\matr{B}^{(k)}\, \vect{s}^{(k)}\, \vect{s}^{(k)T}\, \matr{B}^{(k)}}{\vect{s}^{(k)T}\matr{B}^{(k)}\vect{s}^{(k)}} + \frac{\vect{y}^{(k)}\, \vect{y}^{(k)T}}{\rho^{(k)}} %\text{,}
\]
with
\(
\rho^{(k)} = \vect{y}^{(k)T} \vect{s}^{(k)} %\text{.}
\)
















































\clearpage
\section{Constrained Optimization}\label{sec:constrained_optimization}
\subsection{Introduction and Terminology}\label{sec:intro_constrained_optimization}
Recall Equation~\eqref{eq:form_optimization_problem}, the general formulation of a constrained optimization problem, the definition of the feasible set~\eqref{eq:feasible_set}, the definition of an affine function (Definition \ref{def:affine_function}) and the types of minimizers (Section~\ref{subsec:global_local_optimization}, Definitions~\ref{def:global_minimizer}, \ref{def:local_minimizer}, \ref{def:strict_local_minimizer}).
% \vspace{-\parskip}
% \[
%  \min_{\vect{x} \in \mathbb{R}^{n}} f(\vect{x}) \quad \text{subject to} \quad \begin{cases}c_{i}(\vect{x})=0, & i \in \mathcal{E} \\ c_{i}(\vect{x}) \geq 0, & i \in \mathcal{I}\end{cases}
% \]
% \[
% \Omega = \left\{ \vect{x}\in\mathbb{R}^{n} : c_{i}(\vect{x})=0,\; i\in\mathcal{E},\; c_{i}(\vect{x})\geq 0,\; i\in\mathcal{I} \right\}
% \]

\subsubsection{Classes of Constrained Optimization Problems}
Depending on the type and form of $f$ and $c_i$'s, the type of constrained problem can be radically different and require different classes of optimization algorithms:
\begin{itemize}
  \item $f$ and $c_i$'s are \emph{affine} functions: Linear Programming problem (Section~\ref{sec:linear_programming})
  \item $f$ is \emph{quadratic} and all $c_i$'s are \emph{affine}: Quadratic Programming problem
  \item $f$ is \emph{nonlinear}: Nonlinear Programming problem
  \item $f$ is \emph{convex}, all inequality constraints $c_{i,i\in\mathcal{I}}$ are \emph{convex}, and all equality constraints $c_{i,i\in\mathcal{E}}$ are \emph{affine}: Convex Problem
\end{itemize}

\begin{lemma}[Properties of Convex Problems]\label{lem:properties_convex_problem}
  We consider a constrained optimization problem of the form \eqref{eq:form_optimization_problem}. 
  If the problem is convex, then
  \begin{itemize}
    \item its feasible region \(\Omega\) is a convex set (Definition \ref{def:convex_set})
    \item any local minimizer is also a global minimizer \qedhere
 \end{itemize}
\end{lemma}

\subsubsection{Active Constraints}\label{sec:active_constraints}
are an important concept and in some cases allow us to reduce the number of constraints.

\begin{definition}[Active Set]\label{def:active_set}
  The active set \(\mathcal{A}\) at any feasible \(\vect{x}\) consists of the indices of the constraints that are satisfied with equality at \(\vect{x}\):
  \[
  \mathcal{A}(\vect{x}) = \mathcal{E} \cup \{i \in \mathcal{I} \mid c_i(\vect{x}) = 0\}% \text{.}
  \]
  At a feasible point \(\vect{x}\), the inequality constraint \(i\) is said to be \emph{active} if \(c_i(\vect{x}) = 0\) and \emph{inactive} if the strict inequality \(c_i(\vect{x}) > 0\) is satisfied.
\end{definition}
Active inequality constraints often play a similar role as equality constraints.

If \(\vect{x}^{\ast}\) is a local minimizer, then if the non-active constraints at \(\vect{x}^{\ast}\) were removed, \(\vect{x}^{\ast}\) would remain a local minimizer.



\subsubsection{Feasible Directions}
We need to design algorithms that produce iterates that stay within the feasible set.
That is why we need the concept of \emph{feasible directions}, which tells us in which directions we are allowed to move from a feasible point.
\begin{definition}[Feasible direction]\label{def:feasible_direction}
  Let \(\vect{x}\in\Omega\). A vector \(\vect{d}\in\R^{n}\) is a \emph{feasible direction} at \(\vect{x}\) if there exists \(\varepsilon>0\) such that \(\vect{x}+t\vect{d}\in\Omega,\; \forall t\in(0,\varepsilon]\).
  In other words, by doing a small step in the direction \(\vect{d}\) from \(\vect{x}\), we stay in \(\Omega\).
\end{definition}

\begin{definition}[Set of linearized feasible directions]\label{def:linearized_feasible_directions}
  Given a feasible point \(\vect{x}\in\Omega\), the \emph{set of linearized feasible directions} is
  \begin{equation}\label{eq:linearized_feasible_directions}
    \mathcal{F}(\vect{x}) := \left\{ \vect{d}\in\R^{n} \mid \nabla c_{i}(\vect{x})^{\top}\vect{d}=0 \; \forall i\in\mathcal{E} \land \nabla c_{i}(\vect{x})^{\top}\vect{d}\ge 0 \; \forall i\in\mathcal{I}\cap\mathcal{A}(\vect{x}) \right\}
  \end{equation}
\end{definition}
Note that the inactive constraints are not relevant for finding the feasible directions.

\begin{definition}[Feasible Descent Direction]\label{def:feasible_descent_direction}
  A direction \(\vect{d}\in\mathcal{F}(\vect{x})\) is a feasible descent direction at \(\vect{x} \in \Omega\) if \(\nabla f(\vect{x})^{\top}\vect{d} < 0\).
\end{definition}

\subsubsection{Constraint Qualification Condition}

\begin{definition}[Constraint Qualification Condition]\label{def:constraint_qualification_condition}
The Constraint Qualification Condition holds when all constraints are affine or if the set of active constraint gradients \(\{ \nabla c_{i}(\vect{x}) \mid i\in\mathcal{A}(\vect{x}) \}\) is linearly independent and \(\vect{x} \in \Omega\).
\end{definition}

% \glsreset{licq}
% \begin{definition}[\acrlong{licq}]\label{def:LICQ}
%   \gls{licq} holds at a feasible point \(\vect{x}\in\Omega\) if the set of active constraint gradients
%   \(\{ \nabla c_{i}(\vect{x}) \mid i\in\mathcal{A}(\vect{x}) \}\)
%   is linearly independent.
% \end{definition}

\subsection{Optimality Conditions} 

\subsubsection{Necessary First-Order Conditions}
In \nameref{sec:unconstrained_optimization}, a necessary condition is that if $\vect{x}^*$ is a local minimizer then there is no descent direction at $\vect{x}^*$, i.e.,
\[
\nabla f\left(\vect{x}^*\right)^T \vect{d} \geq 0
\]
for \textcolor{red}{all} directions \(\vect{d} \in \textcolor{red}{\mathbb{R}^n}\), which implies the \nameref{thm:nec_first_order_optimality_multivariate} $\nabla f\left(\vect{x}^*\right)=0$.
In constrained problems, a similar necessary condition holds:

\begin{lemma}[Fundamental Necessary Condition]\label{lem:fundamental_condition}
  If \(\vect{x}^{\ast}\) is a local minimizer and the \nameref{def:constraint_qualification_condition} holds at \(\vect{x}^{\ast}\), then \(\vect{x}^{\ast} \in \Omega\) and there exist no \nameref{def:feasible_descent_direction} at $\vect{x}^*$, i.e.,
  \[
    \nabla f(\vect{x}^{\ast})^{\top} \vect{d} \ge 0
  \]
  for \textcolor{red}{all feasible} directions \(\vect{d}\in\textcolor{red}{\mathcal{F}(\vect{x}^{\ast})}\).
  Since we would need to examine every \(\vect{d}\in\mathcal{F}(\vect{x}^{\ast})\), this condition is not very practical.
\end{lemma}

\begin{theorem}[\gls{kkt}]\label{thm:KKT}
Suppose \(\vect{x}^{\ast}\) is a local minimizer of \eqref{eq:form_optimization_problem} where \(f\) and the \(c_i\)'s are continuously differentiable and the \nameref{def:constraint_qualification_condition} holds at \(\vect{x}^{\ast}\). 
Then there exists a unique \(\vect{\lambda}^{\ast}\in\R^{|\mathcal{E}|+|\mathcal{I}|}\) such that
\begin{subequations}\label{eq:KKT}
  \begin{align}
    &\qquad\qquad\qquad&\nabla f(\vect{x}^{\ast}) - \sum_{i\in\mathcal{E}\cup\mathcal{I}} \lambda_{i}^{\ast} \nabla c_{i}(\vect{x}^{\ast}) &= \vect{0}   &&\text{(stationarity)} \label{eq:KKT_stationarity} \\ % stationarity
    &\qquad\qquad\qquad&c_{i}(\vect{x}^{\ast}) &= 0,                                            \quad \forall i\in\mathcal{E}                             &&\text{(primal feasibility)}                \label{eq:KKT_equal_feas} \\
    &\qquad\qquad\qquad&c_{i}(\vect{x}^{\ast}) &\ge 0,                                          \quad \forall i\in\mathcal{I}                             &&\text{(primal feasibility)}                \label{eq:KKT_inequal_feas} \\
    &\qquad\qquad\qquad&\lambda_{i}^{\ast} &\ge 0,                                              \quad \forall i\in\mathcal{I}                             &&\text{(dual feasibility)}                  \label{eq:KKT_dual_feas} \\
    &\qquad\qquad\qquad&\lambda_{i}^{\ast} \tikzmark{c_sl}c_{i}(\vect{x}^{\ast}) &= 0,   \quad \forall i\in\mathcal{E}\cup\mathcal{I}              &&\text{(complementary slackness)}           \label{eq:KKT_comp_slack} % complementarity slackness
  \end{align}
\tikzexternaldisable%
\begin{tikzpicture}[remember picture,overlay]
\coordinate (c_sl_eq) at (pic cs:c_sl);
\coordinate (c_sl_border) at ($ (current page text area.west |- c_sl_eq)$);
% \node[draw, red, thick, circle, inner sep = 0pt, minimum size=1mm] at (c_sl_border) {}; % for debugging
\node[rectangle,draw,anchor=south west, inner sep=3pt, outer sep=0pt, align=left] at (c_sl_border) {Complementary Slackness \eqref{eq:KKT_comp_slack}:\\
$\;\bullet\; i \in \mathcal{A}(\vect{x}^{\ast}) \Rightarrow c_i(\vect{x}^{\ast})=0$\\
$\;\bullet\; i \notin \mathcal{A}(\vect{x}^{\ast}) \Rightarrow \lambda_i^{\ast}=0$
};
\end{tikzpicture}%
\tikzexternalenable%
\end{subequations}
where \(\vect{\lambda}^{\ast}\) are called the \emph{Lagrange multipliers} at \(\vect{x}^{\ast}\) and the pair \((\vect{x}^{\ast},\vect{\lambda}^{\ast})\) is called a \gls{kkt} point.
\end{theorem}
Defining the \emph{Lagrangian function} as
\begin{equation}
  \mathcal{L}(\vect{x},\vect{\lambda}) := f(\vect{x}) - \sum_{i\in\mathcal{E}\cup\mathcal{I}} \lambda_{i} c_{i}(\vect{x}), \quad (\vect{x},\vect{\lambda})\in\R^{n}\times\R^{|\mathcal{E}|+|\mathcal{I}|}
\end{equation}
the first condition \eqref{eq:KKT_stationarity} can be expressed concisely as
\begin{equation}
\nabla_{\vect{x}} \mathcal{L}(\vect{x}^{\ast},\vect{\lambda}^{\ast}) = \vect{0} \qedhere
\end{equation}
% Equations~\eqref{eq:KKT_stationarity}--\eqref{eq:KKT_comp_slack} are called the \emph{stationarity}, \emph{primal feasibility}, \emph{dual feasibility}, and \emph{complementarity slackness} conditions, respectively.

The last condition \eqref{eq:KKT_comp_slack} implies that if \(c_{i}\) is inactive at \(\vect{x}^{\ast}\), i.e. \(c_{i}(\vect{x}^{\ast}) > 0\), then \(\lambda_{i}^{\ast} = 0\).
Thus, condition \eqref{eq:KKT_stationarity} can be rewritten as
\begin{equation}\label{eq:KKT_stationarity_active}
  \nabla f(\vect{x}^{\ast}) - \sum_{i\in\mathcal{A}(\vect{x}^{\ast})} \lambda_{i}^{\ast} \nabla c_{i}(\vect{x}^{\ast}) = \vect{0}
\end{equation}
Observe that \eqref{eq:KKT_comp_slack} also implies that 
\begin{equation}
  \mathcal{L}(\vect{x}^{\ast},\vect{\lambda}^{\ast}) = f(\vect{x}^{\ast})
\end{equation}



% Satisfaction of the strict complementarity property usually makes it easier for algorithms to determine the active set A(x ∗ ) and converge rapidly to the solution x ∗ .

A special case of \eqref{eq:KKT_comp_slack} is \emph{strict complementarity}, which is satisfied if exactly one of \(\lambda_{i}^{\ast}\) and \(c_{i}(\vect{x}^{\ast})\) is zero for each \(i\in\mathcal{I}\).
In other words, \(\lambda_{i}^{\ast} > 0\) for each \(i\in\mathcal{I} \cap \mathcal{A}(\vect{x}^{\ast})\) and \(\lambda_{i}^{\ast} = 0\) for each \(i\in\mathcal{I} \setminus \mathcal{A}(\vect{x}^{\ast})\).
Satisfaction of the strict complementarity property usually makes it easier for algorithms to determine the active set \(\mathcal{A}(\vect{x}^{\ast})\) and converge rapidly to the solution \(\vect{x}^{\ast}\).

\begin{definition}[Cone]\label{def:cone}
  \(S\subseteq\R^{n}\) is a \emph{cone} if for all \(\vect{x}\in S\) and all \(\alpha \in \R_{\ge 0}\), we have \(\alpha \vect{x} \in S\).
\end{definition}

\begin{lemma}[Farkas]\label{lem:Farkas}
  Let \( \matr{B}\in\R^{n\times m}, \matr{C}\in\R^{n\times p} \) and \( \vect{g}\in\R^{n} \) and let \(K\) be the cone defined by
  \begin{equation}\label{eq:cone_K}
  K : = \left\{ \matr{B} \vect{y} + \matr{C} \vect{w} \mid \vect{y} \in \R^m_{\ge 0}, \vect{w}\in\R^{p} \right\} 
\end{equation}
  where \(\R^m_{\ge 0}\) is the non-negative orthant%
  \footnote{An orthant is the higher-dimensional analogue of a quadrant in the plane or an octant in three dimensions. The non-negative orthant is the generalization of the first quadrant in \(\R^2\).}
  in~\( \R^m \), i.e., the set of vectors in \(\R^m\) whose components are all non-negative.
  Then for any \(\vect{g} \in \R^n\), exactly one of the following conditions holds:
  \begin{enumerate}[label=(\alph*)]
    \item \( \vect{g} \in K \) \label{item:Farkas_in_K}
    \item \( \exists \vect{d}\in\R^{n} \mid \vect{g}^{\top} \vect{d} < 0, \matr{B}^{\top} \vect{d} \ge \vect{0}, \matr{C}^{\top} \vect{d} = \vect{0} \) \qedhere\label{item:Farkas_hyperplane}
  \end{enumerate}
\end{lemma}
  Geometrically Lemma~\ref{lem:Farkas} says that if \(\vect{g} \notin K\), we can find a hyperplane with normal vector \(\vect{d}\) separating \(\vect{g}\) from the cone \(K\) such that \(\vect{g}\) lies on the opposite side of \(\vect{d}\):
\[
\tikzexternaldisable
\begin{tikzpicture}[>=Stealth, scale=0.6, every node/.style={font=\footnotesize}]

  %%% lengths %%%
  \def\bLen{3cm} % length of the b‐arrows
  \def\gLen{2.5cm} % length of the g‐arrows
  \def\dLen{2.5cm} % length of the d‐arrow
  \def\dashedLen{1.1*\bLen} % length of the dashed line
  \def\coneLen{1.2*\bLen}

  %%% Left panel %%%
  \begin{scope}
    \node at (-0.5cm, 1.5cm) {\ref{item:Farkas_in_K}:};
    % shaded cone
    \fill[red!20] 
      (50:\coneLen) -- 
      (35:0) -- 
      (10:\coneLen) -- 
    cycle;
    \node at (40:\bLen) {\(K\)};
    \draw[->, draw=red]   (0,0) -- (50:\bLen);
    \draw[->, draw=red]   (0,0) -- (10:\bLen);
    % b‐vectors
    % \draw[->, draw=red]   (0,0) -- (50:\bLen) node[above]       {$b_1$};
    % \draw[->, draw=red]   (0,0) -- (35:\bLen) node[above right] {$b_2$};
    % \draw[->, draw=red]   (0,0) -- (10:\bLen) node[below right] {$b_3$};
    % g inside the cone at 22°
    \draw[->, draw=black]  (0,0) -- (22:\gLen) node[right] {$\vect{g}$};
  \end{scope}

  %%% Right panel %%%
  \begin{scope}[xshift=9cm]
    \node at (-2cm, 1.5cm) {\ref{item:Farkas_hyperplane}:};
    % same shaded cone
    \fill[red!20] 
      (50:\coneLen) -- 
      (35:0) -- 
      (10:\coneLen) -- 
    cycle;
    \node at (40:\bLen) {\(K\)};
    \draw[->, draw=red]   (0,0) -- (50:\bLen);
    \draw[->, draw=red]   (0,0) -- (10:\bLen);
    % dashed hyperplane at 80°
    \draw[dashed]    (80:-0.5cm) -- (80:\dashedLen);
    % g‐vector at 100°
    \draw[->, draw=black]  (0,0) -- (100:\gLen) node[above] {$\vect{g}$};
    % b‐vectors again
    % \draw[->, draw=red]   (0,0) -- (50:\bLen) node[above]       {$b_1$};
    % \draw[->, draw=red]   (0,0) -- (35:\bLen) node[above right] {$b_2$};
    % \draw[->, draw=red]   (0,0) -- (10:\bLen) node[below right] {$b_3$};
    % d‐vector at –10°
    \draw[->, black] (0,0) -- (-10:\dLen) node[right] {$\vect{d}$};
  \end{scope}

\end{tikzpicture}
\tikzexternalenable
\]


\begin{proof}[Theorem~\ref{thm:KKT}]
  Let \(\vect{x}^{\ast}\) be a local minimizer of \eqref{eq:form_optimization_problem} and assume that the \nameref{def:constraint_qualification_condition} holds at \(\vect{x}^{\ast}\).
  If we define as
  \begin{equation}\label{eq:cone_N}
    N := \left\{ \sum_{i\in\mathcal{A}(\vect{x}^{\ast})} \lambda_{i} \nabla c_{i}(\vect{x}^{\ast}) \;\mid\; \vect{\lambda} \in \R^{|\mathcal{A}(\vect{x}^{\ast})|}, \lambda_{i} \ge 0, i\in \mathcal{A}(\vect{x}^{\ast}) \cap \mathcal{I} \right\}
  \end{equation}
  it is of the form \eqref{eq:cone_K} with \(\matr{B} = \left[ \nabla c_{i}(\vect{x}^{\ast}) \right]_{i\in{\mathcal{I}\cap\mathcal{A}(\vect{x}^{\ast})}}\) and \(\matr{C} = \left[ \nabla c_{i}(\vect{x}^{\ast}) \right]_{i\in{\mathcal{E}}}\).
  
  By \nameref{lem:Farkas}, either \(\nabla f(\vect{x}^{\ast}) \in N\) or there exists \(\vect{d}\in\R^{n}\) such that \(\nabla f(\vect{x}^{\ast})^{\top} \vect{d} < 0\), \(\nabla c_{i}(\vect{x}^{\ast})^{\top} \vect{d} \ge 0\) for all \(i\in\mathcal{A(\vect{x}^{\ast})} \cap \mathcal{I}\), \(\nabla c_{i}(\vect{x}^{\ast})^{\top} \vect{d} = 0\) for all \(i\in\mathcal{E}\).
  In the second case, \(\vect{d}\) would be both a descent direction and a feasible direction. 
  By Lemma~\ref{lem:fundamental_condition}, if \(\vect{x}^{\ast}\) is a local minimizer, then there are no feasible descent directions at \(\vect{x}^{\ast}\) and therefore \ref{item:Farkas_hyperplane} cannot hold, implying that \(\nabla f(\vect{x}^{\ast}) \in N\).

  This means there exists \(\vect{\lambda} \in \R^{|\mathcal{A(\vect{x}^{\ast})|}}\) such that \(\lambda_{i} \ge 0\) for \(i\in\mathcal{A(\vect{x}^{\ast})} \cap \mathcal{I} \) and
  \begin{equation}\label{eq:grad_linear_combination}
    \nabla f(\vect{x}^{\ast}) = \sum_{i\in\mathcal{A}(\vect{x}^{\ast})} \lambda_{i} \nabla c_{i}(\vect{x}^{\ast})% \text{.}
  \end{equation}
  Definining the vector \(\vect{\lambda}^{\ast} \in \R^{|\mathcal{E}|+|\mathcal{I}|}\) as 
  \(
    \lambda_{i}^{\ast} = \begin{cases}
      \lambda_{i} & i \in \mathcal{A}(\vect{x}^{\ast}) \\
      0           & i \in \mathcal{I} \setminus \mathcal{A}(\vect{x}^{\ast})
    \end{cases}
  \),
  we obtain
  \begin{equation}\label{eq:KKT_stationarity_proof}
    \nabla f(\vect{x}^{\ast}) - \sum_{i\in\mathcal{E}\cup\mathcal{I}} \lambda_{i}^{\ast} \nabla c_{i}(\vect{x}^{\ast}) = \vect{0}
  \end{equation}
  proving \eqref{eq:KKT_stationarity}.
  
  \eqref{eq:KKT_equal_feas} and \eqref{eq:KKT_inequal_feas} follow from the fact that \(\vect{x}^{\ast}\) is feasible, since it is a local minimizer of the constrained problem.
  
  Note that \(\lambda_{i}^{\ast} \ge 0\) for all \(i\in\mathcal{I}\) since \(\lambda_{i} \ge 0\) for \(i\in\mathcal{A}(\vect{x}^{\ast}) \cap \mathcal{I}\) and by definition \(\lambda_{i}^{\ast} = 0\) for \(i\in\mathcal{I} \setminus \mathcal{A}(\vect{x}^{\ast})\), proving \eqref{eq:KKT_dual_feas}.

  \eqref{eq:KKT_comp_slack} follows from the definition of \(\vect{\lambda}^{\ast}\) since \(\lambda_{i}^{\ast} = 0\) whenever \(c_{i}(\vect{x}^{\ast}) > 0\), i.e. \(i \notin \mathcal{A}(\vect{x}^{\ast})\) and \(c_{i}(\vect{x}^{\ast}) = 0\) when \(i \in \mathcal{A}(\vect{x}^{\ast})\).
\end{proof}


When the \gls{kkt} conditions are satisfied at a point \(\vect{x}^{\ast}\), a (infinitesimal) move along any vector \(\vect{d} \in \mathcal{F}(\vect{x}^{\ast})\) remains feasible and either increases the first-order approximation of \(f\), if \(\vect{d}^\top \nabla f(\vect{x}^{\ast}) > 0\) (``decided'') or else keeps it constant, if \(\vect{d}^\top \nabla f(\vect{x}^{\ast}) = 0\) (``undecided'').
A decrease is not possible, as we see if we express \(\nabla f(\vect{x}^{\ast})\) as a linear combination of the active constraint gradients
% \begin{equation}\label{eq:KKT_objective_gradient}
% \vect{d}^{\top}\nabla f(\vect{x}^{\ast})
% =
% \sum_{i\in\mathcal{E}}
% \lambda_i^{\ast}
% \underbrace{\vect{d}^{\top} \nabla c_i(\vect{x}^{\ast})}_{\overset{\eqref{eq:linearized_feasible_directions}}{=} 0}
% +
% \sum_{i\in\mathcal{I}\cap\mathcal{A}(\vect{x}^{\ast})} 
% \underbrace{\lambda_i^{\ast}}_{\overset{\eqref{eq:KKT_dual_feas}}{\ge} 0}
% \underbrace{\vect{d}^{\top}\nabla c_i(\vect{x}^{\ast})}_{\overset{\eqref{eq:linearized_feasible_directions}}{\ge} 0}
% \ge 0
% \end{equation}
\newcommand{\linkeq}[1]{\mathrel{\hyperref[#1]{=}}}   % clickable “=”
\newcommand{\linkge}[1]{\mathrel{\hyperref[#1]{\ge}}} % clickable “≥”
\begin{equation}\label{eq:KKT_objective_gradient}
\vect{d}^{\top}\nabla f(\vect{x}^{\ast})
\;=\;
\sum_{i\in\mathcal{E}}
\lambda_i^{\ast}
\underbrace{\vect{d}^{\top}\nabla c_i(\vect{x}^{\ast})}_{\linkeq{eq:linearized_feasible_directions}0}
\;+\;
\sum_{\mathclap{i\in\mathcal{I}\cap\mathcal{A}(\vect{x}^{\ast})}}
\overbrace{\lambda_i^{\ast}}^{\linkge{eq:KKT_dual_feas}0}
\underbrace{\vect{d}^{\top}\nabla c_i(\vect{x}^{\ast})}_{\linkge{eq:linearized_feasible_directions}0}
\;\ge\;
0
\end{equation}
where we used Definition~\ref{def:linearized_feasible_directions} for the dot products and Condition~\eqref{eq:KKT_dual_feas} for the multipliers.






% ---------------------------------------------------------
% Second-order theory
% ---------------------------------------------------------

\subsubsection{Second-Order Conditions}

% Using the first-order Taylor expansion of \(f\) at a point \(\vect{x}\), we know that for feasible directions \(\vect{d}\) with \(\nabla f(\vect{x})^{\top} \vect{d} > 0\), the function \(f\) increases in the direction \(\vect{d}\).

% The \gls{kkt} are first-rder conditions which tell us how the first derivatives of \(f\) and the active constraints \(c_i\) are related to each other at a solution.


For undecided directions \(\vect{d}^\top \nabla f(\vect{x}^{\ast}) = 0\), we canNOT determine from first derivative information wether a move along this direction will increase or decrease \(f\).
Second-order conditions examine the second derivative terms in the Taylor expansions of \(f\) and \(c_i\), to see wether this extra information resolves the question.


\begin{definition}[Critical Cone]\label{def:critical_cone}
  For a \gls{kkt} point \( (\vect{x},\vect{\lambda}) \), the \emph{critical cone} at \((\vect{x},\vect{\lambda})\) is defined as
  \begin{equation}\label{eq:critical_cone}
  % \begin{verticalhack}
    \mathcal{C}(\vect{x},\vect{\lambda}) := \left\{ \vect{d}\in\mathcal{F}(\vect{x}) \mid \nabla c_{i}(\vect{x})^{\top} \vect{d} = 0 \text{ for all } i\in\mathcal{I}\cap\mathcal{A}(\vect{x}) \text{ with } \lambda_{i}>0\right\} % \subseteq \mathcal{F}(\vect{x}) 
  % \end{verticalhack}
  % \qedhere
  \end{equation}
  Equivalently, \(\vect{d} \in \mathcal{C}(\vect{x},\vect{\lambda})\) if and only if all of the following hold:
  \begin{itemize}
    \item \(\nabla c_{i}(\vect{x})^{\top} \vect{d} = 0\) for all \(i\in\mathcal{E}\)
    \item  \(\nabla c_{i}(\vect{x})^{\top} \vect{d} = 0\) for all \(i\in\mathcal{I}\cap\mathcal{A}(\vect{x})\) with \(\lambda_{i}>0\)
    \item \(\nabla c_{i}(\vect{x})^{\top} \vect{d} \ge 0\) for all \(i\in\mathcal{I}\cap \mathcal{A}(\vect{x})\) with \(\lambda_{i}=0\) \qedhere
  \end{itemize}
\end{definition}

From Definition~\ref{def:critical_cone}, it follows that \(\mathcal{C}(\vect{x},\vect{\lambda}) \subseteq \mathcal{F}(\vect{x})\) and, since \(\lambda_i = 0\) for inactive components \(i \in \mathcal{I}\setminus\mathcal{A}(\vect{x})\), for any \(\vect{d} \in \mathcal{C}(\vect{x},\vect{\lambda})\) we also have,
\begin{equation}\label{eq:critical_cone_active}
  \lambda_i \nabla c_i(\vect{x})^{\top} \vect{d} = 0 \quad \forall i \in \mathcal{E} \cup \mathcal{I} 
\end{equation}

The critical cone \(\mathcal{C}(\vect{x},\vect{\lambda})\) contains exactly those directions \(\vect{d} \in \mathcal{F}(\vect{x})\) for which it is not clear from first derivative information alone wether \(f\) will increase or decrease (``undecided'' directions):
\begin{equation}\label{eq:critical_cone_contains_undecided}
  \vect{d}^\top \nabla f(\vect{x})
  \overset{\eqref{eq:KKT_stationarity}}{=} 
  % \vect{d}^\top \sum _{i\in\mathcal{E}\cup\mathcal{I}} \lambda_i \nabla c_i(\vect{x}) 
  % = 
  \sum_{i\in\mathcal{E}\cup\mathcal{I}} \lambda_i \vect{d}^\top \nabla c_i(\vect{x}) 
  \overset{\eqref{eq:critical_cone_active}}{=} 
  0 \quad \forall \vect{d} \in \mathcal{C}(\vect{x},\vect{\lambda})
  \end{equation}

To find out the change of \(f\) in directions \(\vect{d} \in \mathcal{C}(\vect{x},\vect{\lambda})\), we consider the second-order derivative \eqref{eq:hessian} of \(\mathcal{L}(\vect{x},\vect{\lambda})\) with respect to \(\vect{x}\):
\[
\nabla^{2}_{\vect{x}\vect{x}} \mathcal{L}(\vect{x},\vect{\lambda}) := \matr{H}_{f}(\vect{x}) - \sum_{i\in\mathcal{E}\cup\mathcal{I}} \lambda_{i} \matr{H}_{c_{i}}(\vect{x})
\]

\begin{theorem}[Second-Order Necessary Optimality Condition]\label{thm:SONC}
If \( \vect{x}^{\ast} \) is a local minimizer, the \nameref{def:constraint_qualification_condition} holds at \(\vect{x}^{\ast}\) and \( \vect{\lambda}^{\ast} \) is the vector of Lagrange multipliers that satisfies the \gls{kkt}, then:
\begin{equation}\label{eq:SONC}
{\color{Red}
\setlength{\fboxrule}{1pt}
\boxed{ 
\color{black}
\vect{d}^{\top} \nabla^{2}_{\vect{x}\vect{x}} \mathcal{L}(\vect{x}^{\ast},\vect{\lambda}^{\ast}) \vect{d} \ge 0 \quad \forall \vect{d}\in\mathcal{C}(\vect{x}^{\ast},\vect{\lambda}^{\ast})
}
}\qedhere
\end{equation}
\end{theorem}

\begin{theorem}[Second-Order Sufficient Optimality Condition]\label{thm:SOSC}
Let \( (\vect{x}^{\ast},\vect{\lambda}^{\ast}) \) be a \gls{kkt} point. If
\begin{equation}\label{eq:SOSC}
{\color{Red}
\setlength{\fboxrule}{1pt}
\boxed{ 
\color{black}
\vect{d}^{\top} \nabla^{2}_{\vect{x}\vect{x}} \mathcal{L}(\vect{x}^{\ast},\vect{\lambda}^{\ast}) \vect{d} > 0 \quad \forall \vect{d}\in\mathcal{C}(\vect{x}^{\ast},\vect{\lambda}^{\ast}) \setminus \{ \vect{0} \}
}
}
\end{equation}
then \( \vect{x}^{\ast} \) is a strict local minimizer.
\end{theorem}
If \((\vect{x}^{\ast},\vect{\lambda}^{\ast})\) is a \gls{kkt} point such that \(\nabla^{2}_{\vect{x}\vect{x}} \mathcal{L}(\vect{x}^{\ast},\vect{\lambda}^{\ast}) \succ 0\), then \(\vect{x}^{\ast}\) is automatically a (strict) local minimizer!

Differences between Theorems~\ref{thm:SONC} and~\ref{thm:SOSC}:
\begin{itemize}
\item strict inequality in \eqref{eq:SONC}
\item absence of the \nameref{def:constraint_qualification_condition} in \eqref{eq:SOSC}
\end{itemize}


\clearpage


\section{Linear Programming}\label{sec:linear_programming}

\subsection{General and Standard Form}

\subsubsection{``Generic'' Form}\label{sec:genericform_linear_programming}

In a \gls{lp} problem, \(f\) and all \(c_i\) are affine functions.
Thus \eqref{eq:form_optimization_problem} can be written as
\begin{equation}\label{eq:form_linear_programming}
  \boxed{
%  \min_{\vect{x} \in \mathbb{R}^{n}} \vect{c}^\top \vect{x} \quad \text{subject to} \quad \begin{cases}c_{i}(\vect{x})=0, & i \in \mathcal{E} \\ c_{i}(\vect{x}) \geq 0, & i \in \mathcal{I}\end{cases}
  \min_{\vect{x} \in \mathbb{R}^{n}} \vect{c}^\top \vect{x} \quad \text{subject to} \quad
    \begin{cases}
      \vect{a}_i^\top \vect{x} + b_i = 0, & i \in \mathcal{E} \\
      \vect{a}_i^\top \vect{x} + b_i \geq 0, & i \in \mathcal{I}
    \end{cases}
     }
\end{equation}
with \(\vect{c} \in \R^{n}\), \(\vect{a}_i \in \R^{n}\), \(b_i \in \R\), \(i\in\mathcal{E}\cup\mathcal{I}\).
This is called the \emph{generic form} of a \gls{lp} problem.

\underline{Remarks:}
\begin{itemize}
\item We do not lose generality by not including a constant term in \eqref{eq:form_linear_programming} since the location is unaffected by an additive constant.
\item Since affine functions are convex (Definition~\ref{def:convex_function}), any local minimmizer of \eqref{eq:form_linear_programming} is a global minimizer.
\item \(\vect{c}^\top \vect{x}\) is often called the \emph{cost} and the value \(\vect{c}^\top \vect{x}^{\ast}\) at a minimizer \(\vect{x}^{\ast}\) the \emph{optimal cost}.
\item \nameref{sec:generalform_linear_programming} and \nameref{sec:standardform_linear_programming} are special cases of \eqref{eq:form_linear_programming}.
\end{itemize}

A \gls{lp} problem is either \emph{infeasible} ($\Omega = \emptyset$), \emph{unbounded} (the cost can be made arbitrarily small) or has \emph{at least one minimizer}.
Note that in last case, the set of minimizers could also be unbounded.


\subsubsection{General Form}\label{sec:generalform_linear_programming}
\textcolor{orange}{Transformation}  from \nameref{sec:genericform_linear_programming} to \nameref{sec:generalform_linear_programming}:
\begin{itemize}
  \item \(\vect{a}_i^\top \vect{x} + b_i = 0    \quad \longrightarrow \quad \vect{a}_i^\top \vect{x} + b_i \geq 0 \land -\vect{a}_i^\top \vect{x} - b_i \geq 0\)
  \item \(\vect{a}_i^\top \vect{x} + b_i \geq 0 \quad \longrightarrow \quad \vect{a}_i^\top \vect{x} \geq -b_i\)
\end{itemize}
% \[
% \begin{aligned}
%   \vect{a}_i^\top \vect{x} + b_i = 0    \quad &\longrightarrow \quad \vect{a}_i^\top \vect{x} + b_i \geq 0 \land -\vect{a}_i^\top \vect{x} - b_i \geq 0 \\
%   \vect{a}_i^\top \vect{x} + b_i \geq 0 \quad &\longrightarrow \quad \vect{a}_i^\top \vect{x} \geq -b_i
% \end{aligned}
% \]
Applying these transformations to \eqref{eq:form_optimization_problem}, we obtain \(m \geq |\mathcal{E} \cup \mathcal{I}|\) inequality constraints and \eqref{eq:form_linear_programming} can be written as
\begin{equation}\label{eq:form_general_linear_programming}
  {\color{Red}
\boxed{ 
\color{black}
    \min_{\vect{x} \in \mathbb{R}^{n}} \vect{c}^\top \vect{x} \quad \text{subject to} \quad
    \matr{A} \vect{x} \geq \vect{b}
  }
  }
\end{equation}
with \(\matr{A} \in \R^{m\times n}\), \(\vect{b} \in \R^{m}\) and \(\vect{c} \in \R^{n}\).
This is called the \emph{general form} of a \gls{lp} problem.
This form is important to understand the geometry of the feasible set.



% ---------------- USER INPUT ---------------------------------
\colorlet{achsYellow}{yellow!70!white}
\colorlet{vecGreen}{green!70!white}
\colorlet{planeBlue}{cyan!25}

% axis intercepts
\def\xint{5}
\def\yint{4}
\def\zint{0.8}

% normal vector starting point (x_Q, y_Q)
\def\qx{1.4} \def\qy{3}

% label position
\def\labx{3} \def\laby{2.5}

% four XY corners of the plane patch (clockwise)
\def\pax{4} \def\pay{6}
\def\pbx{6} \def\pby{-1}
\def\pcx{-1} \def\pcy{-1}
\def\pdx{-1} \def\pdy{6}

% normal vector
\def\nLength{2.4}   % |n|
% -------------------------------------------------------------

% --------------- DERIVED QUANTITIES --------------------------
% Plane coefficients from intercept form  x/\xint + y/\yint + z/\zint = 1  →  a x + b y + c z = d
% a = yint*zint,  b = xint*zint,  c = xint*yint,  d = xint*yint*zint
\pgfmathsetmacro{\aCoeff}{\yint*\zint}
\pgfmathsetmacro{\bCoeff}{\xint*\zint}
\pgfmathsetmacro{\cCoeff}{\xint*\yint}
\pgfmathsetmacro{\dConst}{\xint*\yint*\zint}

% Unit normal n = (a,b,c)/|n|
\pgfmathsetmacro{\nMag}{sqrt(\aCoeff*\aCoeff + \bCoeff*\bCoeff + \cCoeff*\cCoeff)}
\pgfmathsetmacro{\nx}{\aCoeff/\nMag}
\pgfmathsetmacro{\ny}{\bCoeff/\nMag}
\pgfmathsetmacro{\nz}{\cCoeff/\nMag}

% Q  (compute z so it lies on the plane)
\pgfmathsetmacro{\qz}{(\dConst - \aCoeff*\qx - \bCoeff*\qy)/\cCoeff}

% label position (compute z so it lies on the plane)
\pgfmathsetmacro{\labz}{(\dConst - \aCoeff*\labx - \bCoeff*\laby)/\cCoeff}

% patch‑corners z value (compute z so they lie on the plane)
\pgfmathsetmacro{\paz}{(\dConst-\aCoeff*\pax-\bCoeff*\pay)/\cCoeff}
\pgfmathsetmacro{\pbz}{(\dConst-\aCoeff*\pbx-\bCoeff*\pby)/\cCoeff}
\pgfmathsetmacro{\pcz}{(\dConst-\aCoeff*\pcx-\bCoeff*\pcy)/\cCoeff}
\pgfmathsetmacro{\pdz}{(\dConst-\aCoeff*\pdx-\bCoeff*\pdy)/\cCoeff}


\tdplotsetmaincoords{60}{20}

A set of the form
\(
  \{ \vect{x} \in \R^{n} : \vect{a}^\top \vect{x} = b \}
\)
is called a \emph{hyperplane} in \(\R^{n}\) and it separates \(\R^n\) into the two sets
\(\{\vect{x} \in \R^{n} : \vect{a}^\top \vect{x} \geq b \}\) and \(\{\vect{x} \in \R^{n} : \vect{a}^\top \vect{x} \leq b \}\), which are called \emph{halfspaces}.
\(\vect{a}\) is orthogonal to the hyperplane:
\[
\begin{tikzpicture}[tdplot_main_coords,scale=0.7,every node/.style={font=\footnotesize}]

  % axis intercepts
  \coordinate (A) at (0,0,\zint);
  \coordinate (B) at (0,\yint,0);
  \coordinate (C) at (\xint,0,0);

  % Aufpunkt Q
  \coordinate (Q) at (\qx,\qy,\qz);

  % label
  \coordinate (L) at (\labx,\laby,\labz);

  % normal vector tip
  \coordinate (nEnd)  at ($ (Q) + (\nLength *\nx, \nLength *\ny, \nLength *\nz) $);

  % plane patch corners
  \coordinate (P1) at (\pax,\pay,\paz);
  \coordinate (P2) at (\pbx,\pby,\pbz);
  \coordinate (P3) at (\pcx,\pcy,\pcz);
  \coordinate (P4) at (\pdx,\pdy,\pdz);
  
  % --- coordinate axes (first part)
  \draw[black,thick] (-0.4,0,0) -- (C);
  \draw[black,thick] (0,-0.4,0) -- (B);
  \draw[black,thick] (0,0,-0.4) -- (A);

  % --- plane patch
  \path[fill=planeBlue,opacity=0.7,draw=none] (P1) -- (P2) -- (P3) -- (P4) -- cycle;
  \node[] at (L) {$\left\{\vect x \in \R^n : \vect{a}^\top \vect{x} = b \right\}$}; % how to make this "lie in the plane"?

  % --- coordinate axes (second part)
  --- coordinate axes (first part)
\draw[black,thick] (C) -- (6,0,0) node[anchor=west]{$x_1$};
\draw[black,thick] (B) -- (0,6,0) node[anchor=east]{$x_2$};
\draw[black,thick] (A) -- (0,0,3) node[anchor=east]{$x_3$};

  % --- normal vector
    \draw[very thick,draw=vecGreen,-{>[width=5pt,length=5pt]}]  (Q) -- (nEnd)  node[anchor=west]{$\vect a$};

  % --- axis‑intercept markers
\foreach \pt/\lab in {
        (A)/{$\tfrac{b}{a_{3}}$},
        (C)/{$\tfrac{b}{a_{1}}$},
        (B)/{$\tfrac{b}{a_{2}}$}}{%
  \node[tdplot_screen_coords,circle,fill=achsYellow,draw=black,inner sep=1.0pt] at \pt {};
  \ifthenelse{\equal{\pt}{(C)}}%   <-- string test
    {\node[tdplot_screen_coords,anchor=south,font=\footnotesize] at \pt {\lab};}%
    {\node[tdplot_screen_coords,anchor=east,font=\footnotesize] at \pt {\lab};}%
}

\end{tikzpicture}
\]


The feasible set of a \gls{lp} problem thus is an intersection of \(m\) halfspaces, i.e. a set of the form \(\{ \vect{x} \in \R^{n} : \matr{A} \vect{x} \geq \vect{b} \}\).
It is a \emph{polyhedron} (Definition~\ref{def:polyhedron}).

\begin{definition}[Polyhedron]\label{def:polyhedron}
  A set of the form
  \begin{equation}\label{eq:polyhedron}
    P = \left\{ \vect{x} \in \R^{n} : \vect{a}_i^\top \vect{x} = b_i, \; i \in \mathcal{E}, \; \vect{a}_i^\top \vect{x} \geq b_i, \; i \in \mathcal{I} \right\}
  \end{equation}
  is called a {polyhedron}.
A bounded polyhedron is called a \emph{polytope}.
\end{definition}

\subsubsection{Standard Form}\label{sec:standardform_linear_programming}
The \emph{standard form} of a \gls{lp} problem is
\begin{equation}\label{eq:form_standard_linear_programming}
  {\color{Red}
\boxed{ 
\color{black}
    \min_{\vect{x} \in \mathbb{R}^{n}} \vect{c}^\top \vect{x} \quad \text{subject to} \quad
    \begin{cases}
      \matr{A} \vect{x} = \vect{b} \\
      \vect{x} \geq \vect{0}
    \end{cases}
  }}
\end{equation}
with \(\matr{A} \in \R^{m\times n}\), \(\vect{b} \in \R^{m}\) and \(\vect{c} \in \R^{n}\).
This form is important for designing algorithms.

The standard form can be interpreted as follow:
\begin{itemize}
\item the columns \(\vect{A}_1, \ldots, \vect{A}_n\) of \(\matr{A}\) are the \emph{resource vectors}
\item the non-negative variables \(x_j\) are the \emph{amount}/\emph{quantity} of resource \(j\)
\item the \(c_j\) are the \emph{unit cost} of resource \(j\)
\item the \(\vect{b}\) is the \emph{target} vector
\end{itemize}
We can therefore interpret a \gls{lp} problem in standard form as a minimal cost ``synthesis'' problem where (non-negative) quantities of some resources (ingredients) are used to get a target product (food) containing \(b_i\) of nutrient \(i\), 
where \(c_j\) is the unit cost of ingredient \(j\) and \(a_{ij}\) is the amount of nutrient \(i\) in ingredient \(j\).



\textcolor{orange}{Transformation} from \nameref{sec:standardform_linear_programming} to \nameref{sec:generalform_linear_programming} (easier):
% \begin{itemize}
%   \item \(\matr{A}    \quad \longrightarrow \quad \matr{A}'= 
%   \begin{pmatrix}
%     \matr{A} \\
%     -\matr{A} \\
%     \matr{I}_{n}
%   \end{pmatrix}\)
%   \item \(\vect{b}    \quad \longrightarrow \quad \vect{b}'= 
%   \begin{pmatrix}
%     \vect{b} \\
%     -\vect{b} \\
%     \vect{0}_{n}
%   \end{pmatrix}\)
% \end{itemize}
\[
\matr{A}' = \begin{pmatrix}
  \matr{A} \\
  -\matr{A} \\
  \matr{I}_{n}
\end{pmatrix},
\qquad
\vect{b}' = \begin{pmatrix}
  \vect{b} \\
  -\vect{b} \\
  \vect{0}_{n}
\end{pmatrix}
\]


\textcolor{orange}{Transformation} from \nameref{sec:generalform_linear_programming} to equivalent\footnote{Strictly speaking, we obtain a different problem, with a different set of variables and thus different feasible set. However, the transformed problem is equivalent to the original in the sense that it has the same optimal cost (or is infeasible, in the case the original was also infeasible).} problem in \nameref{sec:standardform_linear_programming}:
\begin{itemize}
\item non-positive variable: \(x_j \leq 0 \quad \longrightarrow \quad x_j' = -x_j \geq 0\)
\item free variable (neither \(x_j \leq 0\) nor \(x_j \geq 0\)): \(x_j \in \R \quad \longrightarrow \quad x_j = x_j^+ - x_j^-, \quad x_j^+, x_j^- \geq 0\)
\item inequality constraint: \(\vect{a}_i^\top \vect{x} \geq b_i \quad \longrightarrow \quad \vect{a}_i^\top \vect{x} - s_i = b_i, \quad s_i \geq 0\) (slack/surplus variable)
\end{itemize}


\subsection{Geometric Properties}\label{sec:geometric_properties_linear_programming}
\subsubsection{Corner Points}\label{sec:corner_points_linear_programming}
A general property of \gls{lp} problems is that minimizers can be found at ``corner points'', which we will define in 3 equivalent ways (Definition \ref{def:extreme_point}, \ref{def:vertex}, \ref{def:basic_feasible_solution}).
\begin{definition}[Extreme Point]\label{def:extreme_point}
\(\vect{x} \in P\) is an \emph{extreme point} of a polyhedron \(P\) if there exist no \(\vect{y},\vect{z} \in P \setminus \{\vect{x}\}\) and \(\lambda \in [0,1]\) such that
\(
  \vect{x} = \lambda \vect{y} + (1-\lambda) \vect{z}
\).
\end{definition}
Intuitively, this means that \(\vect{x}\) cannot be in-between two other points of \(P\).
\begin{definition}[Vertex]\label{def:vertex}
\(\vect{x} \in P\) is a \emph{vertex} of a polyhedron \(P\) if there exists \(\vect{c} \in \R^{n}\) such that
\(
  \vect{c}^\top \vect{x} < \vect{c}^\top \vect{y}
\)
for all \(\vect{y} \in P\setminus\{\vect{x}\}\).
\end{definition}
Intuitively, this means that \(\vect{x}\) is a unique solution (strict minimizer) of a \gls{lp} problem with feasible set \(P\).

Recall that \(\vect{c}^\top \vect{x} = b\) defines a family of parallel hyperplanes indexed by \(b\), and that \(\vect{c}\) points in the direction where \(\vect{c}^\top \vect{x} = b\) increases.
Thus, if \(\vect{x}^\ast\) is a vertex of \(P\), the hyperplane \(\vect{c}^\top \vect{x} = b^\ast\) where \(b^\ast = \vect{c}^\top \vect{x}^\ast\) intersects \(P\) in only one point, \(\vect{x}^\ast\), and \(b^\ast\) is the smallest possible value of parameter \(b\) such that the hyperplane intersects \(P\) in at least one point.

Definition \ref{def:extreme_point} and \ref{def:vertex} are purely geometric, i.e., they do not depend on a specific representation of the polyhedron \(P\), but they are difficult to use for finding the corner points of \(P\).
Definition \ref{def:basic_feasible_solution} uses the algebraic representation \eqref{eq:polyhedron} of the polyhedron.
It is less intuitive, but more practical, as it gives us a procedure for identifying all corner points.

Consider a polyhedron \(P\) defined in the ``generic form'' (Definition~\ref{def:polyhedron}).
Note that the general and standard from of \(P\) are special cases of this form.
\begin{definition}[Basic Solution]\label{def:basic_solution}
The vector \(\vect{x} \in \R^{n}\) is a \emph{basic solution} of \(P\) if all equality constraints are satisfied, i.e., \(\vect{a}_i^\top \vect{x} = b_i\) for all \(i \in \mathcal{E}\)
and
there are \(n\) constraints in \(\mathcal{A}(\vect{x})\) such that their gradients are linearly independent.
\end{definition}
\underline{Remarks:}
\begin{itemize}
\item If the number of constraints \(m = |\mathcal{E} \cup \mathcal{I}|\) defining \(P\) is less than \(n\), there cannot be \(n\) linearly independent constraints, thus there exists no basic solution of \(P\).
\item If we choose \(n\) linearly independent constraints, there exists a unique point \(\vect{x} \in \R^{n}\) at which they are all active (the intersection of \(n\) non-parallel hyperplanes in \(\R^{n}\) is a unique point).
      Thus, there can only be a finite number of basic solutions given a finite number of constraints and this number is bounded by \(\binom{m}{n}\). 
\item A basic solution of a polyhedron \(P\) does not necessarily belong to \(P\), since we do not require that all inequality constraints are satisfied at \(\vect{x}\).
\end{itemize}
\begin{definition}[Basic Feasible Solution]\label{def:basic_feasible_solution}
The vector \(\vect{x} \in \R^{n}\) is a \emph{basic feasible solution} of \(P\) if it is a basic solution and \(\vect{x} \in P\).
\end{definition}
\underline{Remarks:}
\begin{itemize}
  \item As mentioned, the 3 characterizations of corner points (Definition \ref{def:extreme_point}, \ref{def:vertex}, \ref{def:basic_feasible_solution}) are equivalent.
  \item Although the number of corner points is finite, it can be very large. \begin{example}[Unit Cube]\label{ex:unit_cube}
    The unit cube \(\{ \vect{x} \in \R^{n} : 0 \leq x_j \leq 1, \; j = 1,\ldots,n \}\) has \(2^n\) basic feasible solutions.
  \end{example}
  \item Not all polyhedra have basic feasible solutions (e.g. one constraint in \(\R^{n}\) with \(n > 1\)).
\end{itemize}

\begin{example}
\label{ex:basic_feasible_solutions}
Consider the polyhedron \(P \subset \R^{2}\) represented in the following figure and defined as the intersection of 4 halfspaces.
\[
\begin{tikzpicture}[scale=0.5]
  \draw[name path=L1]  (-1.0,2.5) -- (5.5,-0.5);       % ℓ₁
  \draw[name path=L2]  (-1.0,-2.5) -- (5.5,0.5);       % ℓ₂
  \draw[name path=L3]  (-0.5,3)  -- (1.5,-2);          % ℓ₃
  \draw[name path=L4]  (-0.5,-3) -- (1.5,2);           % ℓ₄

  \path [name intersections={of=L1 and L3, by=A}]; % A = ℓ₁∩ℓ₃
  \path [name intersections={of=L2 and L4, by=B}]; % B = ℓ₂∩ℓ₄
  \path [name intersections={of=L2 and L3, by=C}]; % C = ℓ₂∩ℓ₃
  \path [name intersections={of=L3 and L4, by=D}]; % D = ℓ₃∩ℓ₄
  \path [name intersections={of=L4 and L1, by=E}]; % E = ℓ₄∩ℓ₁
  \path [name intersections={of=L1 and L2, by=F}]; % F = ℓ₁∩ℓ₂

  \fill[gray!20] (C) -- (D) -- (E) -- (F) -- cycle;
  \draw              (C) -- (D) -- (E) -- (F) -- cycle;

  \draw (A) -- (E);
  \draw (B) -- (D);

  \foreach \pt/\pos in {A/left, B/left, C/right, D/left, E/right, F/above}
    {
    \fill (\pt) circle (1.8pt);
    \node[\pos] at (\pt) {\footnotesize$\pt$};
    }

  \node at ($($(D)!0.5!(F)$)!0.5!($(C)!0.5!(E)$)$) {$P$};
\end{tikzpicture}
\]
$n=2$ constraints are linearly independent if they are not parallel.
The points \(A, B, C, D, E, F\) are basic solutions but only \(C, D, E, F\) are basic feasible solutions.
\end{example}

% \subsubsection{Basic Solutions for Polyhedra in Standard Form}\label{sec:basic_solutions_standard_form}









% \clearpage

% \section{chatV3}




% % —————————————————————————–
% %  Constrained Optimisation – Part 1
% %  Concise notes after Deborah Sulem (Spring 2025)
% %  Custom macros: \vect{} for vectors, \matr{} for matrices
% % —————————————————————————–

% % —————————————————————————–
% % Generic problem statement and notation
% % —————————————————————————–
% \begin{equation}\label{eq:form_optimization_problem}
% \min_{\vect{x}\in\R^{n}} f(\vect{x})
% \quad\text{s.t.}\quad
% c_{i}(\vect{x}) = 0,\; i\in\mathcal{E},
% \qquad
% c_{i}(\vect{x}) \ge 0,\; i\in\mathcal{I}.
% \end{equation}
% Here
% \begin{itemize}
% \item $f: \R^{n}\to\R$ is \emph{continuously differentiable} (unless stated otherwise);
% \item $\vect{x}\in\R^{n}$ is the decision vector;
% \item $\mathcal{E}$ ($|\mathcal{E}| = m_{e}$) and $\mathcal{I}$ ($|\mathcal{I}| = m_{i}$) index equality and inequality constraints, respectively;
% \item each $c_{i}:\R^{n}\to\R$ is a continuous constraint function.
% \end{itemize}
% The problem is \emph{constrained} when $\mathcal{E}\cup\mathcal{I}\neq\varnothing$; otherwise (i.e.~$\mathcal{E}=\mathcal{I}=\varnothing$) it reduces to the classical \emph{unconstrained} optimization problem.

% % —————————————————————————–
% % Feasible region, active set, types of minimizers
% % —————————————————————————–
% \begin{definition}[Feasible point and feasible region]\label{def:feasible_region}
% A point $\vect{x}\in\R^{n}$ is \emph{feasible} for~\eqref{eq:form_optimization_problem} if
% \begin{align*}
% c_{i}(\vect{x}) &= 0, & \text{for } i &\in \mathcal{E},\\[1mm]
% c_{i}(\vect{x}) &\ge 0, & \text{for } i &\in \mathcal{I}.
% \end{align*}
% The set of all feasible points is the \emph{feasible region} (or set)
% \[
% \Omega := \bigl\{ \vect{x}\in\R^{n} : c_{i}(\vect{x}) = 0\; (i\in\mathcal{E}),\; c_{i}(\vect{x}) \ge 0\; (i\in\mathcal{I}) \bigr\}.
% \]
% When convenient, problem~\eqref{eq:form_optimization_problem} is written compactly as $\min_{\vect{x}\in\Omega} f(\vect{x})$.
% \end{definition}

% \begin{definition}[Active set]\label{def:active_set}
% For a feasible point $\vect{x}\in\Omega$, the \emph{active set} is
% \[
% \mathcal{A}(\vect{x}) := \mathcal{E} \cup \bigl\{ i\in\mathcal{I} : c_{i}(\vect{x}) = 0 \bigr\}.
% \]
% Equality constraints are \emph{always} active; an inequality constraint is \emph{active} when it holds with equality, otherwise it is \emph{inactive}.
% \end{definition}

% \begin{definition}[Types of minimizers]\label{def:minimizers}
% Let $\vect{x}^{\ast}\in\Omega$.
% \begin{enumerate}[label=(\alph*),leftmargin=*]
% \item $\vect{x}^{\ast}$ is a \emph{global minimizer} if $f(\vect{x}^{\ast}) \le f(\vect{x})$ for every $\vect{x}\in\Omega$.
% \item $\vect{x}^{\ast}$ is a \emph{local minimizer} if there exists $\delta>0$ such that
% \[
% f(\vect{x}^{\ast}) \le f(\vect{x})\quad\text{for all }\vect{x}\in\Omega\cap B(\vect{x}^{\ast},\delta).
% \]
% \item $\vect{x}^{\ast}$ is a \emph{strict local minimizer} if the above inequality is strict for every $\vect{x}\neq\vect{x}^{\ast}$ in the same neighbourhood.
% \end{enumerate}
% Clearly, \textit{global $\Rightarrow$ strict local $\Rightarrow$ local}.
% If $f$ and the active $c_{i}$’s are convex, the converse (local $\Rightarrow$ global) also holds.
% \end{definition}

% % —————————————————————————–
% % Feasible directions (existing content retained)
% % —————————————————————————–
% % (Content assumed to be present and correct.)

% % —————————————————————————–
% % Tangent cone, geometric first-order condition
% % —————————————————————————–
% \begin{definition}[Feasible sequence, tangent vector and tangent cone]\label{def:tangent_cone}
% \leavevmode
% \begin{enumerate}[label=(\roman*),leftmargin=*]
% \item A sequence $\{\vect{x}^{k}\}_{k\in\N}\subset\Omega$ is \emph{feasible in $\vect{x}^{\ast}$} if
% $\vect{x}^{k}\neq\vect{x}^{\ast}$, $\vect{x}^{k}\to\vect{x}^{\ast}$ and each $\vect{x}^{k}$ is feasible.
% \item A vector $\vect{d}\in\R^{n}$ is a \emph{tangent vector to $\Omega$ at $\vect{x}^{\ast}$} if there exist a feasible sequence $\{\vect{x}^{k}\}$ and positive scalars $t_k\to 0$ such that
% \[
% \lim_{k\to\infty} \frac{\vect{x}^{k}-\vect{x}^{\ast}}{t_k} = \vect{d}.
% \]
% \item The \emph{tangent cone} (or Bouligand cone) is the set of all tangent vectors:
% \[
% \mathcal{T}_{\Omega}(\vect{x}^{\ast}) := \bigl\{\vect{d}\in\R^{n} : \vect{d}\text{ is tangent to }\Omega\text{ at }\vect{x}^{\ast}\bigr\}.
% \]
% \end{enumerate}
% \end{definition}
% It is straightforward that $\mathcal{T}_{\Omega}(\vect{x}^{\ast})$ is indeed a cone, i.e., it is closed under non-negative scalar multiplication.

% \begin{lemma}\label{lem:T_F_relation}
% For any feasible $\vect{x}$ we always have $\mathcal{T}_{\Omega}(\vect{x})\subseteq\mathcal{F}(\vect{x})$. If \gls{licq} holds at $\vect{x}$, then $\mathcal{T}_{\Omega}(\vect{x}) = \mathcal{F}(\vect{x})$.
% \end{lemma}
% \begin{proof}[Idea of the proof]
% The inclusion $\mathcal{T}_{\Omega}(\vect{x})\subseteq\mathcal{F}(\vect{x})$ follows directly from the first-order Taylor expansion of each constraint along a tangent sequence. Equality under \gls{licq} uses the fact that the Jacobian of the active constraints has full row rank, which allows one (via the implicit-function theorem) to construct a feasible curve with any prescribed infinitesimal direction $\vect{d}\in\mathcal{F}(\vect{x})$.
% \end{proof}

% \begin{theorem}[Geometric first-order necessary condition]\label{thm:geom_FONC}
% If $\vect{x}^{\ast}$ is a local minimizer of~\eqref{eq:form_optimization_problem} and $f$ is continuously differentiable, then
% \[
% \nabla f(\vect{x}^{\ast})^{\top}\vect{d} \;\ge\; 0 \quad \forall\,\vect{d}\in\mathcal{T}_{\Omega}(\vect{x}^{\ast}).
% \]
% \end{theorem}
% \begin{proof}
% Suppose, by way of contradiction, that there exists a $\vect{d}\in\mathcal{T}_{\Omega}(\vect{x}^{\ast})$ with $\nabla f(\vect{x}^{\ast})^{\top}\vect{d}<0$. By the definition of the tangent cone, there is a feasible sequence $\{\vect{x}^{k}\}\to\vect{x}^{\ast}$ with corresponding positive scalars $t_k\downarrow 0$ such that $\frac{\vect{x}^{k}-\vect{x}^{\ast}}{t_k}\to\vect{d}$. A first-order Taylor expansion then gives
% \[
% f(\vect{x}^{k}) = f(\vect{x}^{\ast}) + t_k\,\nabla f(\vect{x}^{\ast})^{\top}\vect{d} + o(t_k),
% \]
% which is strictly less than $f(\vect{x}^{\ast})$ for large enough $k$, contradicting the local minimality of $\vect{x}^{\ast}$.
% \end{proof}
% Combining Lemma~\ref{lem:T_F_relation} with Theorem~\ref{thm:geom_FONC} immediately yields Lemma~\ref{lem:fundamental_condition} under the assumption of \gls{licq}.

% % —————————————————————————–
% % Proofs of fundamental optimality results
% % —————————————————————————–
% \begin{proof}[Proof of Lemma~\ref{lem:fundamental_condition}]
% Under \gls{licq}, $\mathcal{F}(\vect{x}^{\ast}) = \mathcal{T}_{\Omega}(\vect{x}^{\ast})$ by Lemma~\ref{lem:T_F_relation}; hence the result follows directly from Theorem~\ref{thm:geom_FONC}.
% \end{proof}

% \begin{proof}[Sketch proof of Theorem~\ref{thm:KKT}]
% Apply Farkas’ lemma (Lemma~\ref{lem:Farkas}) with $\matr{B}$ chosen as the gradients of the active inequalities and $\matr{C}$ as the gradients of the equalities. The geometric condition provided by Lemma~\ref{lem:fundamental_condition} forces $\nabla f(\vect{x}^{\ast})$ to lie within the cone generated by those gradients. This cone representation yields multipliers $\vect{\lambda}^{\ast}$ that satisfy the stationarity condition. Complementary slackness is then enforced by setting $\lambda_{i}^{\ast}=0$ for any inactive inequality. Uniqueness of $\vect{\lambda}^{\ast}$ follows from the linear independence (\gls{licq}) assumption.
% \end{proof}

% \begin{proof}[Sketch proof of Theorem~\ref{thm:SONC}]
% For any $\vect{d}\in\mathcal{C}(\vect{x}^{\ast},\vect{\lambda}^{\ast})$, construct a feasible sequence $\vect{x}^{k}=\vect{x}^{\ast}+t_{k}\vect{d}+o(t_{k})$ as in Lemma~\ref{lem:T_F_relation}. A second-order Taylor expansion of the Lagrangian then shows
% \[
% f(\vect{x}^{k}) = f(\vect{x}^{\ast}) + \frac{t_{k}^{2}}{2}\,\vect{d}^{\top}\nabla^{2}_{\vect{x}\vect{x}}\mathcal{L}(\vect{x}^{\ast},\vect{\lambda}^{\ast})\,\vect{d} + o(t_{k}^{2}).
% \]
% Since $f(\vect{x}^{k})\ge f(\vect{x}^{\ast})$, it follows that 
% \[
% \vect{d}^{\top}\nabla^{2}_{\vect{x}\vect{x}\,}\mathcal{L}(\vect{x}^{\ast},\vect{\lambda}^{\ast})\,\vect{d} \ge 0.
% \]
% \end{proof}

% \begin{proof}[Idea of the proof of Theorem~\ref{thm:SOSC}]
% If the quadratic form $\vect{d}^{\top}\nabla^{2}_{\vect{x}\vect{x}}\mathcal{L}(\vect{x}^{\ast},\vect{\lambda}^{\ast})\,\vect{d}$ is positive for every nonzero $\vect{d}$ in the critical cone, then the Taylor expansion provides
% \[
% f(\vect{x}^{k}) \ge f(\vect{x}^{\ast}) + \frac{\sigma}{4}\,\bigl\|\vect{x}^{k}-\vect{x}^{\ast}\bigr\|^{2}
% \]
% for all sufficiently small perturbations, which establishes that $\vect{x}^{\ast}$ is a strict local minimizer.
% \end{proof}

% % —————————————————————————–
% % Strict complementary slackness and reduced Hessian
% % —————————————————————————–
% \begin{definition}[Strict complementary slackness]\label{def:SCS}
% At a KKT point $(\vect{x}^{\ast},\vect{\lambda}^{\ast})$, the \emph{strict complementary slackness} condition holds if for each inequality index $i\in\mathcal{I}$ we have either $c_{i}(\vect{x}^{\ast})>0$ or $\lambda_{i}^{\ast}>0$ (i.e., not both equal to zero).
% \end{definition}
% When strict complementary slackness holds, the critical cone simplifies to the null space of the active-constraint Jacobian. Choosing a basis matrix $\matr{Z}\in\R^{n\times(n-m)}$ for this null space – with $m=|\mathcal{A}(\vect{x}^{\ast})|$ – the second-order conditions become
% \[
% \matr{Z}^{\top}\nabla^{2}_{\vect{x}\vect{x}}\mathcal{L}(\vect{x}^{\ast},\vect{\lambda}^{\ast})\matr{Z}\succeq 0 \quad (\text{necessary}),
% \qquad
% \matr{Z}^{\top}\nabla^{2}_{\vect{x}\vect{x}}\mathcal{L}(\vect{x}^{\ast},\vect{\lambda}^{\ast})\matr{Z}\succ 0 \quad (\text{sufficient}).
% \]
% This matrix is often referred to as the \emph{reduced Hessian}.

% % —————————————————————————–
% % Sensitivity interpretation of Lagrange multipliers
% % —————————————————————————–
% \paragraph{Economic / sensitivity interpretation.}
% Let $(\vect{x}^{\ast},\vect{\lambda}^{\ast})$ be a KKT point. Assume that $\mathcal{A}(\vect{x}^{\ast})$ remains unchanged under a small perturbation. Then, if one tightens a constraint $c_{i}$ from $0$ to $\varepsilon>0$, a first-order sensitivity analysis shows
% \[
% f_{\min}(\varepsilon) \;\approx\; f(\vect{x}^{\ast}) + \lambda_{i}^{\ast}\,\varepsilon.
% \]
% Thus, $\lambda_{i}^{\ast}$ represents the marginal cost of making constraint $i$ more stringent. Inactive inequalities have $\lambda_{i}^{\ast}=0$ because a small relaxation or tightening does not affect the optimum.

% % —————————————————————————–
% % End of document
% % —————————————————————————–




% \clearpage





% %--------------------------------------------------------------------
% % CONTINUATION — Constrained Optimisation
% %--------------------------------------------------------------------

% \section{constraint chatGPT version 3}

% \subsection{Feasible Directions}

% \begin{definition}[Feasible Direction]\label{def:feasible_direction}
% Let $\vect{x}\in\Omega$ be a feasible point.  A vector $\vect{d}\in\mathbb{R}^{n}$ is a \emph{feasible direction} at $\vect{x}$ if there exists an $\varepsilon>0$ such that
% \[
%     \vect{x}+t\vect{d}\in\Omega,\qquad \forall\,t\in(0,\varepsilon].
% \]
% In words, we can move a sufficiently small step from $\vect{x}$ along~$\vect{d}$ while staying feasible.
% \end{definition}

% \begin{remark}
% Checking Definition~\ref{def:feasible_direction} directly is usually cumbersome.  The next definition provides an equivalent \emph{first-order} characterisation that only involves the constraints that are active at~$\vect{x}$.
% \end{remark}

% \begin{definition}[Set of Linearised Feasible Directions]\label{def:linearized_feasible_directions}
% For a feasible point $\vect{x}\in\mathbb{R}^{n}$ the \emph{set of linearized feasible directions} is
% \[
%   \mathcal{F}(\vect{x})
%   :=\bigl\{
%       \vect{d}\in\mathbb{R}^{n} :
%       \nabla c_{i}(\vect{x})^{\mathsf T}\vect{d}=0 \;\forall i\in\mathcal{E},\,
%       \nabla c_{i}(\vect{x})^{\mathsf T}\vect{d}\ge 0 \;\forall i\in\mathcal{I}\cap\mathcal{A}(\vect{x})
%     \bigr\}.
% \]
% Equivalently, $\vect{d}\in\mathcal{F}(\vect{x})$ iff the first-order Taylor expansions of the active constraints remain satisfied when moving in direction~$\vect{d}$.
% \end{definition}

% \subsection{First-Order Optimality}

% \begin{lemma}[Fundamental Necessary Condition]\label{lem:fundamental_necessary}
% Suppose $\vect{x}^{\star}$ is a local minimizer of~\eqref{eq:form_optimization_problem} and a \emph{constraint qualification} (to be specified later) holds at $\vect{x}^{\star}$. Then
% \[
%     \nabla f(\vect{x}^{\star})^{\mathsf T}\vect{d}\ge 0,
%     \qquad\forall\,\vect{d}\in\mathcal{F}(\vect{x}^{\star}).
% \]
% That is, no feasible direction is a descent direction at a local minimizer.
% \end{lemma}

% \begin{proof}
% Let $\vect{d}\in\mathcal{F}(\vect{x}^{\star})$ and choose $\varepsilon>0$ such that $\vect{x}^{\star}+t\vect{d}$ is feasible for all $t\in(0,\varepsilon]$.  Because $\vect{x}^{\star}$ is a local minimizer we have $f(\vect{x}^{\star}+t\vect{d})\ge f(\vect{x}^{\star})$.  Dividing by~$t>0$ and letting $t\downarrow0$ yields the stated inequality.
% \end{proof}

% Before stating the main result we introduce the Lagrangian.

% \begin{definition}[Lagrangian]
% The \emph{Lagrangian} associated with~\eqref{eq:form_optimization_problem} is
% \[
%   \mathcal{L}(\vect{x},\boldsymbol{\lambda})
%   := f(\vect{x}) - \sum_{i\in\mathcal{E}\cup\mathcal{I}} \lambda_{i}\,c_{i}(\vect{x}),
%   \qquad \boldsymbol{\lambda}\in\mathbb{R}^{|\mathcal{E}|+|\mathcal{I}|}.
% \]
% \end{definition}

% \begin{theorem}[Karush-Kuhn-Tucker (KKT) Conditions]\label{thm:KKT}
% Let $\vect{x}^{\star}$ be a local minimizer at which a constraint qualification holds.  Then there exists a vector of multipliers $\boldsymbol{\lambda}^{\star}\in\mathbb{R}^{|\mathcal{E}|+|\mathcal{I}|}$ such that
% \begin{align}
% \nabla_{\!\vect{x}}\mathcal{L}(\vect{x}^{\star},\boldsymbol{\lambda}^{\star}) &= \boldsymbol{0}, &&\tag{KKT1}\label{eq:KKT_stationarity}\\
% c_{i}(\vect{x}^{\star}) &= 0,            &&\forall i\in\mathcal{E}, \tag{KKT2}\\
% c_{i}(\vect{x}^{\star}) &\ge 0,          &&\forall i\in\mathcal{I}, \tag{KKT3}\\
% \lambda^{\star}_{i} &\ge 0,              &&\forall i\in\mathcal{I}, \tag{KKT4}\\
% \lambda^{\star}_{i}\,c_{i}(\vect{x}^{\star}) &= 0, &&\forall i\in\mathcal{E}\cup\mathcal{I}. \tag{KKT5}\label{eq:KKT_complementarity}
% \end{align}
% Equations~\eqref{eq:KKT_stationarity}-\eqref{eq:KKT_complementarity} are called the \emph{stationarity}, \emph{primal feasibility}, \emph{dual feasibility}, and \emph{complementarity slackness} conditions, respectively.  A pair $(\vect{x}^{\star},\boldsymbol{\lambda}^{\star})$ satisfying them is a \emph{KKT point}.
% \end{theorem}

% \begin{proof}[Sketch of proof]
% We employ two ingredients.
% \begin{enumerate}[label=\arabic*., wide, labelwidth=!]
% \item The fundamental necessary condition (Lemma~\ref{lem:fundamental_necessary}).
% \item Farkas' lemma (Lemma~\ref{lem:Farkas}).
% \end{enumerate}
% Writing $\mathcal{F}(\vect{x}^{\star})$ as a polyhedral cone generated by the gradients of the active constraints and applying Farkas' lemma to $\nabla f(\vect{x}^{\star})$, one obtains the existence of multipliers $\boldsymbol{\lambda}^{\star}$ fulfilling the stationarity and sign conditions.  Complementarity follows by construction; see any standard reference on nonlinear programming for details.
% \end{proof}

% \begin{lemma}[Farkas' Lemma]\label{lem:Farkas}
% Let $B\in\mathbb{R}^{n\times m}$, $C\in\mathbb{R}^{n\times p}$ and $\vect{g}\in\mathbb{R}^{n}$.  Exactly one of the following statements is true:
% \begin{enumerate}[label=\textbf{F\arabic*.}]
% \item $\vect{g}=B\vect{y}+C\vect{w}$ for some $\vect{y}\ge\boldsymbol{0}$, $\vect{w}\in\mathbb{R}^{p}$. \label{item:F1}
% \item There exists $\vect{d}\in\mathbb{R}^{n}$ such that $\vect{g}^{\mathsf T}\vect{d}<0$, $B^{\mathsf T}\vect{d}\ge\boldsymbol{0}$ and $C^{\mathsf T}\vect{d}=\boldsymbol{0}$. \label{item:F2}
% \end{enumerate}
% The two alternatives are mutually exclusive and exhaustive.
% \end{lemma}

% \subsection{Constraint Qualification}

% \begin{definition}[Linear Independence Constraint Qualification — \gls{licq}]\label{def:licq}
% \gls{licq} is said to hold at a feasible point $\vect{x}$ if the set of active constraint gradients $\bigl\{\nabla c_{i}(\vect{x}) : i\in\mathcal{A}(\vect{x})\bigr\}$ is linearly independent.
% \end{definition}

% \begin{remark}
% \begin{enumerate}[label=(\roman*)]
% \item If all constraint functions are affine, \gls{licq} is satisfied everywhere.
% \item When \gls{licq} holds at $\vect{x}^{\star}$ the multipliers $\vect{\lambda}^{\star}$ solving Theorem~\ref{thm:KKT} are unique.
% \end{enumerate}
% \end{remark}

% \subsection{Second-Order Optimality}

% For second-order results we need the \emph{critical cone} at a KKT point.

% \begin{definition}[Critical Cone]\label{def:critical_cone}
% Let $(\vect{x}^{\star},\boldsymbol{\lambda}^{\star})$ be a KKT point.  The \emph{critical cone} is
% \[
%   \mathcal{C}(\vect{x}^{\star},\boldsymbol{\lambda}^{\star})
%     := \Bigl\{ \vect{d}\in\mathcal{F}(\vect{x}^{\star}) :
%          \nabla c_{i}(\vect{x}^{\star})^{\mathsf T}\vect{d}=0
%          \ \forall i\in\mathcal{I}\cap\mathcal{A}(\vect{x}^{\star})\text{ with }\lambda^{\star}_{i}>0
%        \Bigr\}.
% \]
% \end{definition}

% \begin{definition}[Lagrangian Hessian]
% The \emph{Hessian of the Lagrangian} at $(\vect{x},\boldsymbol{\lambda})$ is
% \[
%     \nabla^{2}_{\!\vect{x}\vect{x}}\mathcal{L}(\vect{x},\boldsymbol{\lambda})
%     = \nabla^{2}f(\vect{x})
%       - \sum_{i\in\mathcal{E}\cup\mathcal{I}} \lambda_{i}\,\nabla^{2}c_{i}(\vect{x}).
% \]
% \end{definition}

% \begin{theorem}[Second-Order Necessary Condition]\label{thm:SONC}
% Under the assumptions of Theorem~\ref{thm:KKT}, let $(\vect{x}^{\star},\boldsymbol{\lambda}^{\star})$ be a KKT point.  Then
% \[
%     \vect{d}^{\mathsf T}\nabla^{2}_{\!\vect{x}\vect{x}}\mathcal{L}(\vect{x}^{\star},\boldsymbol{\lambda}^{\star})\,\vect{d}\ge 0,
%     \qquad \forall\,\vect{d}\in\mathcal{C}(\vect{x}^{\star},\boldsymbol{\lambda}^{\star}).
% \]
% \end{theorem}

% \begin{theorem}[Second-Order Sufficient Condition]\label{thm:SOSC}
% Let $(\vect{x}^{\star},\boldsymbol{\lambda}^{\star})$ be a KKT point.  If
% \[
%     \vect{d}^{\mathsf T}\nabla^{2}_{\!\vect{x}\vect{x}}\mathcal{L}(\vect{x}^{\star},\boldsymbol{\lambda}^{\star})\,\vect{d}>0,
%     \qquad \forall\,\vect{d}\in\mathcal{C}(\vect{x}^{\star},\boldsymbol{\lambda}^{\star})\setminus\{\boldsymbol{0}\},
% \]
% then $\vect{x}^{\star}$ is a \emph{strict} local minimizer of~\eqref{eq:form_optimization_problem}.
% \end{theorem}

% \begin{remark}
% If $\nabla^{2}_{\!\vect{x}\vect{x}}\mathcal{L}(\vect{x}^{\star},\boldsymbol{\lambda}^{\star})\succ\boldsymbol{0}$ (positive definite) on \emph{all} directions in $\mathbb{R}^{n}$, the sufficient condition is automatically satisfied.
% \end{remark}








% \printglossary[type=\acronymtype,title={List of Acronyms}]








% \clearpage

% \section{Exam Topics}
% \begin{itemize}[nosep, itemsep=0pt, parsep=0pt, before={\parskip=0pt}]
%   \item In particular, revise:
%   \begin{itemize}[nosep, itemsep=0pt, parsep=0pt, before={\parskip=0pt}]
%     \item Optimality conditions
%     \item Definitions of directional derivative and descent directions
%     \item Definitions and properties of convex, strongly convex, Lipschitz and smooth functions
%     \item Properties of multivariate quadratic functions
%     \item Wolfe conditions
%   \end{itemize}
%   \item Do you know how to recognise a convex function? A smooth function?
%   \item Do you know how to recognise that a point is a local minimizer?
%   \item Do you know what is a descent direction?
%   \item Do you know what is a quadratic function and in which cases it has a global minimizer?
%   \item Do you know how to apply Gradient Descent and Newton's algorithms?
%   \item Do you know how to apply the Wolfe conditions?
%   \item Is a square matrix with all entries equal to 0 positive-semidefinite? Positive-definite?
%   \item What are the eigenvalues of the following matrix? Its condition number? Its spectral norm? Its inverse? The eigenvalues of its inverse?
%   \[
%   \matr{A}=\begin{pmatrix}
%   1 & 0 & 0 \\
%   0 & 2.5 & 0 \\
%   0 & 0 & 0.5
%   \end{pmatrix}
%   \]
% \end{itemize}





% \section*{Answers}

% \begin{description}[leftmargin=1.5cm, style=nextline]

%   \item[Q1: What are the optimality conditions?] 
%   \textbf{A:} 
%   \begin{itemize}[leftmargin=2em]
%     \item \textbf{First-Order Necessary Condition:} For an unconstrained problem 
%       \[
%       \min_{\vect{x}} f(\vect{x}),
%       \]
%       a necessary condition for \(\vect{x}^*\) to be a local minimizer is 
%       \[
%       \nabla f(\vect{x}^*) = 0.
%       \]
%     \item \textbf{Second-Order Necessary Condition:} If \(\nabla f(\vect{x}^*) = 0\), a necessary condition is that the Hessian is positive semidefinite:
%       \[
%       \nabla^2 f(\vect{x}^*) \succeq 0.
%       \]
%     \item \textbf{Second-Order Sufficient Condition:} If \(\nabla f(\vect{x}^*) = 0\) and 
%       \[
%       \nabla^2 f(\vect{x}^*) \succ 0 \quad (\text{positive definite}),
%       \]
%       then \(\vect{x}^*\) is a strict local minimizer.
%   \end{itemize}

%   \item[Q2: What are the definitions of the directional derivative and descent directions?]
%   \textbf{A:}
%   \begin{itemize}[leftmargin=2em]
%     \item \textbf{Directional Derivative:} For a function \(f: \mathbb{R}^n \to \mathbb{R}\), the directional derivative at \(\vect{x}\) in the direction \(\vect{d}\) is defined as
%       \[
%       f'(\vect{x}; \vect{d}) = \lim_{t \to 0} \frac{f(\vect{x} + t \,\vect{d}) - f(\vect{x})}{t},
%       \]
%       provided the limit exists.
%     \item \textbf{Descent Direction:} A vector \(\vect{d}\) is a descent direction at \(\vect{x}\) if
%       \[
%       \nabla f(\vect{x})^T \,\vect{d} < 0.
%       \]
%       This implies that for sufficiently small \(t > 0\), \(f(\vect{x} + t \,\vect{d}) < f(\vect{x})\).
%   \end{itemize}

%   \item[Q3: What are the definitions and properties of convex, strongly convex, Lipschitz, and smooth functions?]
%   \textbf{A:}
%   \begin{itemize}[leftmargin=2em]
%     \item \textbf{Convex Function:} \(f\) is convex if for all \(\vect{x}, \vect{y}\) and \(\lambda \in [0,1]\),
%       \[
%       f(\lambda \,\vect{x} + (1-\lambda) \,\vect{y}) \le \lambda \,f(\vect{x}) + (1-\lambda) \,f(\vect{y}).
%       \]
%       For differentiable functions, this is equivalent to:
%       \[
%       f(\vect{y}) \ge f(\vect{x}) + \nabla f(\vect{x})^T (\vect{y} - \vect{x}) \quad \forall \,\vect{x},\vect{y}.
%       \]
%     \item \textbf{Strongly Convex Function:} \(f\) is strongly convex with parameter \(\mu > 0\) if
%       \[
%       f(\vect{y}) \ge f(\vect{x}) + \nabla f(\vect{x})^T (\vect{y} - \vect{x}) + \frac{\mu}{2} \|\vect{y} - \vect{x}\|^2.
%       \]
%       Equivalently, its Hessian satisfies
%       \[
%       \nabla^2 f(\vect{x}) \succeq \mu \matr{I} \quad \text{for all } \vect{x}.
%       \]
%     \item \textbf{Lipschitz Continuity:} A function \(f\) (or its gradient) is Lipschitz continuous if there exists a constant \(L \ge 0\) such that
%       \[
%       \|f(\vect{x}) - f(\vect{y})\| \le L \,\|\vect{x} - \vect{y}\|,
%       \]
%       or for gradients,
%       \[
%       \|\nabla f(\vect{x}) - \nabla f(\vect{y})\| \le L \,\|\vect{x} - \vect{y}\|.
%       \]
%     \item \textbf{Smooth Function:} A function is \(L\)-smooth if it is differentiable and its gradient is Lipschitz continuous with constant \(L\); that is,
%       \[
%       f(\vect{y}) \le f(\vect{x}) + \nabla f(\vect{x})^T (\vect{y}-\vect{x}) + \frac{L}{2}\|\vect{y}-\vect{x}\|^2.
%       \]
%   \end{itemize}

%   \item[Q4: What are the properties of multivariate quadratic functions?]
%   \textbf{A:}
%   \begin{itemize}[leftmargin=2em]
%     \item \textbf{General Form:} A quadratic function in \(\mathbb{R}^n\) can be expressed as
%       \[
%       f(\vect{x}) = \frac{1}{2} \,\vect{x}^T \matr{A} \,\vect{x} + \vect{b}^T \,\vect{x} + c,
%       \]
%       where \(\matr{A}\) is a symmetric matrix, \(\vect{b}\) is a vector, and \(c\) is a scalar.
%     \item \textbf{Convexity:} The function is convex if \(\matr{A} \succeq 0\) (i.e., \(\matr{A}\) is positive semidefinite) and strictly convex if \(\matr{A} \succ 0\) (i.e., \(\matr{A}\) is positive definite).
%     \item \textbf{Global Minimiser:}
%       \begin{itemize}
%         \item If \(\matr{A} \succ 0\), then the unique global minimizer is
%           \[
%           \vect{x}^* = -\,\matr{A}^{-1}\,\vect{b}.
%           \]
%         \item If \(\matr{A} \succeq 0\) but is singular, a global minimizer exists if and only if \(\vect{b}\) lies in the range (column space) of \(\matr{A}\); however, the minimizer may not be unique.
%       \end{itemize}
%   \end{itemize}

%   \item[Q5: What are the Wolfe conditions?]
%   \textbf{A:} The Wolfe conditions are used in line search methods to choose a step size \(\alpha\) and consist of:
%   \begin{itemize}[leftmargin=2em]
%     \item \textbf{Armijo (Sufficient Decrease) Condition:}
%       \[
%       f(\vect{x}_k + \alpha \,\vect{d}_k) \le f(\vect{x}_k) + c_1 \,\alpha \,\nabla f(\vect{x}_k)^T \vect{d}_k,
%       \]
%       where \(0 < c_1 < 1\).
%     \item \textbf{Curvature Condition:}
%       \[
%       \nabla f(\vect{x}_k + \alpha \,\vect{d}_k)^T \vect{d}_k \,\ge\, c_2 \,\nabla f(\vect{x}_k)^T \vect{d}_k,
%       \]
%       with \(c_1 < c_2 < 1\).
%   \end{itemize}

%   \item[Q6: How do you recognise a convex function? A smooth function?]
%   \textbf{A:}
%   \begin{itemize}[leftmargin=2em]
%     \item \textbf{Convex Functions:} Verify that the line segment between any two points on the graph lies above the graph; for differentiable functions, a nonnegative second derivative (or positive semidefinite Hessian) confirms convexity.
%     \item \textbf{Smooth Functions:} These are functions that are continuously differentiable, often with a Lipschitz continuous gradient.
%   \end{itemize}

%   \item[Q7: How do you recognise that a point is a local minimizer?]
%   \textbf{A:} A point \(\vect{x}^*\) is a local minimizer if:
%   \begin{itemize}[leftmargin=2em]
%     \item \textbf{First-Order Test:} \(\nabla f(\vect{x}^*) = 0\).
%     \item \textbf{Second-Order Test:} The Hessian \(\nabla^2 f(\vect{x}^*)\) is positive definite (or at least positive semidefinite with additional conditions to rule out saddle points).
%     \item There exists a neighborhood \(U\) of \(\vect{x}^*\) such that \(f(\vect{x}^*) \le f(\vect{x})\) for all \(\vect{x} \in U\).
%   \end{itemize}

%   \item[Q8: What is a descent direction?]
%   \textbf{A:} A descent direction \(\vect{d}\) at a point \(\vect{x}\) satisfies:
%   \[
%   \nabla f(\vect{x})^T \vect{d} < 0,
%   \]
%   ensuring that a sufficiently small move in the direction \(\vect{d}\) decreases the function value.

%   \item[Q9: What is a quadratic function and in which cases does it have a global minimizer?]
%   \textbf{A:}
%   \begin{itemize}[leftmargin=2em]
%     \item \textbf{Quadratic Function Form:}
%       \[
%       f(\vect{x}) = \frac{1}{2} \,\vect{x}^T \matr{A} \,\vect{x} + \vect{b}^T \,\vect{x} + c.
%       \]
%     \item \textbf{Global Minimiser Existence:}
%       \begin{itemize}
%         \item If \(\matr{A} \succ 0\), the function is strictly convex and the unique global minimizer is
%           \[
%           \vect{x}^* = -\,\matr{A}^{-1}\,\vect{b}.
%           \]
%         \item If \(\matr{A} \succeq 0\) (but not positive definite), a global minimizer exists if \(\vect{b}\) lies in the range of \(\matr{A}\); otherwise, the function is unbounded below. The minimizer may not be unique in that case.
%       \end{itemize}
%   \end{itemize}

%   \item[Q10: How do you apply Gradient Descent and Newton's algorithms?]
%   \textbf{A:}
%   \begin{itemize}[leftmargin=2em]
%     \item \textbf{Gradient Descent:} Update iteratively via
%       \[
%       \vect{x}_{k+1} = \vect{x}_k - \alpha_k \,\nabla f(\vect{x}_k),
%       \]
%       where the step size \(\alpha_k\) can be determined by a line search (often using Wolfe or Armijo conditions).
%     \item \textbf{Newton's Method:} Update using second-order information:
%       \[
%       \vect{x}_{k+1} = \vect{x}_k - \bigl[\nabla^2 f(\vect{x}_k)\bigr]^{-1} \,\nabla f(\vect{x}_k).
%       \]
%       This method exhibits quadratic convergence near the minimizer provided the Hessian is positive definite and the initial guess is close enough.
%   \end{itemize}

%   \item[Q11: How do you apply the Wolfe conditions during a line search?]
%   \textbf{A:}
%   \begin{enumerate}[leftmargin=2em]
%     \item Choose an initial step size \(\alpha\) and compute the trial point \(\vect{x}_k + \alpha \,\vect{d}_k\).
%     \item Verify the \textbf{Armijo condition}:
%       \[
%       f(\vect{x}_k + \alpha \,\vect{d}_k) \le f(\vect{x}_k) + c_1 \,\alpha \,\nabla f(\vect{x}_k)^T \vect{d}_k.
%       \]
%     \item Check the \textbf{Curvature condition}:
%       \[
%       \nabla f(\vect{x}_k + \alpha \,\vect{d}_k)^T \vect{d}_k \,\ge\, c_2 \,\nabla f(\vect{x}_k)^T \vect{d}_k.
%       \]
%     \item If both conditions are met, accept \(\alpha\); otherwise, adjust (typically reduce) \(\alpha\) and repeat.
%   \end{enumerate}

%   \item[Q12: Is a square matrix with all entries equal to 0 positive-semidefinite? Positive-definite?]
%   \textbf{A:}
%   \begin{itemize}[leftmargin=2em]
%     \item \textbf{Positive-Semidefinite:} Yes, because for any vector \(\vect{x}\),
%       \[
%       \vect{x}^T \matr{0}\,\vect{x} = 0 \ge 0.
%       \]
%     \item \textbf{Positive-Definite:} No, since for any nonzero \(\vect{x}\),
%       \[
%       \vect{x}^T \matr{0}\,\vect{x} = 0 \quad (\text{not strictly greater than } 0).
%       \]
%   \end{itemize}

%   \item[Q13: Given the matrix 
%     \(
%     \matr{A} = \begin{pmatrix}
%     1 & 0 & 0 \\
%     0 & 2.5 & 0 \\
%     0 & 0 & 0.5
%     \end{pmatrix},
%     \)
%     what are its eigenvalues, condition number, spectral norm, inverse, and the eigenvalues of its inverse?]
%   \textbf{A:}
%   \begin{itemize}[leftmargin=2em]
%     \item \textbf{Eigenvalues:} They are the diagonal entries:
%       \[
%       \lambda_1 = 1,\quad \lambda_2 = 2.5,\quad \lambda_3 = 0.5.
%       \]
%     \item \textbf{Condition Number:} The 2-norm condition number is given by
%       \[
%       \kappa(\matr{A}) = \frac{2.5}{0.5} = 5.
%       \]
%     \item \textbf{Spectral Norm:} This is the largest absolute eigenvalue:
%       \[
%       \|\matr{A}\|_2 = 2.5.
%       \]
%     \item \textbf{Inverse of \(\matr{A}\):} Since \(\matr{A}\) is diagonal,
%       \[
%       \matr{A}^{-1} = \begin{pmatrix}
%       1 & 0 & 0 \\
%       0 & \frac{1}{2.5} & 0 \\
%       0 & 0 & \frac{1}{0.5}
%       \end{pmatrix} 
%       = \begin{pmatrix}
%       1 & 0 & 0 \\
%       0 & 0.4 & 0 \\
%       0 & 0 & 2
%       \end{pmatrix}.
%       \]
%     \item \textbf{Eigenvalues of \(\matr{A}^{-1}\):} These are the reciprocals of the eigenvalues of \(\matr{A}\):
%       \[
%       1,\quad 0.4,\quad 2.
%       \]
%   \end{itemize}

% \end{description}








\end{document}